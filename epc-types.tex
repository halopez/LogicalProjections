\subsection{Session Types for the End-Point Calculus}
\label{Logic4Struct-EndPointTypes}
Session types for the EPC use the syntax of session types in
equation \ref{Logic4Struct:eq:sessiontypes}. Basically, the type
discipline of the EPC stems from the Global Calculus, but assigns
session types to every single participant instead of the whole
choreography. In this way, the session typing in the EPC describes the
end-point behaviour.
% \begin{definition}[
% End-Point Typing Judgements
  An {\it end-point typing judgment} contains judgements for processes in  the form $\tprovesA{\Gamma}{P}{\Delta}$ (where $P$ is
    typed as a behaviour for $A$) and judgements for networks
    $\tproves{\Gamma}{N}{\Delta}$. In both,  mappings $\Gamma$ and $\Delta$ are {\em service} and {\em session typings} respectively.
%    triple on processes or
%   networks:\\
%   \begin{tabular}{ll}
%     (Processes) & $\tprovesA{\Gamma}{P}{\Delta}$ (where $P$ is
%     typed as a behaviour for $A$)\\
%     (Networks) & $\tproves{\Gamma}{N}{\Delta}$.  
%   \end{tabular}\\
%   where the mappings $\Gamma$ and $\Delta$ are the {\em service} and the {\em session typing} respectively.
% \end{definition}
Here, $\Gamma$ 
% (service typing)
 and $\Delta$ 
% (session typing) 
are defined
as:
\begin{align*}
&\Gamma  \ \; ::=\;\ \emptyset
                  \;\ |\;\ \Gamma, a@A\!:\!(\VEC{k})\alpha
                  \;\ |\;\ \Gamma, \ol{a}@A\!:\!(\VEC{k})\alpha
                  \;\ |\;\ \Gamma, x@A\!:\!\theta
                  \;\ |\;\ \Gamma, X\!:\!\Delta
%\qquad
\\
&\Delta  \ \; ::=\;\ \emptyset
              \;\ |\;\ \Delta, \VEC{k}@A\!:\!\alpha
              \;\ |\;\ \Delta, \VEC{k}\!:\!\bot
\end{align*}

Above, $a@A\!:\!(\VEC{k})\alpha$ indicates the service located at $A$
which is invoked with fresh session channels $\VEC{k}$ and offers
service of the shape $\alpha$, while $\ol{a}@A\!:\!(\VEC{k})\alpha$
indicates the type abstraction for the dual invocation, i.e. a client
of a service offered by $A$ which invokes with fresh channels $\VEC{k}$ and
engages in interactions abstracted as $\alpha$. Note $@A$ indicates
the location of a service in both forms. 
As before, $\VEC{k}$ should be a vector of pairwise distinct session
channels which should cover all session channels in $\alpha$, and
$\alpha$ does not contain free type variables. $(\VEC{k})$ binds
occurrences of session channels in $(\VEC{k})$ in $\alpha$, which
induces the standard alpha-equality. A central concept in this type
discipline is the notion of duality for session types, which is defined as:
\[
\ol{\!(\VEC{k})\alpha}@A = ? (\VEC{k}) \ol{\alpha}@A ~~~~~~
\ol{? (\VEC{k}) \alpha@A} = \! (\VEC{k}) \ol{\alpha}@A 
\]
where the notion of duality $\alpha$ of $\alpha$ remains the same.




The typing rules are almost identical as the ones from the original
presentation of the EPC \cite{carbone7scc}, where the only difference
lies on the separation between input-output types and
selection-branching types as originally presented in
\cite{honda1998lpa}. Here we only comment some examples on the typing
rules, and the full type system can be found in Appendix
\ref{Logic4Struct:appendix:EPC-Types}. As in our account of the type
system for the Global Calculus, we here devote our attention to the
rules for session initiation and communication.  The two rules
\Did{E-TInit.In},\Did{E-TInit.Out} describe session initiation
primitives:


\begin{align*}
  \myrule{ E-TInit.In} 
  { \typeruleE{\typeEnv }{A}{ P \ccc
      \VEC{k}@A:\alpha} 
    \quad a \not\in dom(\typeEnv) 
    \quad    client(\typeEnv) 
  } { 
    \typeruleE{\typeEnv, a @ A: (\VEC{k}) \alpha }{A}{
      !a(\VEC{k}) \pfx P \ccc \emptyset}
  }
  \quad 
  \myrule{ E-TInit.Out} 
  { \typeruleE{\typeEnv, a@B:(\VEC{k})\alpha}{A}{ P \ccc \Delta \cdot
      \VEC{k}@A:\alpha}  } 
  { \typeruleE{\typeEnv, a @B: (\VEC{k})\alpha }{A}{ \overline{a}
      <. \VEC{k} .> P \ccc \Delta}}
\end{align*}

In \Did{E-TInit.In}, the premise only allows for typings of session
channels involved in the session initialisation of service $a$, that
is,  only the channels in $\VEC{k}$. This linearity condition blocks
free session channels from occurring during a replicated input. The
condition $a \not\in dom(\typeEnv)$ prevents from self-calls and
ensures that the type assignment occurs at the side of the client.
Requirements for the complementary typing rule \Did{E-TInit.In} are
analogous, although the linearity condition is removed. Communication
rules are standard for session types, for instance, the rule
\Did{E.TOut} is used to type message outputs:

\begin{align*}
  \myrule{    E.TOut}
    { 
      k \in \VEC{k} \quad
      \typeruleE{\typeEnv  }{A}{ 
        P  \ccc \Delta \cdot \VEC{k}@A:\alpha} \quad 
      \typeruleE{\typeEnv  }{}{ e : \theta} 
    } 
    { \typeruleE{\typeEnv  }{A}{
        \send{k}{e}  P  \ccc \Delta
      \cdot \VEC{k}@A: k \uparrow \theta
    \pfx \alpha}}
\end{align*}

Here, process $\send{k} {e}  P$ types after evaluation that the typing
of $e$ corresponds to a correct value type and that the continuation
$P$ behaves as established by the session type  in $\Delta \cdot
\VEC{k}@A:\alpha$. Analogous requirements hold for typing the input
process $ \receive{k}{x}  P$.

% \begin{theorem}\label{theorem:epc:sr}\

%   \begin{enumerate}
    
%   \item (Subject Congruence) If $\tproves{\Gamma}{M}{\Delta}$ and
%     $M\LITEQ N$ then $\tproves{\Gamma}{N}{\Delta}$;

%   \item (Subject Reduction) If $\Gamma\vdash N\rhd\Delta$ and
%     $N\rightarrow N'$ then $\Gamma\vdash N'\rhd\Delta$.

%   \end{enumerate}

% \end{theorem}


%%%%%%%%%%%%%%%%%%%%%%%%%%%%%%%%%%%%%%%%%%%%%%%%%%%%
%%% Local Variables: 
%%% mode: latex
%%% TeX-master: "../Thesis"
%%% End: 
%%%%%%%%%%%%%%%%%%%%%%%%%%%%%%%%%%%%%%%%%%%%%%%%%%%%
