\subsection{End Point Projection}
\label{Logic4Struct:sec:epp-types}


The relation between global and local views at the specification of
communication protocols is given at the level of types. The central
idea is that one can \emph{project} the behaviour (type) of a
global specification given in terms of choreography into a parallel
composition of the behaviours of end-points. The mapping is far from
trivial, and needs to preserve causal relations between messages and
threads, namely \emph{connectedness}, \emph{well-threadedness} and
\emph{coherence}. The next subsection
presents a recap from the work at \cite{carbone7scc}. We will use
these definitions (specially Theorem \ref{Logic4Struct:theorem:epp}
and Definition \ref{Logic4Struct:definition:pruning}) in order to
relate the work on end-point projections with their corresponding
logical counterpart.
In order to give the formal definition of end point projection, we
first annotate global specifications with identifiers for threads.

% \begin{definition}[Annotated Interactions]
  An annotated interaction is an annotation of a choreography with
  $t$'s denoting each thread in play. Annotated interactions are
  written $\mathcal{A,A', ...}$, and they are given by
  the following grammar:

  \begin{align*} 
    \CAL{A} ::= & ~~~\init{A^{t_1}}{ B ^{t_2}}{a}{k} \pfx \CAL{A}
    && |~~\CAL{A}_1 |^t \CAL{A}_2 \\
    & |~~\interact{A^{t_1}}{ B ^{t_2}}{k}{e}{y} \pfx \CAL{A}
    && |~~\mu^t X^A \pfx \CAL{A}\\
    & |~~\choice{A^{t_1}}{B ^{t_2}}{k}{l}{\CAL{A}}
    && |~~ X^A_t \\
    & |~~\itn{e@A^{t}}{\CAL{A}_1}{\CAL{A}_2}
    && |~~\INACT
\end{align*}
where each $t$ is a natural number. We call $t,t',\cdots$ occurring in
an annotated interaction, \emph{threads}. Each $\CAL{A}$ can be
regarded as an abstract syntax built from a constructor in its root
(either a prefix or a parallel product), if the tree is originated
from a single thread, or a pair of threads if the interaction involves
an interaction (session initiation, message communication or
selection/branching). The following is the consistent annotation of
(\ref{Logic4Struct:example:syntax}).


\begin{align}
      & \init{\text{Cust}^1}{\text{AC}^2}{\text{ob}}{k_1,k_2} \pfx
    \interact{\text{Cust}^1}{\text{AC}^2}{k_1}{\text{booking}}{x} \pfx \tag{$OB_{\CAL{A}}$} \\
     & ~~~~~~\interact{\text{AC}^2}{\text{Cust}^1}{k_2}{\text{offer}}{y} \pfx
    \interact{\text{Cust}^1}{\text{AC}^2}{k_1}{\text{accept}}{z} \pfx
    \INACT \notag 
\end{align}

This annotation, though simple, would become more complicated if there
more than one session initiation was involved in the choreography.
 Take for instance the case where
$\init{\text{Cust}}{\text{AC}}{\text{ob}}{k_1,k_2}$ is decomposed by
the sequence of processes
$\init{\text{Cust}}{\text{AC}}{\text{ob}}{k_1}\pfx$ $
\init{\text{AC}}{\text{Cust}}{\text{ob}}{k_2}$ We can have different
annotations for Cust and AC. The sequence:
$\init{\text{Cust}^1}{\text{AC}^2}{\text{ob}}{k_1}\pfx
\init{\text{AC}^2}{\text{Cust}^3}{\text{ob}}{k_2}$ generates a valid
annotation as it places each session initiation between the customer
and the AC in different threads, and any other annotation would be
invalid.
% \end{definition}


A choreography $\chor$ is \emph{connected}, if in communication
actions, only the message reception leads to activity (at the
receiving participant), and that such activity should immediately
follow the reception of messages. 
Informally, for each participant $A$ in the set of
participants of a choreography $\chor$, a communication activity
originated by $A$ should have been immediately preceeded by a communication
activity where $A$ had acted as a receiver, or been preceded by a
self-contained action (evaluation of expressions, for instance).

% With respect to threads, it is convenient to introduce some notation: $t_1$
%     (resp. $t_2$) is the \emph{active thread} of $\CAL{A}$ by $B$ (resp.
%     the \emph{passive thread} of $\CAL{B}$ by $C$) if the root of $\CAL{A}$
%     is initialisation/communication from 
%     $B$ to $C$ and it is annotated by $(t_1 , t_2 )$. We define
%     predecessors and successors in annotated interactions as normally
%     defined in ASTs: A direct predecessor is a predecessor which has no
%     intermediate predecessor. Symmetrically we define successor and
%     direct successor. 
% \begin{definition}[Threads]
%   \begin{enumerate}
%   \item If the root of $\CAL{A}$ is initialisation/communication from
%     $B$ to $C$ and is annotated by $(t_1 , t_2 )$, then $t_1$
%     (resp. $t_2$) is the active thread of $\CAL{A}$ by $B$ (resp.
%     the passive thread of $\CAL{B}$ by $C$). If the root of $\CAL{A}$ is
%     another constructor then its annotation $t$ is both its active
%     thread and its passive thread.
%   \item If $\CAL{A'}$ occurs as a proper subtree of $\CAL{A}$, then
%     (the root of) $\CAL{A}$ is a predecessor of (the root of)
%     $\CAL{A'}$. A direct predecessor is a predecessor which has no
%     intermediate predecessor. Symmetrically we define successor and
%     direct successor. 
% \end{enumerate}
% \end{definition}


\paragraph{Consistent annotations} In order to provide meaningful
projections between choreographies and its end-points, we need to
define a notion of ``consistent annotation'', that is, an annotation
$\CAL{A}$ such that it respects causality conditions, and can be
realised by a projection.
% \begin{definition}[Consistently Annotated Interactions] \label{Logic4Struct:def:cons-annotated-ints}
%   A well-typed annotated connected interaction (choreography)
%   $\mathcal{A}$ is consistent if the following conditions hold for
%   each of its subtrees, say $\mathcal{A'}$:
Such conditions are: 1) Causal Consistency: if a participant annotated
with $t$ is passive in an interaction (a receiver), then the
subsequent interaction will be marked with $t$ as well, or it will be
a self-contained action, 2) Session Consistency: Two actions in
$\CAL{A}$ identified by the same session name are annotated with the
same thread, and 3) Distinctness Condition: The input of session
initiation is always given a fresh thread.

%   \begin{itemize}
%   \item[--]Freshness Condition (FC):  If $t$ is by $\mathcal{A}$ at
%     some node and by $\mathcal{B}$ at another node then $\mathcal{A}$
%       and $\mathcal{B}$ always coincide. Further, if $\mathcal{A'}$
%       starts with an initialisation, then its passive thread should be
%       fresh w.r.t. all of its predecessors (if any).
%     \item[--]Session Consistency (SC): If $\mathcal{A'}$ starts with a
%       communication between $\mathcal{B}$ and $\mathcal{C}$ via (say)
%       $k$ and another subtree $\mathcal{A''}$ of $\mathcal{A}$ starts
%       with a communication via $k$ or an initialisation which opens
%       $k$, then the thread by $\mathcal{B}$ (resp. by $\mathcal{C}$)
%       of $\mathcal{A'}$ should be equal to the thread by $\mathcal{B}$
%       (resp. by $\mathcal{C}$) of $\mathcal{A''}$.
%     \item[--]Causal Consistency (CC): If $\mathcal{A''}$ is the direct
%       successor of $\mathcal{A'}$, then the active thread of
%       $\mathcal{A''}$ should coincide with the passive thread of
%       $\mathcal{A'}$.  We also say $\chor$ is well-threaded when there
%       is a well-formed annotation $\mathcal{A}$ of $\chor$ (i.e. the
%       result of erasing the annotations from $\mathcal{A}$ coincides
%       with $\chor$).
%     \end{itemize}
%     We also say $\chor$ is well-threaded when there is a well-formed annotation $\mathcal A$ of $\chor$ (i.e.: The result of erasing the annotations from $\mathcal A$ coincides with $\chor$).
% \end{definition}


The \emph{Well-threadedness} condition ensures global specifications
are free from unrealisable dependencies among actions. We say $\CAL{A}$
is well-threaded if it is connected and it has a consistent
annotation. 



% \begin{definition}[Mergeability]
\paragraph{Mergeability}
 Annotations in a choreography allow for the extraction of threads
 directly from the global behaviour. As threads are 
 sequences of actions to be executed at each end-point, we need to ensure that
 threads generated from choreographical annotations are meaningful, in
 the sense that they project only to the required end-points, and
 threads describing the behaviour of the same end point are
 encapsulated (\emph{merged})  on a single service description. 
  Mergeability, denoted by $\mergeable$, is the smallest equivalence
  over typed terms up to $\equiv$, closed under all typed contexts and

\begin{align*}
& \myruleg{M-Sel} 
   { 
	\forall i \in (J \cap H). (P_i \mergeable Q_i) \quad \forall j \in J\backslash H. \forall h \in H\backslash J. \mathsf{l}_j \neq \mathsf{l}_h
    }
    { \branching{k}{\Sigma_{j \in J} \mathsf{l_j} \pfx P_j}  \mergeable \branching{k}{\Sigma_{h \in H} \mathsf{l_h} \pfx P_h}
    } \quad 
%  \myruleg{M-In} 
%  { 
% 	x = y \quad P \mergeable Q
%  }
%  { \receive{k}{x} P \mergeable \receive{k}{y} Q   }
 \myruleg{M-Zero} 
 { 
	fsc(P) = 0
 }
 { P \mergeable \INACT   }
\end{align*}
When $P \mergeable Q$, we say that $P$ and $Q$ are mergeable.

% \end{definition}


Above, a context is any end-point calculus process with some
holes. $\Did{M-Sel}$ is for branching and says that we can allow
differences in branches which do not overlap, but we do demand each
pair of behaviours with the same operation to be identical.


The operation $P \merge Q$ allows for merging typed processes as long
as they are mergeable according to the rules above. $P \merge Q$ is a
partial commutative binary operator on typed processes which is
well-defined iff $P \mergeable Q$. We see an example of the merging
rules, and the full set can be consulted in Appendix
\ref{Logic4Struct:appendix:EPP-merging}. The merging of two branching
processes $\branching{k}{\Sigma_{i \in I} \mathsf{l_i} \pfx P_i} $
and $\branching{k}{\Sigma_{i \in J}}$ is given as:
\begin{align*}
	\branching{k}{\Sigma_{i \in I} \mathsf{l_i} \pfx P_i}\merge \branching{k}{\Sigma_{i \in J} \mathsf{l_i} \pfx Q_i} & \DEFEQ \branching{k}{ \begin{array}{l} \Sigma_{i \in I\cap J} \mathsf{l_i} \pfx P_i \merge Q_i \\ +\Sigma_{i \in I \backslash J} \mathsf{l_i} \pfx P_i  \\ +\Sigma_{i \in  J \backslash I} \mathsf{l_i} \pfx Q_i \end{array}  }
\end{align*}
That is, the resulting merge  groups in a single session branching all the
options coming from multiple branches that have the same session key.





Given a consistent annotation, we can project each of its threads onto
an end-point process. The thread
projection $\tp{\CAL{A}}{t}$ is a partial operation that uses the
merge operator, some of the rules are given below (the full set are
included in Appendix \ref{Logic4Struct:appendix:EPP-thread-proj}):


{\small
\[
\tp{\init {A^{t_1}}{B^{t_2}}{b}{\tilde k}\pfx\mathcal A}{t}
    \DEFEQ
  \quad  \left\{ 
      \begin{array}{ll}
        \initOut{b}{\tilde k}\pfx\tp{\mathcal A}{t_1}
        &         
        \textrm{if}\ t=t_1\\
        \repInitIn{b}{\tilde k}\pfx\tp{\mathcal A}{t_2}
        & 
        \textrm{if}\ t=t_2\\
        \tp{\mathcal A}{t}
        & 
        \text{otherwise}
      \end{array}
    \right.
\]

\[\tp{\choice {A^{t_1}}{B^{t_2}}{k}{l}{\CAL{A}}}{t}
    \DEFEQ
    \quad\left\{ 
      \begin{array}{ll}
        \selection {k} {l_i}\tp{\mathcal A_i}{t}
       &                 \textrm{if}\ t=t_1
\\
        \branching k {\sum_i l_i}\pfx\tp{\mathcal A_i}{t}
        & 
        \textrm{if}\ t=t_2\\
        \tp{\mathcal A}{t}
        & 
        \text{otherwise}
      \end{array}
    \right.
\]

\[ \tp{\itn{e@A^{t'}}{\mathcal A_1}{\mathcal A_2}}{t}
    \DEFEQ
   \quad    \left\{ 
      \begin{array}{ll}
        \itn{e}{\tp{\mathcal A_1}{t'}}{\tp{\mathcal A_2}{t'}}
        &         
        \textrm{if}\ t=t'\\
        \tp{\mathcal A_1}{t}\merge\tp{\mathcal A_2}{t}
        & 
        \text{otherwise}
      \end{array}
    \right.
\]
}









\begin{definition}[Coherent Interactions]
  Given a well-threaded, consistently annotated interaction $\CAL{A}$,
  we say that $\CAL{A}$ is coherent if the following two conditions
  hold:
  \begin{enumerate}
    \item For each thread $t$ in $\CAL{A}$, $TP(\CAL{A},t)$ is
      well-defined. 
    \item For each pair of threads $t_1,t_2$ in $\CAL{A}$ with
        $t_1 \equiv_A t_2$, we have $TP(A,t_1) \mergeable
        TP(\CAL{A}, t_2)$. 
      \end{enumerate}
\end{definition}

Below, 
% we say $\chor$ is restriction-free whenever it contains no
% terms of the form $\new{a}\chor'$ as its subterm. Additionally, 
$part(\chor)$ denotes the set of participants names occurring in
$\chor$. Recall also being coherent entails being well-typed,
connected and well-threaded.


\begin{definition}[End-Point Projection]
\label{Logic4Struct:def:epp}
% Let $\chor$ be a coherent interaction such that $\chor\LITEQ \new {\tilde
%   s}\chor'$ where $\chor'$ is restriction-free. Let $\mathcal A$ be a
% consistent annotation of $\chor'$.  
Let $\chor$ be a coherent interaction, and %  such that $\chor\LITEQ \new {\tilde
%   s}\chor'$
%  where $\chor$ is restriction-free. 
 $\mathcal A$ be a
consistent annotation of $\chor$.  
% Then the {\em end point projection of
%   $\CAL{A}$ under a state $\sigma$}, denoted $\epp{\new{\tilde
%     s}\CAL{A}}{\sigma}$, is given as the following network.
Then the {\em end point projection of
  $\CAL{A}$ under a state $\sigma$}, denoted $\epp{\CAL{A}}{\sigma}$, is given as the following network.
\[
\epp{\CAL{A}}{\sigma} \DEFEQ
\Pi_{A \in part(\chor)}\ 
\pr
{A}
{\Pi_{[t]}\bigsqcup_{t'\in[t]}\tp{\mathcal A}{t'}}
{\sigma@A}
\]
\end{definition} 

The mapping given above is defined after choosing a specific
annotation of an interaction. The following result shows the map in
fact does not depend on a specific (consistent) annotation chosen, as
far as a global description has no incomplete threads, i.e. it has no
free session channels (which is what programmers/designers usually
produce).

\begin{theorem}[Soundness and Completeness of End-point Projections\cite{carbone7scc}]
  \label{Logic4Struct:theorem:epp}
  Assume $\mathcal A$ is well-typed, strongly connected, well-threaded
  and coherent. Assume further $\typerule{\typeEnv}{}{\mathcal A \ccc
    \Delta}$ and $\typerule{\typeEnv}{}{\sigma}$. Then the following
  properties hold: 

\begin{itemize}
  \item (soundness) if $\epp{\mathcal{A}}{\sigma}\action{}  \network$
    then there exists $\mathcal A'$ such that $(\sigma, \mathcal{A}) \action{}
(\sigma', \mathcal{A'})$ such that $\epp{\mathcal{A'}}{\sigma'} \prune \equiv_{rec} \network$.

 \item (completeness) If $(\sigma, \mathcal{A}) \action{}  (\sigma', \mathcal{A'})$ then there exist $\network$ such
   that $\epp{\mathcal{A}}{\sigma} \action{} \network$ and $\epp{\mathcal{A'}}{\sigma'} \prune \network$.

  \item (soundness with action labels) if $\epp{\mathcal{A}}{\sigma}\action{\emm}  \network$
    then there exists $\mathcal A'$ such that $(\sigma, \mathcal{A}) \action{\ell}
(\sigma', \mathcal{A'})$ such that $\epp{\mathcal{A'}}{\sigma'} \prune \equiv_{rec} \network$.

 \item (completeness with action labels) If $(\sigma, \mathcal{A}) \action{\ell}  (\sigma', \mathcal{A'})$ then there exist $\network$ such
   that $\epp{\mathcal{A}}{\sigma} \action{\emm} \network$ and $\epp{\mathcal{A'}}{\sigma'} \prune \network$.

\end{itemize}
\end{theorem}

Where $\equiv_{rec}$ denotes equality induced by the unfolding of
process recursion. The asymmetric relation $P \prune Q$ indicates that $P$ is the result
of cutting off ``unnecessary branches'' of $Q$, in the light of P's
own typing, is formally defined as follows:


\begin{definition}[Pruning]\label{Logic4Struct:definition:pruning}
 Let $\typeEnv |-_{A} P |> \Delta$ for $\typeEnv$ and $\Delta$
minimal and $\typeEnv, \typeEnv' |-_{A} Q |> \Delta$. If further we
have $Q \equiv Q_0 \pp !R$ where $\typeEnv |- Q_0 |> \Delta$, $\typeEnv'
|-_{A} R |> \Delta$ and $P \mergeable Q_0$, then we can write: $\typeEnv |-_A P \prune Q |>
\Delta$ or $P \prune Q$ for short, and say $P$
prunes $Q$ under $\typeEnv; \Delta$. $\prune$ is extended to networks accordingly.
\end{definition}


%%% Local Variables: 
%%% mode: latex
%%% TeX-master: "../Thesis"
%%% End: 
