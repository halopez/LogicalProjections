\section{Proof System for \GL}\label{Logic4Struct:sec:proofSys}

% In a previous version of this
% article \cite{Carbone2010Towards-a-Modal}, we showed that \GL is
% undecidable for the global calculus with recursion, and
Here we  present a
model checking algorithm (in the form of a proof system) to decide
when a global logic formula is satisfied by a recursion-free
configuration of the global calculus. Indeed, similarly to
\cite{ct:csl01}, it turns out that the logic is decidable on the
recursion-free choreographies\footnote{As described in \cite{Carbone2010Towards-a-Modal}, removing recursion yields a
  decidability result in \GL}. We also prove the soundness and
completeness of the proposed proof system w.r.t.~the assertion
semantics.

In order to reason about judgments $\chor |=_\sigma \phi$, we propose
a proof (or inference) system for assertions of the form $\chor
|-_\sigma \phi$.  Intuitively, we want $\chor |-_\sigma \phi$ to be as
approximate as possible to $\chor |=_\sigma \phi$ (ideally, they
should be equivalent).  We write $\chor |-_\sigma \phi $ for the
provability judgement where $(\sigma,\chor)$ is a configuration and
$\phi$ is a formula.  

\begin{notation}
  We define the set of continuations, and the reachable set of
  configurations from an source configuration  $(\sigma, \chor)$, after an action
  $\ell$, as follows:
  \begin{align*}
    \textsf{Next}(\sigma,\chor,\ell) & \DEFEQ
    \{(\sigma',\chor') \mid (\sigma,\chor) \action{\ell} (\sigma',\chor')\}
    \\
    \textsf{Reachable}(\sigma,\chor) & \DEFEQ
    \{(\sigma',\chor') \mid (\sigma,\chor) \action{}^* (\sigma',\chor')\}
    \, .
  \end{align*}

Normalisation is required by the proof system to infer equality of
choreographies up to structural equi\-va\-len\-ce (Specially for the
$[ \cdot ] \pp [ \cdot ]$ operator).

  % We define $\textsf{Norm}(\chor)$ to be the normalisation of a
%   recursion-free choreography $\chor$:
 We define $\textsf{Norm}(\chor)$ to be a normalisation function from
  recursion-free choreographies into multisets of choreographies:
  \begin{align*}    
    &\textsf{Norm}(\interact{A}{B}{k}{e}{y}\pfx \chor) \DEFEQ
    [ \interact{A}{B}{k}{e}{y}\pfx \chor ]
    \\
    &\textsf{Norm}(\choice{A}{B}{k}{l}{\chor}) \DEFEQ
    [ \choice{A}{B}{k}{l}{\chor} ]
    \\
    &\textsf{Norm}(\init{A}{B}{a}{k}\pfx \chor) \DEFEQ
    [\init{A}{B}{a}{k}\pfx \chor]
    \\
    &\textsf{Norm}(\itn{e@A}{\chor_1}{\chor_2}) \DEFEQ
    [\itn{e@A}{\chor_1}{\chor_2}]    
    \\    
    &\textsf{Norm}(\INACT) \DEFEQ {[} \ {]}
    \\
    &\textsf{Norm}(\chor_1 \pp \chor_2) \DEFEQ [
    P_1,\dots,P_n,Q_1,\dots,Q_m ]
    \quad \text{if }
    \begin{array}{ll}
      \textsf{Norm}(\chor_1) = [P_1,\dots,P_n] & \text{and} \\
      \textsf{Norm}(\chor_2) = [Q_1,\dots,Q_m] & .
    \end{array}
  \end{align*}

\end{notation}

\begin{lemma}[Normalisation preserves structural equivalence]
  \label{Logic4Struct:lem:normalisation}
  Let $\chor$ be a recursion-free choreography and
  $\textsf{Norm}(\chor) = [P_1,\dots,P_n]$, then $\chor \equiv
  \prod_{i=1}^n P_i$.
\end{lemma}
\begin{proof}
  By induction on the structure of the choreography $\chor$.
   \begin{description}
   \item[Case $\chor = \INACT$:] We have $\textsf{Norm}(\INACT) = [\
     ]$, and $\prod_{i=1}^0 P_i = \INACT \equiv \INACT$.
   \item[Case $\chor = \chor_1 \pp \chor_2$:] We have that
     $\textsf{Norm}(\chor_1) = [P_1,\dots,P_n]$,
     $\textsf{Norm}(\chor_2) = [Q_1,\dots,Q_m]$, and $\prod_{i=1}^n P_i
     \equiv \chor_1$, $\prod_{j=1}^m Q_j \equiv \chor_2$ by induction
     hypothesis. Then, we can derive that $\prod_{i=1}^n P_i \pp
     \prod_{j=1}^m Q_j \equiv \chor_1 \pp \chor_2$.
   \item[All the other cases:] We have that
     $\textsf{Norm}(\chor) = [P_1]$, where $P_1 = \chor$, then
     $\prod_{i=1}^1 P_i \equiv \chor$. 
   \end{description}
\end{proof}

\begin{definition}[Entailment]
  We say that a choreography $\chor$ \emph{entails} a formula $\phi$
  under a state $\sigma$, written $\chor|-_\sigma \phi$, iff the
  assertion $\chor|-_\sigma \phi$ has a proof in the proof system
  given in Table~\ref{Logic4Struct:table:Global:proofSys}.
\end{definition}

\begin{table}
  \begin{gather*}
    \myruleg{P_{end}}{\textsf{Norm}(\chor) = [\ ]}
    {\typeruleE{\chor}{\sigma}{\endF}}
    ~
    \myruleg{P_{and}}{\typeruleE{\chor}{\sigma}{\phi} \quad
      \typeruleE{\chor}{\sigma}{\chi}} {\typeruleE{\chor}{\sigma}{\phi
        \land \chi}}
    ~
    \myruleg{P_{neg}}{ \chor \not |-_{\sigma} \phi}
    {\typeruleE{\chor}{\sigma}{\neg \phi}}
    ~
    \myruleg{P_{may}}{\exists (\sigma',\chor') \in
      \textsf{Reachable}(\sigma,\chor).\  \typeruleE{\chor'}{\sigma'}{\phi}}
    {\typeruleE{\chor}{\sigma}{\may \phi}}
    \\[1ex]
    \myruleg{P_{par}}{ \begin{matrix}[c]
        \textsf{Norm}(\chor) = [P_1,\dots,P_n] \\
        \exists I,J.\
        I\cup J = \{1,\dots,n\} \wedge
        I\cap J = \emptyset \wedge
        \typeruleE{\prod_{i\in I} P_i}{\sigma}{\phi_1} \wedge
        \typeruleE{\prod_{j\in J} P_j}{\sigma}{\phi_2} \end{matrix}
     }
    {\typeruleE{\chor}{\sigma}{ \phi_1 \pp \phi_2}}
  \\[1ex]
    \myruleg{P_{\exists}}
    {\exists w\in fn(\chor)\cup fn(\phi).\ 
      \typeruleE{\chor}{\sigma}{\phi[w/\var]}}
    {\typeruleE{\chor}{\sigma}{\exists \var \pfx \phi}}
    \qquad\qquad
    \myruleg{P_{exp}}{
      \sigma(e_1 @ A) \Downarrow v \quad \sigma(e_2 @ B) \Downarrow v}
    {\typeruleE{\chor}{\sigma}{ (e_1 @ A = e_2 @ B) }}
    \\[1ex]
    \myruleg{P_{action}}{\exists (\sigma',\chor') \in
      \textsf{Next}(\sigma,\chor,\ell).\  \typeruleE{\chor'}{\sigma'}{\phi}}
    {\typeruleE{\chor}{\sigma}{\langle\ell \rangle \phi}}
  \end{gather*}
  \caption{Proof system for the Global Calculus.}
  \label{Logic4Struct:table:Global:proofSys}
\end{table}

Let us now describe some of the inference rules of the proof system.
The rule $\mathsf{P_{end}}$ relates the inaction terms with the
termination formula. The rules $\mathsf{P_{and}}$ and
$\mathsf{P_{neg}}$ denote rules for conjunction and negation in
classical logic, respectively. The rule for parallel composition is
represented in $\mathsf{P_{par}}$; it does not indicate the behaviour
of a given choreography, but but provides information about the
structure of the process: $\mathsf{P_{par}}$ juxtaposes the behaviour
of two processes and combines their respective formulae by the use of
a separation operator. The next rule, $\mathsf{P_{action}}$ requires
that the process $P$ in the configuration $\sigma$ can perform an
action labelled $\ell$, so we must search for a continuations of
$(\sigma,\chor)$ after an action $\ell$ and find a configuration which
satisfies the rest of the formula, i.e., $\phi$.  Analogously,
$\mathsf{P_{may}}$ looks for a continuation in the reachable
configuration of $(\sigma,\chor)$ in order to satisfy $\phi$.  The rule
$\mathsf{P_\exists}$ says that in order to satisfy an $\exists t\pfx
\phi$, it is sufficient to find a value $w$ for $t$ in the free names
used by the choreography $\chor$ or in the free names used by the
formula $\phi$. Finally, the rule $\mathsf{P_{exp}}$ denotes
evaluation of expressions.

We now proceed to prove the soundness of the proof system with respect
to the semantics of assertions presented before.

\begin{lemma}[Structural congruence preserves satisfiability]
  \label{Logic4Struct:lemma:StructuralSatisfiability} If $\chor\equiv\chor'$ and
  $\chor|=_\sigma \phi$, then $\chor' |=_\sigma \phi$.
\end{lemma}
\begin{proof}
    It follows by induction on the structural congruence rules in
    $\equiv$ and second induction on the height of the derivation tree
    for   $\chor|=_\sigma \phi$.
    % It follows from simple case analysis over $\equiv$.
    We have the following cases:
   \begin{description}
   	\item[Case $P' \equiv P \pp \INACT$:]
		We have that  $P |=_\sigma \phi$ and $P \equiv P \pp \INACT $. By the definition of $|=$, 
		$P' |=_\sigma \phi \pp \psi$ iff $P' \equiv P \pp Q$, $P |=_\sigma \phi$ and $Q |=_\sigma \psi$. 
		Substituting $Q$ with $\INACT$, we have that $P' |=_\sigma \phi \pp \true$, that implies $P' |=_\sigma \phi$
	\item[Cases $P \pp Q \equiv Q \pp P$, $P \pp (Q \pp R) \equiv
          (P \pp Q) \pp R$ and $P \equiv_\alpha P'$:] They are
          straight-forward  as $\pp$ is commutative and associative in \GL and variable substitution does not affect provability.
%	\item[Case $P \equiv_\alpha P'$:]
   \end{description}
\end{proof}


\begin{theorem}[Soundness]\label{Logic4Struct:thm:soundness}
  For any configuration $(\sigma,\chor)$, where $\chor$ is
  recursion-free, and every formula $\phi$, if $\chor|-_\sigma \phi$
  then $\chor|=_\sigma \phi$.
\end{theorem}
\begin{proof}
The proof proceeds by induction on the height of the derivation tree
for $|-_\sigma$.
  \begin{description}
  \item[Case $\mathsf{P_{end}}$:] Straight consequence of
    Lemmas~\ref{Logic4Struct:lem:normalisation} and
    \ref{Logic4Struct:lemma:StructuralSatisfiability}, indeed $\chor \equiv \INACT$
    and $\chor |=_\sigma \endT$.
  \item[Case $\mathsf{P_{and}}$:] By induction hypothesis and
    conjunction.
  \item[Case $\mathsf{P_{neg}}$:] We have that $\chor|-_\sigma \lnot
    \phi$, so by $\mathsf{P_{neg}}$ we get $\chor \not |-_\sigma
    \phi$. By induction hypothesis we have that $\chor \not |=_\sigma
    \phi$, which is the necessary condition to deduce $\chor |=_\sigma
    \lnot \phi$.
  \item[Case $\mathsf{P_{par}}$:] We have that $\chor |-_\sigma
    \phi_1\pp \phi_2$, then $\textsf{Norm}(\chor) = [P_1,\dots,P_n]$,
    and there exist $I,J$ such that $I\cup J = \{1,\dots,n\}$, $I\cap J
    = \emptyset$, $\typeruleE{\prod_{i\in I} P_i}{\sigma}{\phi_1}$, and
    $\typeruleE{\prod_{j\in J} P_j}{\sigma}{\phi_2}$. By induction
    hypothesis we know that $\prod_{i\in I} P_i |=_\sigma \phi_1$ and
    $\prod_{j\in J} P_j |=_\sigma \phi_2$, then by
    Lemma~\ref{Logic4Struct:lem:normalisation} we have $\chor \equiv \prod_{i\in I}
    P_i \pp \prod_{j\in J} P_j$, hence it is immediate to prove that
    $\chor |=_\sigma \phi_1 \pp \phi_2$.
  \item[Case $\mathsf{P_{action}}$:] We have that $\chor |-_\sigma
    \langle \ell \rangle \phi$ and by $\mathsf{P_{action}}$ then
    $\chor' |-_{\sigma'} \phi$ and $(\sigma', \chor') \in
    \textsf{Next}(\sigma, \chor, \ell)$. From the induction hypothesis
    we have that $\chor' |=_{\sigma'} \phi$, then we have to show that
    $\chor |=_\sigma \actionF{\ell} \phi$. From the assertion
    semantics we know that $C |=_\sigma \actionF{\ell} \phi$ iff
    $(\sigma, \chor') \action{\ell} (\sigma', \chor')$ and $\chor'
    |=_{\sigma'} \phi$, which holds immediately by the selection of
    $(\sigma', \chor') \in \textsf{Next}(\sigma, \chor,\ell)$ and the
    induction hypothesis.
  \item[Case $\mathsf{P_{may}}$:] We have that $\chor |-_\sigma \may
    \phi$ and by $\mathsf{P_{may}}$ then $\chor' |-_{\sigma'} \phi$ and
    $(\sigma', \chor') \in \textsf{Reachable}(\sigma, \chor)$. From the
    induction hypothesis we have that $\chor' |=_{\sigma'} \phi$, then
    we have to show that $\chor |=_\sigma \may \phi$. From the
    assertion semantics we know that $C |=_\sigma \may \phi$ $ \iff
    (\sigma, \chor') \action{} ^* (\sigma', \chor')$ and $\chor'
    |=_{\sigma'} \phi$, which holds immediately by the selection of
    $(\sigma', \chor') \in \textsf{Reachable}(\sigma, \chor)$ and the
    induction hypothesis.
  \item[Case $\mathsf{P_{\exists}}$:] We have that $\chor |-_\sigma
    \exists t. \phi$ and by $\mathsf{P_{\exists}}$ we have that
    $\exists w \in fn(\chor) \cup fn(\phi)$ and $\chor |-_\sigma \phi
    [w/t]$. By induction hypothesis we know that $C |=_\sigma \phi
    [w/t]$ with appropriate $w \in fn(\chor) \cup fn(\phi)$, then
    $\chor |=_\sigma \exists t. \phi$ follows from the definition of
    the assertion semantics.
  \item[Case $\mathsf{P_{exp}}$:] It holds directly by checking if
    $\sigma(e_1@A) \Downarrow v$ and $\sigma(e_2@B) \Downarrow
    v$. 
  \end{description}
\end{proof}

The proof of completeness relies on a renaming lemma, that states that
logical properties are preserved over name permutations. These kind of
lemmas are standard among logical frameworks for process calculi
(e.g. \cite{cg:popl00,DBLP:conf/cmsb/MiculanB06}) and the general
  proof technique proceeds by induction on the syntax of formulae.


\begin{lemma}[Renaming preserves satisfiability]\label{Logic4Struct:lem:exists}
  Let $(\sigma,\chor)$ be a configuration with $\chor$ a recursion
  free choreography and a state $\sigma$. Let $\{n_1,\dots,n_k\}
  = fn(\chor) \cup fn(\phi)$. We can state that given  a formula $\phi$, $\chor |=_\sigma \exists t\pfx
  \phi$ if and only if $\exists m\in \{n_1,\dots,n_k\}$ such that $\chor
  |=_\sigma \phi[m/t]$.
\end{lemma}
\begin{proof}
  Both directions are proven via induction on the structure of
  $\phi$. We include some of the most relevant cases:
  \begin{description}
    \item[Case  $\phi = \phi_1 \pp \phi_2 $:] \hfill 
          \begin{description}
            \item[($=>$ side):] \hfill \\
              Given that $\chor |=_\sigma \exists t\pfx \phi_1 \pp
              \phi_2$ then $\chor \equiv \chor_1 \pp \chor_2$ and
              $\chor_1 |=_\sigma \exists t \pfx \phi_1 $ and $\chor_2
              |=_\sigma \exists t \pfx \phi_2$. We have to show that
              $\exists m \in fn(\chor) \cup fn(\phi_1 \pp \phi_2)$
              such that $\chor |=_\sigma (\phi_1 \pp \phi_2)[m/t]$.

              From the definition of $\phi$ and the induction
              hypothesis, we have that: 
              \begin{equation}
                \exists m \in fn(\chor_1) \cup
              fn(\phi_1) \text{ such that } \chor_1 |=_\sigma
              \phi_1[m/t]
              \end{equation} 
              Similarly, we know that 
              \begin{equation}
                \exists m \in
              fn(\chor_2) \cup fn(\phi_2) \text{ such that } \chor_2
              |=_\sigma \phi_2[m/t]
              \end{equation} 
              Note that, from $\chor \equiv
              \chor_1 \pp \chor_2$ we can infer that $fn(\chor) =
              fn(\chor_1) \cup fn(\chor_2).$, and from $\phi = \phi_1
              \pp \phi_2$. Then, $\exists m \in fn(\chor) \cup
              fn(\phi) $ such that $\chor  |=_\sigma (\phi_1 \pp
              \phi_2)[m/t]$.

              \item[($<=$ side):] \hfill \\
                Pick an $m$ such that $m \in fn(\chor) \cup
                fn(\phi_1 \pp \phi_2)$ and $\chor |=_\sigma (\phi_1
                \pp \phi_2)[m/t]$. we have to show that $\chor
                |=_\sigma \exists t\pfx (\phi_1 \pp \phi_2)$.

                From the assertion semantics, $\chor |=_\sigma
                (\phi_1 \pp \phi_2)[m/t]$ iff $\chor \equiv \chor_1
                \pp \chor_2$ and $\chor_1 |=_\sigma \phi_1[m/t]$ and
                $\chor_2 |= \phi_2 [m/t]$. 

                Now assume the induction hypothesis  $\chor_1 |=_\sigma
                \exists t \pfx \phi_1$ and $\chor_2 |= \exists t \pfx
                \phi_2$. Note  that $fn(\chor) \cup fn(\phi_1 \pp
                \phi_2) = fn(\chor_1) \cup fn(\chor_2) \cup fn(\phi_1
                \cup fn(\phi_2))$. Then, applying the induction hypothesis, we can
                conclude that $\chor |=_\sigma \exists t \pfx (\phi_1
                \pp \phi_2)$.
           \end{description}
      \item[Case $\phi = \actionF{\ell}\phi' $:] \hfill 
        \begin{description}
          \item[($=>$ side):] \hfill \\
            Given that $\chor |=_\sigma \exists t \pfx \actionF{\ell}
            \phi'$, then $\chor |=_\sigma \actionF{\ell}
            \phi'[m/t]$. Applying the definition in the assertion
            semantics, $(\sigma, \chor[m/t]) \action{\ell[m/t]}
            (\sigma',\chor'[m/t])$.  We have to show that $\exists m
            \in fn(\chor) \cup fn(\actionF{\ell}\phi')$ such that $\chor
            |=_\sigma \actionF{\ell} \phi'[m/t]$.

            We consider two cases for $m$: 1) $m \in fn(\chor') \cup
            fn(\phi') $, then after direct application of the
            induction hypothesis $C' |=_{\sigma'} \phi'[m/t]$ we get
            $\chor |=_\sigma \actionF{\ell} \phi'[m/t]$. 2)  In the
            case that $m \in (fn(\chor) \cup
            \fn(\actionF{\ell}\phi'))/(fn(\chor') \cup fn(\phi'))$.
            We know  that $fn(\chor') \subseteq \fn(\chor)$ and
            $\fn(\actionF{\ell} \phi') = fn(\phi')$. Then $m \in
            fn(\chor)/fn(\chor')$. From the application of  the  induction hypothesis
            $\chor'[m/t] |=_{\sigma'} \phi'[m/t]$ to  $\chor |=_\sigma \actionF{\ell} \phi'[m/t]$, we have that $\chor[m/t]
            |=_\sigma \actionF{\ell[m/t]}\phi'[m/t]$ implying 
            $\chor[m/t] |=_\sigma \actionF{\ell} \phi[m/t]$, 
            %, and given
%             $fn(\chor) = fn(\chor[m/t]) $
             then $\chor |=_\sigma
            \actionF{\ell} \phi'[m/t]$.

            \item[($<=$ side):] \hfill\\
              Pick $m \in fn(\chor) \cup fn(\actionF{\ell} \phi')$
              such that $\chor |=_\sigma \actionF{\ell}
              \phi'[m/t]$. We have to show that $\chor |=_\sigma
              \exists t \pfx \actionF{\ell} \phi'$.

              From the induction hypothesis, we have that $\chor'
              |=_{\sigma'} \phi'[m/t]$, which implies $\chor'
              |=_{\sigma'} \exists t \pfx \phi'$. Note that
              $fn(\chor') \cup fn(\phi') \subseteq fn(\chor) \cup
              fn(\actionF{\ell} \phi')$, then $\chor |=_\sigma \exists
              t \pfx \actionF{\ell} \phi$.
          \end{description}
        \item[Case $\phi = \exists t' \pfx \phi' $:] \hfill 
        \begin{description}
          \item[($=>$ side):] \hfill \\
            $\chor |=_\sigma \exists t. \exists t' \phi'$ then $\chor
            |=_\sigma (\exists t'. \phi')[m/t]$. We have to show that
            $\exists m \in fn(\chor) \cup fn(\exists t'. \phi')$ such
            that $\chor |=_\sigma (\exists t'. \phi')[m/t]$. From the
            the assertion semantics, we have that $\chor
            |=_\sigma \exists t \pfx \phi'[m/t] <=> \chor |=_\sigma \phi'[m'/t'][m/t]$. Select a name $m$, and use $\alpha-$conversion to
            ensure that $\sigma$ leaves unchanged all bound variables
            in the choreography $\chor$ and the formula $\phi'$.  Then $m \neq m'$ and
            $\chor |=_\sigma \phi[m'/t'][m/t]$. From the induction
            hypothesis we know that $m' \in fn(\chor) \cup fn(\exists
            t'. \phi')$ such that $\chor |=_\sigma \phi[m'/t']$. Given
            $fn(\exists t. \exists t' \phi') = fn(\exists t'. \phi')
            \cup t$, then $\chor |=_\sigma \phi'[m'/t'][m/t]$.
            \item[($<=$ side):] \hfill\\
              Pick a name $n \in fn(\chor) \cup fn(\exists x. \phi)$
              such that $\chor |=_\sigma \exists x . \phi'[m'/m]$. We use $\alpha-$conversion to
            ensure that $\sigma$ leaves unchanged all bound variables
            in the choreography $\chor$ and the formula $\phi'$.  We have to show that $\chor |=_\sigma \exists
              x. \exists m. \phi'$. After $\alpha$-renaming, we have to consider the possible
              selections of $n$.
              \begin{description}
                \item[($n\neq m' \land n\neq m$):] We have that $\chor
                  |=_\sigma \phi' [m'/m][n/x]$. Since $n\neq m'$ then
                  $\phi[m'/m][n/x] = \phi' [n/x][m'/m]$. Since $n\neq
                  m$ then $m' \not \in fn(\chor) \cup fn(\phi'[n/x])$
                  therefore $m' \in fn(\chor) \cup fn(\exists x. \phi' )$,
                  and  $\chor |=_\sigma \exists x. \exists m .
                  \phi'$.
                  \item[($n\neq m' \land n=m$):] We have that $\chor
                    |=_\sigma \phi'[m'/m][m/x]$. Since $n =m$ then
                    $\phi'[m'/m][m/x] = \phi'[m'/x]$. Then, $m' \in
                    fn(\chor) \cup fn(\exists m. \phi')$  and,
                    applying the induction hypothesis $\chor |=_\sigma \exists
                    m \pfx \phi'$ leads to  $\chor |=_\sigma \exists x. \exists m
                    \pfx \phi' \land x = m$.
%                     ..... $\exists x \pfx \exists m \pfx \phi'$. From $n \neq
%                     m'$ then $m' \in fn(\chor) \cup fn(\exists m. \phi')$
                \end{description}
          \end{description}
   \end{description}
%   (Sketch) By induction on the structure of $\phi$.  It is similar to
%   the proof of \cite[Lemma~5.3(3)]{cg:popl00}.
\end{proof}

\begin{theorem}[Completeness]\label{Logic4Struct:thm:completeness}
  For any configuration $(\sigma,\chor)$, where $\chor$ is
  recursion-free, and every formula $\phi$, if $\chor|=_\sigma \phi$
  then $\chor|-_\sigma \phi$.
\end{theorem}
\begin{proof}
  By rule induction on the derivation of $|=_\sigma$.
   \begin{description}
   \item[Case $\chor |=_\sigma \endT$:] We have that $\chor \equiv
     \INACT$ and hence $\textsf{Norm}(\chor) = [\ ]$ by
     Lemma~\ref{Logic4Struct:lem:normalisation}. Now, the thesis follows immediately
     from the application of $ \mathsf{P_{end}}$.
   \item[Case $\chor |=_\sigma (e_1 @ A = e_2 @ B)$:] It follows
     immediately by the application of $\mathsf{P_{exp}}$.
   \item[Case $\chor |=_\sigma \langle \ell \rangle \phi'$:] Take
     $(\sigma, \chor) \action{\ell} (\sigma',\chor')$ and $\chor'
     |=_{\sigma'} \phi' $, we have by induction hypothesis that
     $\typeruleE{\chor'}{\sigma'}{\phi'}$. Now, we have to show that
     $\chor |-_\sigma \langle \ell \rangle \phi'$.  By the fact that
     $(\sigma, \chor) \action{\ell} (\sigma',\chor')$, we have that
     $(\sigma', \chor')\in \textsf{Next}(\sigma,\chor,\ell)$, hence, we
     can apply rule $\mathsf{P}_{action}$ and we are done.
   \item[Case $\chor |=_\sigma \phi \land \chi$:] We have that $\chor
     |=_\sigma \phi$ and $\chor |=_\sigma \chi$. From the induction
     hypothesis we have that $\chor |-_\sigma \phi$ and $ \chor
     |-_\sigma \chi$. The application of $\mathsf{P_{and}}$ lead to
     $\chor |-_\sigma \phi \land \chi$ as desired.
   \item[Case $\chor |=_\sigma \lnot \phi$:] From the definition of the
     assertion semantics we have that $\chor |=_\sigma \lnot \phi $ iff
     $\chor \not |=_\sigma \phi$. We have to show that $\chor |-_\sigma
     \lnot \phi$. We proceed by contradiction. Take a $(\phi, \chor)$
     such that $\chor |-_\sigma \phi$, then from
     Theorem~\ref{Logic4Struct:thm:soundness} we have that $\chor |=_\sigma \phi$,
     which is a contradiction to $\chor |=_\sigma \lnot \phi$.
   \item[Case $\chor |=_\sigma \exists \var \pfx \phi$:] We have that
     $\chor |=_\sigma \exists t. \phi$ and by the definition in the
     assertion semantics we have that $\chor |=_\sigma \phi [w/t]$ for
     an appropriate $w$. By induction hypothesis we know that $\chor
     |-_\sigma \phi[w/t]$. Lemma~\ref{Logic4Struct:lem:exists} guarantees that there
     exists $w \in fn(\chor)\cup fn(\phi)$ in order to derive $\chor
     |-_\sigma \exists t. \phi$ from $\mathsf{P_{\exists}}$.
   \item[Case $\chor |=_\sigma <<>> \phi$:] Take $(\sigma, \chor)
     \action{}^* (\sigma',\chor')$ and $\chor' |=_{\sigma'} \phi'$, we
     have by induction hypothesis that
     $\typeruleE{\chor'}{\sigma'}{\phi'}$. Now, we have to show that
     $\chor |-_\sigma <<>> \phi'$.  By the fact that $(\sigma, \chor)
     \action{}^* (\sigma',\chor')$, we have that $(\sigma', \chor')\in
     \textsf{Reachable}(\sigma,\chor)$, hence, we can apply rule
     $\mathsf{P}_{may}$ and we are done.
   \item[Case $\chor |=_\sigma \phi \pp \chi$:] We have that $\chor
     \equiv \chor_1 \pp \chor_2$ and $\chor_1 |=_\sigma \phi \land
     \chor_2 |=_\sigma \chi$. From the induction hypothesis $\chor_1
     |-_\sigma \phi$ and $\chor_2 |-_\sigma \chi$. Now by
     Lemma~\ref{Logic4Struct:lem:normalisation} we have that $\chor_1 \equiv
     \prod_{i\in I} P_i$ and $\chor_2 \equiv \prod_{j\in J} P_j$ for
     some $I,J$. So, we can derive $\chor \equiv \prod_{i\in I} P_i \pp
     \prod_{j\in J} P_j$, and hence $\mathsf{P_{par}}$ leads to
     $\chor_1 \pp \chor_2 |-_\sigma \phi \pp \chi$. 
   \end{description} 
\end{proof}

\begin{theorem}[Termination]\label{Logic4Struct:thm:termination}
  For any configuration $(\sigma,\chor)$, where $\chor$ is
  recursion-free, and every formula $\phi$, proof-checking algorithm
  terminates.
\begin{proof}
  First, notice that all the functions \textsf{Norm}, \textsf{Next},
  and \textsf{Reachable} are total and computable. The proof is by
  induction over the structure of $\phi$.
   \begin{description}
   \item[Case $\phi = \endT$:] $\typeruleE{\chor}{\sigma}{\endT}$ iff
     $\textsf{Norm}(\chor) = [\ ]$.
   \item[Case $\phi = \phi_1 \land \phi_2$:] By conjunction and
     induction hypothesis on $\typeruleE{\chor}{\sigma}{\phi_1}$ and
     $\typeruleE{\chor}{\sigma}{\phi_2}$.
   \item[Case $\phi = \neg \phi'$:] $\typeruleE{\chor}{\sigma}{\phi}$
     iff $\typeruleE{\chor}{\sigma}{\phi'}$ does not hold. But by
     induction hypothesis we can construct a terminating proof or
     confutation for $\typeruleE{\chor}{\sigma}{\phi'}$. Hence the proof
     for $\typeruleE{\chor}{\sigma}{\phi}$ terminates as well.
   \item[Case $\phi = \phi_1 \pp \phi_2$:] Suppose
     $\textsf{Norm}(\chor) = [P_1,\dots,P_n]$. Notice that there exists
     a finite number of possible partitions of the index set
     $\{1,\dots,n\}$ as 
     $I \cup ,J$. Hence, for every $I \in ,J$ we can compute
     $\typeruleE{\prod_{i\in I} P_i}{\sigma}{\phi_1}$ and
     $\typeruleE{\prod_{j\in J} P_j}{\sigma}{\phi_2}$, which both
     terminate by induction hypothesis. By applying
     Lemma~\ref{Logic4Struct:lem:normalisation} we prove the thesis.
   \item[Case $\phi = \langle \ell \rangle \phi'$:] First, notice that
     the set $\textsf{Next}(\sigma,\chor,\ell)$ is finite, because the
     choreographies are finite, i.e., there are a finite number of
     actionable transition in a given configuration. For each
     configuration $(\sigma',\chor') \in
     \textsf{Next}(\sigma,\chor,\ell)$,
     $\typeruleE{\chor'}{\sigma'}{\phi'}$ terminates by induction
     hypothesis.
   \item[Case $\phi = <<>> \phi'$:] As before, notice that the set
     $\textsf{Reachable}(\sigma,\chor)$ is finite, because the
     choreographies are finite, i.e., the choreographies are recursion
     free. For each configuration $(\sigma',\chor') \in
     \textsf{Reachable}(\sigma,\chor)$,
     $\typeruleE{\chor'}{\sigma'}{\phi'}$ terminates by induction
     hypothesis.
   \item[Case $\phi = \exists t\pfx \phi'$:] To prove existence is
     sufficient to check every derivation by substituting $t$ with a
     name $w\in fn(\chor)\cup fn(\phi)$. Notice that $fn(\chor) \cup
     fn(\phi)$ is finite, because both $\chor$ and $\phi$ are so. So,
     for every $w$, we can construct a terminating derivation for
     $\typeruleE{\chor}{\sigma}{\phi'[w/t]}$ by induction hypothesis.
   \item[Case $\phi = (e_1@A = e_@@B):$]
     $\typeruleE{\chor}{\sigma}{(e_1@A = e_@@B)}$ iff $e_1@A \Downarrow
     v$ and $e_@@B \Downarrow v$.  
   \end{description}
\end{proof}
\end{theorem}



% \section{Proof System}\label{Logic4Struct:sec:proofSys}
% %
% In this section we present a proof system for the global logic.  In
% order to reason about judgments $\chor |=_\sigma \phi$, we propose a
% proof (or inference) system for assertions of the form $\chor
% |-_\sigma \phi$. Intuitively, we want $\chor |-_\sigma \phi$ to be as
% approximate as possible to $\chor |=_\sigma \phi$ (ideally, they
% should be equivalent). We write $\chor |-_\sigma \phi $ for the
% provability judgement where $(\sigma,\chor)$ is a configuration and
% $\phi$ is a formula.

% \begin{definition}[Exhibition - Choreography]
%   We say that a choreography $\chor$ \emph{exhibits} a formula $\phi$
%   under an environment $\sigma$, written $\chor|-_\sigma \phi$, iff
%   the assertion $\chor|-_\sigma \phi$ has a proof in the proof system
%   given in Table~\ref{Logic4Struct:table:Global:proofSys}.
% \end{definition}
% %
% \begin{table}[t]
%   \begin{align*}
%     & \myruleg{P_{init}}{\typeruleE{\chor}{\sigma}{\phi}}
%     {\typeruleE{\init{A}{B}{a}{k} \pfx \chor} {\sigma}{\langle
%         \initF{A}{B}{a(k)}\rangle\phi}} \qquad
%     \myruleg{P_{and}}{\typeruleE{\chor}{\sigma}{\phi} \quad
%       \typeruleE{\chor}{\sigma}{\chi}} {\typeruleE{\chor}{\sigma}{\phi
%         \land \chi}}
%     \\[1mm]
%     & \myruleg{P_{sel}}{\typeruleE{\forall i\in I.\
%         \chor_i}{\sigma}{\phi_i}} {\typeruleE{
%         \choice{A}{B}{k}{l}{\chor}}{\sigma}{\bigwedge_{i\in I} \langle
%         \branchF{A}{B}{l_i}{k}\rangle \phi_i}} \qquad
%     \myruleg{P_{neg}}{ \chor \not |-_{\sigma} \phi}
%     {\typeruleE{\chor}{\sigma}{\neg \phi}}
%     \\[1mm]
%     & \myruleg{P_{com}}{\typeruleE{\chor}{\sigma}{\phi}}
%     {\typeruleE{\interact{A}{B}{k}{e}{y}\pfx \chor }
%       {\sigma}{\langle\comF{A}{B}{k}\rangle \phi}} \qquad
%     \myruleg{P_{par}}{\typeruleE{\chor_1}{\sigma}{\phi_1} \quad
%       \typeruleE{\chor_2}{\sigma}{\phi_2}} {\typeruleE{\chor_1 \pp
%         \chor_2}{\sigma}{ \phi_1 \pp \phi_2}}
%     \\[1mm]
%     & \myruleg{P_{end}}{} {\typeruleE{ \INACT }{\sigma}{\endF}} \quad
%     \myruleg{P_{may1}}{ \typeruleE{\chor}{\sigma}{\phi}}
%     {\typeruleE{\chor}{\sigma}{\may \phi}} \quad \myruleg{P_{may2}}{
%       \typeruleE{\chor}{\sigma}{\phi \lor \nextOp \may \phi}}
%     {\typeruleE{\chor}{\sigma}{\may \phi}} \quad \myruleg{P_{exp}}{
%       \sigma(e_1) \Downarrow v \quad \sigma(e_2) \Downarrow v}
%     {\typeruleE{\chor}{\sigma}{ (e_1 = e_2) }}
%     \\[1mm]
%     & \myruleg{P_{ifT}}{\sigma(e@A) \Downarrow \true \quad
%       \typeruleE{\chor_1}{\sigma}{\phi}} {\typeruleE{ \itn
%         {e@A}{\chor_1}{\chor_2} }{\sigma}{\phi}} \qquad
%     \myruleg{P_{ifF}}{\sigma(e@A) \Downarrow \false \quad
%       \typeruleE{\chor_2}{\sigma}{\phi}} {\typeruleE{ \itn
%         {e@A}{\chor_1}{\chor_2} }{\sigma}{\phi}}
%     \\[1mm]
%     & \myruleg{P_{\exists}}{\exists w\in fn(\chor) \cup \{k\}.\
%       \typeruleE{\chor}{\sigma}{\phi[w/\var]} ~\text{ $k$ fresh}}
%     {\typeruleE{\chor}{\sigma}{\exists \var \pfx \phi}} \qquad
%     \myruleg{P_{struct}}{\typeruleE{\chor}{\sigma}{\phi} \quad
%       \chor\equiv \chor'} {\typeruleE{\chor'}{\sigma}{\phi}}
%     % &
%     % \myruleg{P_{var}}{ \mathbf{-} } {\typeruleE{ X }{}{\phi}}
%     % \qquad
%     % \myruleg{P_{rec}}{\typeruleE{C\MSUBS{\mu X.C}{X}}{}{\phi} }
%     % {\typeruleE{ \mu X\pfx C }{}{ \phi }}
%  \end{align*}
%  \caption{Proof system for the Global Calculus.}
%  \label{Logic4Struct:table:Global:proofSys}
% \end{table}

% \marginpar{Change the description aside accordingly to the change done
%   to the proof system} Let us now describe some of the inference rules
% of the proof system.  Any configuration satisfies $\mathsf{P_{true}}$.
% %$\mathsf{P_{var}}$ is the standard rule for recursion variables.
% The rule $\mathsf{P_{sel}}$ can be explained as follows: suppose we
% are given a process $P = \choice{A}{B}{k}{l}{\chor}$, a set of branch
% labels $\{l_i \, | \, i \in I\} $ (determined by typing) and we are
% given a proof that each $\chor_i$ satisfies $\phi_i$, then we
% certainly have a proof saying that every derivation of $P$ should
% satisfy a guard $l_i$ followed by a formula $\phi_i$.  The initiation
% and interaction rule $\mathsf{P_{init}, P_{com}}$ behave similarly to
% $\mathsf{P_{sel}}$: given an initiation/communication process in $P$
% and a proof that its continuation satisfies the proof term $\phi$, we
% can derive a proof that $P$ will first exhibit an
% initiation/communication action followed by $\phi$. The conditional
% rule $\mathsf{P_{if}}$ is standard. The subsumption rule
% $\mathsf{P_{sub}}$ is the standard consequence rule as found in Hoare
% logic.
% % and the rule $\mathsf{P_{rec}}$ just indicates that the formula
% % $\phi$ satisfied by a recursion will be also satisfied by its
% % unfolding.

% The rules for parallel composition and hiding are represented in
% $\mathsf{P_{par}}$ and $\mathsf{P_{res}}$ respectively, and they do
% not indicate the behaviour of a given choreography, but hint
% information about the structure of the process: $\mathsf{P_{par}}$
% juxtaposes the behaviour of two processes and combines their
% respective formulae by the use of a separation operator;
% $\mathsf{P_{res}}$ hides a variable $x$ in a formula $\phi$: The
% intuition is that since $\new x \chor$ is a choreography $\chor$ with
% $x$ restricted, if $\chor$ proves $\phi$ and $x$ is a fresh variable
% then $\new x \chor$ should satisfy $\phi$ with hidden $x$.

% We now proceed to prove the soundness of the proof system with respect
% to the semantics of assertions presented before.

% % \begin{lemma}[Monotonicity] \label{Logic4Struct:lemma:monotonicity} If $C --> C'$
% %   and $C' |=_\sigma \phi'$, then $C |=_\sigma \phi$ and $\phi \supset \phi'$.
% % \end{lemma}
% % \begin{proof}
% %   We have to proceed by induction on the structure of $-->$. I have
% %   not checked the details, and probably a discussion on this would
% %   be good (I don't think monotonicity is a good name for this?)
% % \end{proof}


% \begin{lemma}[Structural congruence preserves validity]
%   \label{Logic4Struct:lemma:StructuralLogic} If $\chor\equiv\chor'$ and
%   $\chor|-_\sigma \phi$, then $\chor' |-_{\sigma'} \phi$ and $\sigma
%   \subseteq \sigma'$.
% \end{lemma}
% \begin{proof}
%   It follows trivially from the rule $\mathsf{P_{struct}}$.
% \end{proof}

% \marginpar{adapt the following proofs to the new proof system}
% \begin{theorem}[Soundness]\label{Logic4Struct:thm:soundness}
%   For any choreography $\chor$, if $\chor|-_\sigma \phi$ then
%   $\chor|=_\sigma \phi$.
% \end{theorem}
% \begin{proof}
%   It follows by induction on the derivation of $|-_\sigma$.
%   \begin{description}
%   \item {Case $\mathsf{P_{end}}$:} We have that $\chor= \INACT$, then
%     $\INACT |-_\sigma \endF$ and $\INACT |=_\sigma \endF$ immediately.
%   \item {Case $\mathsf{P_{init}}$:} We have that $\chor =
%     \init{A}{B}{a}{k} \pfx \chor'$ and for $\phi = \langle
%     \initF{A}{B}{a(k)} \rangle \phi'$, then $\chor |-_\sigma \phi$ and
%     $\chor' |-_\sigma \phi'$.  We know that $\chor
%     \action{\initF{A}{B}{a(k)}} \chor'$ and also that $\chor'|=_\sigma
%     \phi'$ by induction hypothesis.  Hence, $\chor|=_\sigma \langle
%     \initF{A}{B}{a(k)} \rangle \phi'$ follows directly by definition
%     of $|=$.
%   \item {Case $\mathsf{P_{com}}$:} Analogous to the $\mathsf{P_{init}}$ case.
%   \item {Case $\mathsf{P_{sel}}$:} We have that $\chor =
%     \choice{A}{B}{k}{l}{\chor'} $ and $\chor|-_\sigma \bigwedge_{i \in
%       I} \langle \branchF{A}{B}{l_i}{k} \rangle \phi $. Because of
%     $\mathsf{P_{sel}}$ we have that $\forall i \in I.\, \chor_i
%     |-_\sigma \phi_i$ and by induction hypothesis, $\forall i \in I.\,
%     \chor_i |=_\sigma \phi_i$.  By definition of the semantics of
%     $\langle \ell \rangle \phi$ and conjunction, then:
%     \begin{equation*}
%       \chor |=_\sigma \bigwedge_{i \in I} \langle \branchF{A}{B}{l_i}{k} \rangle
%       \phi_i \text{ iff } \bigwedge_{i \in I} (\chor
%       \action{\branchF{A}{B}{l_i}{k}} \chor' \land \chor' |=_\sigma \phi_i )
%     \end{equation*}

%     Take $\ell = \branchF{A}{B}{l_i}{k}$ for $i \in I$, then $\chor
%     |=_\sigma \bigwedge_{i \in I} \langle \ell \rangle \phi_i$ if
%     $\ell = \branchF{A}{B}{l_i}{k}$ (trivially true) and if
%     $\chor\equiv (\nu s)(\choice{A}{B}{k}{l}{\chor'} | \chor'')$ which
%     is equivalent to $\chor$ up-to substitution $[s/k]$. Moreover,
%     $\exists \chor' \text{ s.t. } \chor --> \chor' \text{ and } \chor'
%     |=_\sigma \phi_i $ holds by induction hypothesis, so we are done.


%   \item {Case $\mathsf{P_{ifT}, P_{ifF}}$:} We take only the proof for
%     $\mathsf{P_{ifT}}$, the other works similarly. We have that $\chor
%     = \itn{e}{\chor_1}{\chor_2}$, by induction hypothesis we have that
%     $\chor_1 |-_\sigma e => \phi$ and $\chor_2 |-_\sigma \lnot e =>
%     \phi$.  We have to show that $\chor |=_\sigma \nextOp \phi $.
%     Assume a $\sigma$ s.t. $\sigma |-_\sigma e @ A \Downarrow \true$
%     (The other case is symmetric), then we get by the use of
%     \Did{G-IfT} with $\sigma$ and $\chor$ that $( \sigma, \chor) -->
%     (\sigma, \chor_1)$. Additionally we have that:
%     \begin{align*}
%       \chor_1  & |-_\sigma e @ A \Downarrow \true => \phi \\
%       \chor_1 & |-_\sigma \false \lor \phi \\
%       \chor_1 & |-_\sigma \phi
%     \end{align*}
%     Then, $\chor_1 = \chor' $ in $\exists \chor'\pfx \chor' |=_\sigma
%     \phi \text{ s.t. } \chor --> \chor' $, so we are done.


%   \item {Case $\mathsf{P_{res}}$:} We have that $\chor = \new{x}
%     \chor' $ and by induction hypothesis we have that $\chor'
%     |=_\sigma \phi $. We have to show that $\chor |=_\sigma \new{x}
%     \phi$ from $\chor |-_{\sigma} \new{x} \phi$.  From
%     $\mathsf{P_{res}}$, we know that $\chor' |-_\sigma \phi$ with $k$
%     a fresh variable. By definition, $\chor\ |=_\sigma \new{x} \phi $
%     if $\chor \equiv \new{x} \chor' $ and $\chor' |=_\sigma \phi$, so
%     we are done.

 
%  %%%%%%%%%%%%%HL: Recursion -- Removed for the progress report %%%%%%%%%%%
% % 
%  % \item Case $\mathsf{P_{rec}}$. We have that $C = \mu X^A\pfx C' $
%  %   and assuming the inductive hypothesis we have that $X |-
%  %   \phi$. We have to show that $\mu X \pfx C |=_\sigma \phi $.

% %Now, consider the following chain of formulae:

% %    \begin{align*}
% %      \mu^0 X \pfx \phi & =  \true \\
% %      & \dots \\
% %      \mu^{n+1} X \pfx \phi & =  \phi[\mu^n X\pfx \phi / X]
% %      \end{align*}

% %      By induction over ordinal numbers and the induction hypothesis
% %      we obtain
% %      \begin{equation}
% %        \mu X \pfx C |-_\sigma (\mu^n X \pfx \phi)
% %        \end{equation}
% %        Which is immediately true for $n= 0 $ and if is true for $n=m$
% %        then applying the inductive hypothesis and structural
% %        congruence is also true for $n = m + 1$. We can generalise
% %        this for all ordinals in $n$, so

% %        \begin{equation}
% %          \mu X \pfx C |-_\sigma (\mu X. \phi)
% %          \end{equation}

%   \item {Case $\mathsf{P_{par}}$:} Immediate.
%     \qed
%   \end{description}


% \end{proof}

% %%%%%%%%%%%%%%%%%%%%%%%%%%%%%%%%%
% % Completeness 
% %%%%%%%%%%%%%%%%%%%%%%%%%%%%%%%%%
% The following lemmata are used to prove completeness:

% \begin{lemma}[Structural congruence preserves satisfiability]
%   If $C \equiv C '$ and $C |=_\sigma \phi$, then $C' |=_\sigma \phi$
% \end{lemma}
% \begin{proof}
%   Follows directly from Lemma~\ref{Logic4Struct:lemma:StructuralLogic} and
%   Theorem~\ref{Logic4Struct:thm:soundness}. \qed
% \end{proof}

% \begin{lemma}[Operational semantics preserves substitutions]
% If $C \action{\ell} C'$ then  $C [w/\var] \action{\ell[w/\var]}
% C'[w/\var]$ for all $\ell$ and appropriate $w$.
% \begin{proof}
%   Follows by rule induction over $\action{\ell}$. \qed
% \end{proof}
% \end{lemma}
% \begin{lemma}[Substitution lemma] \label{Logic4Struct:lemma-substitution}
% For any choreography C; if $C |=_\sigma \phi$ then $C |=_\sigma \phi
% [w/\var]$ for an appropriate $w$.
% \end{lemma}
% \begin{proof}
% Follows by induction on the satisfaction relation of $C |=_\sigma
% \phi$. 
% % (The cases are in my notebook, yet to put them on latex). 
% \qed
% \end{proof}


% \begin{theorem}[Completeness]
%   For any choreography $C$, if $C |=_\sigma \phi$ then $C |-_\sigma \phi$.
% \end{theorem}

% \begin{proof}
%   By rule induction on the derivation of $|=_\sigma$.


%   \begin{itemize}
%   \item $(C |=_\sigma \true)$. Follows immediately from the application of $ \mathsf{T_{sub}}$.

%   \item $(C |=_\sigma \langle \ell \rangle \phi ')$. Take $C \action{\ell} C'
%     \land C' |=_\sigma \phi' $, we have to show that $C |-_\sigma \langle \ell
%     \rangle \phi'$.

%     We proceed by case analysis of $\ell$. 

%     \begin{itemize}
%       \item Take $\ell = \initF{A}{B}{a}$, then we know that $C |=_\sigma \langle
%     \ell \rangle \phi$ iff $C \action{\initF{A}{B}{a(k)}} D$ and $D |=_\sigma
%     \new{k} \phi$. We can also infer that $C = \init{A}{B}{a}{k} \pfx
%     C'$ and, by the application of $\Did{G-{init}}$ we get that $C
%     \action{\ell} D$ with $D = \new{k} C'$. From the application of
%     $\mathsf{P_{res}}$ and the induction hypothesis $\new{k}C' |-_\sigma
%     \new{k} \phi$ we get that $C' |-_\sigma \phi$, which is the necessary
%     condition to apply $\mathsf{P_{init}}$ over $C$, so we are done.

%   \item Take $\ell = \comF{A}{B}{k}$.  Then we know that $C =
%     \interact{A}{B}{k}{e}{x} \pfx C'$ and $C \action{\ell} C'$. 
%     $C |=_\sigma \langle \ell \rangle \phi ' $ iff $C \action{\ell} C'$ and $C'
%     |=_\sigma \phi' $. We can apply the induction hypothesis $C' |-_\sigma \phi'$
%     along $\mathsf{P_{com}}$ to get $C =
%     \interact{A}{B}{k}{e}{x} \pfx C' |-_\sigma \langle \comF{A}{B}{k} \rangle
%     \phi'$, so we are done.

%     A similar reasoning also holds for the case where $\ell =
%     \branchF{A}{B}{l}{k}$.
% \end{itemize}
% %%%%%%%%%%%%%%%%%%%%%%%%%%%%%%%%%%%%%%%%%%%%%%%%%%%
% % \nextOp \phi is just a generalisation from \langle \ell \rangle \phi
% % with free \ell, therefore this case is excluded from the proof.
% %%%%%%%%%%%%%%%%%%%%%%%%%%%%%%%%%%%%%%%%%%%%%%%%%%%

%   \item $(C |=_\sigma \new{x} \phi')$. Take $C \equiv (\nu v) C'$ and $C' |=_\sigma
%     \phi' (x \mapsto v)$. By structural congruence we have that
%     $\new{u} C' \equiv_\alpha \new{x} C'$ with fresh $x$ on $C'$, then
%     we can apply $\mathsf{P_{res}}$ with the induction hypothesis $C'
%     |-_\sigma \phi'$ to get $\new{x} C' |-_\sigma \new{x} \phi'$ and we are done.

%   \item $(C |=_\sigma \phi \land \chi)$. Follows immediately from the
%     application of the induction hypothesis $C |-_\sigma \phi \land C |-_\sigma
%     \chi$ and $\mathsf{P_{and}}$.

%   \item $(C |=_\sigma \phi \pp \chi)$. Immediate given $C \equiv C_1 \pp C_2$
%     and the application of the induction hypothesis $C_1 |-_\sigma \phi \land
%     C_2 |-_\sigma \chi$ with $\mathsf{P_{par}}$.

%   \item $(C |=_\sigma <<>> \phi)$. Take $(\sigma, C) --> ^* (\sigma',
%     C') $ and $C' |=_{\sigma'} \phi$, we have to show that $C |-_\sigma
%     <<>> \phi$.  We can express $(\sigma, C) -->^* (\sigma', C')$ as a
%     finite sequence of transitions $(\sigma, C) -->^n (\sigma',
%     C')$. We proceed by induction on $n$. 

%     If $n=1$, then $(\sigma, C) -->^1 (\sigma', C')$ and then the same
%     cases for $(\sigma, C) \action{\ell} (\sigma', C')$ can be
%     applied, so the proof will consider the same cases as the one in
%     $(C |=_\sigma \langle \ell \rangle \phi)$. Because we know
%     $\langle \ell \rangle \phi => \may \phi$, then we know that $C
%     |=_\sigma \may \phi$ and we are done.

%     If $n > 1$, then we know that $(\sigma, C) -->^{n-1} (\sigma'',
%     C'') --> (\sigma', C')$. By Induction hypothesis, we know that
%     $C'' |=_{\sigma''} \may \phi$. We need to show that $C
%     |=_{\sigma} \may \phi$. From $(\sigma'', C'') --> (\sigma', C')$
%     with any $\ell$, we know that $C |=_{\sigma} \nextOp \may \phi$,
%     which is subsumed by $\may \phi$ and we are done.

%   \item $(C |=_\sigma e_1 = e_2)$. Follows immediately by the
%     application of $\mathsf{P_{exp}}$.

%   \item $(C |=_\sigma \exists \var \pfx \phi)$.
%     Take $C[w/\var] |=_\sigma \phi$ for some appropriate $w$. We want
%     to show that $C |-_\sigma \exists \var \pfx \phi$. 
%     From lemma \ref{Logic4Struct:Logic4Struct:lemma-substitution} we have that $C |=_\sigma \phi
%    [w/\var]$.
% %  \item $(C |=_\sigma \phi \wand \chi)$.

%   \item $(C |=_\sigma \lnot \phi)$. TBD.

%   \end{itemize} \qed
% \end{proof}




%%% Local Variables: 
%%% mode: latex
%%% TeX-master: "../Thesis"
%%% End: 
