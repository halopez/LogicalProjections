\section{Undecidability of Global Logic} \label{Logic4Struct:sec:undecidability}
%
In this section we focus on the undecidability of the global logic for
the global calculus with recursion given in
Section~\ref{Logic4Struct:sec:globalCalc}.  In order to prove that the global logic
is undecidable, we use a reduction from the Post Correspondence
Problem (PCP)~\cite{Post:pcp} similarly to the one proposed in
\cite{ct:csl01}. The idea is to encode in the global calculus a
``program'' which simulates the construction of PCP.
%
We first give a formal definition of the PCP.  In the sequel, $\cdot$
denotes word concatenation.
%
\begin{definition}[PCP]
  Let $s,t,\ldots$ range over $\Sigma^*$ where $\Sigma = \{0, 1\}$ and
  let $\epsilon$ be the empty word.  An instance of PCP is a set of
  pairs of words $\{(s_1,t_1), \ldots, (s_n,t_n)\}$ over
  $\Sigma^*\times\Sigma^*$.  The Post Correspondence Problem is to
  find a sequence $i_0,i_1,\dots,i_k$ ($1 \leq i_j \leq n$ for all
  $0\leq j \leq k$) such that $s_{i_0}\cdot \ldots \cdot s_{i_k} =
  t_{i_0}\cdot \ldots \cdot t_{i_k}$.
\end{definition}

\NI Intuitively, PCP consists of finding some string in $\Sigma^*$
which can be obtained by the concatenation $s_{i_0}\cdot \ldots \cdot
s_{i_k}$ as well as by $t_{i_0}\cdot \ldots \cdot t_{i_k}$. This
problem is undecidable~\cite{Post:pcp}.
%
Our goal is to find a GC term that takes a random pair of words from
an instance of PCP and append them to an ``incremental pair'' of words
which encodes the current state of the sequences $s_{i_0}\cdot \ldots
\cdot s_{i_k}$ and $t_{i_0}\cdot \ldots \cdot t_{i_k}$.  Technically,
we need a choreography that assigns randomly a natural number in $\{1,
\dots, n\}$ to a variable $r$ in some participant $B$, and another
choreography that picks a pair of words from the PCP instance,
accordingly to value in the variable $r@B$, and then appends them to
the ``incremental pair'' of words in $A$.
Formally, % We suggest the following
% encoding:
\begin{definition}[Encoding of PCP]\label{Logic4Struct:def:PCPencoding}

  Let $A_1,\dots,A_n,A,B$ be participants and let $a,b$ be shared names for
  sessions, then define the two choreographies as shown below:
  \begin{align*}
    &
    \begin{array}{rcl}
      \textsf{Random}(A_1,\dots,A_n,B,a) &
      \DEFEQ &
      \phantom{{}\pp\ {}}\mu X \pfx
      \init{A_1}{B}{a}{k} \pfx
      \interact{A_1}{B}{k}{1}{r} \pfx X\\[1mm]
      &&
      \pp\ \mu X \pfx
      \init{A_2}{B}{a}{k} \pfx
      \interact{A_2}{B}{k}{2}{r} \pfx X\\[1mm]
      &&
      \pp\ \ldots\\[1mm]
      &&
      \pp\ \mu X \pfx
      \init{A_n}{B}{a}{k} \pfx
      \interact{A_n}{B}{k}{n}{r} \pfx X
      \end{array}
\end{align*}
\begin{align*}
    &
    \begin{array}{rcl}
      \textsf{Append}(A,B,b) &
      \DEFEQ &
      \mu X\pfx
      \init{A}{B}{b}{k} \pfx
      \interact{A}{B}{k}{str1}{tmp1} \pfx
      \interact{A}{B}{k}{str2}{tmp2} \pfx {} \\
      &
      &
      \mathbf{if}\ r@B = 1\ \mathbf{then} \\
      &
      &
      \quad
      \interact{B}{A}{k}{tmp1 \cdot s_1}{str1} \pfx
      \interact{B}{A}{k}{tmp2 \cdot t_1}{str2} \pfx X \\
      &
      &
      \mathbf{else}\ \mathbf{if}\ r@B = 2\ \mathbf{then} \\
      &
      &
      \quad
      \interact{B}{A}{k}{tmp1 \cdot s_2}{str1} \pfx
      \interact{B}{A}{k}{tmp2 \cdot t_2}{str2} \pfx X \\
      &
      &
      \mathbf{else} \ \mathbf{if}\ r@B = 3\ \mathbf{then} \\
      &
      &
      \qquad \vdots \\
      &
      &
      \mathbf{else} \ \mathbf{if}\ r@B = n\ \mathbf{then} \\
      &
      &
      \quad
      \interact{B}{A}{k}{tmp1 \cdot s_n}{str1} \pfx
      \interact{B}{A}{k}{tmp2 \cdot t_n}{str2} \pfx X \\
      &
      &
      \mathbf{else }\ X
    \end{array}
  \end{align*}
  We define the initial configuration $(\sigma,\chor)$ to be formed by
  the choreography and the state below:
  {\small
  \begin{align*}
    \chor & \DEFEQ
    \textsf{Random}(A_1,\dots,A_n,B,a) \pp  \textsf{Append}(A,B,b) \\
    \sigma & \DEFEQ
    [str1@A \mapsto \epsilon,\ str2@A \mapsto \epsilon,\
    tmp1@B \mapsto \epsilon,\ tmp2@B \mapsto \epsilon,\
    r@B \mapsto 1] \, .
  \end{align*}
  }
  For encoding the PCP existence question ($s_{i_0}\cdot \ldots \cdot
  s_{i_k} = t_{i_0}\cdot \ldots \cdot t_{i_k}$) we can encode it as a
  \GL formula:
  {\small
  \begin{equation*}
    \phi \DEFEQ
    <<>> \Big(
    (str1@A = str2@A) \land
    (str1@A \neq \epsilon) \land
    (str2@A \neq \epsilon)
    \Big) \, .
  \end{equation*}
  }
\end{definition}
\NI Above, each participant $A_i$ (with $i\in \{1,\dots,n\}$)
recursively opens a session with participant $B$ and writes in the
variable $r@B$ the value $i$. Moreover, the participant $B$ stores the
knowledge of all the word pairs $(s_i,t_i)$, while the participant $A$
takes randomly a word pair from $B$ and then append it to his
incremental pair of words: $(str1,str2)$.  Next, the formula $\phi$
states that there exists a computational path from the initial
configuration to a configuration which stores in $str1$ and $str2$ two
equal non-empty strings.

\begin{theorem}
  The global logic is undecidable for the global calculus with
  recursion.
\end{theorem}
\begin{proof} 
  (Sketch) The statement $\chor |=_\sigma \phi$ holds iff the encoded
  PCP has a solution.  Indeed, if the initial configuration
  $(\sigma,\chor)$ satisfies the formula $\phi$ then it means there
  exists a configuration $(\sigma',\chor')$ where $(str1@A = str2@A)
  \land (str1@A \neq \epsilon) \land (str2@A \neq \epsilon)$
  holds. Hence, there is a sequence of $i_0,\dots,i_k$ such that $str1
  = s_{i_0}\cdot \ldots \cdot s_{i_k} = t_{i_0}\cdot \ldots \cdot
  t_{i_k} = str2$, that is, the instance of PCP has a solution.
\end{proof}


\begin{remark}\label{Logic4Struct:remark:vars}
  The undecidability result presented in this section shows that the
  global calculus is considerably expressive, in spite of the fact
  that the choreography approach offers a simplification in the
  specification of concurrent communicating systems as argued in
  \cite{carbone7scc}. The encoding in
  Definition~\ref{Logic4Struct:def:PCPencoding} shows that allowing
  state variables (hence local variables that can be accessed by
  various threads) increases the expressive power of the
  language. Indeed, we could just look at GC as a simple concurrent
  language with a ``shared'' store where assignment to variables is
  just in-session communication. In this view, we conjecture that
  removing variables and focusing only on communication would make the
  logic decidable.
\end{remark}
% In the next sections, we discuss when and how to decide the global
% logic formulae on a sub-calculus of the global calculus.

%%% Local Variables: 
%%% mode: latex
%%% TeX-master: "main"
%%% End: 
