{\large\textbf{Abstract:}} 
% \begin{abstract}
%   We explore logical reasoning for the global calculus, a coordination
%   model based on the notion of choreography, with the aim to provide a
%   methodology for specifying and verifying structured
%   communications.  Starting with an extension of Hennesy-Milner logic,
%   we propose a proof system for the logic that allows for verification
%   of properties among participants in a choreography. Additionally,
%   some examples of properties on service specifications are drawn, and
%   we provide hints on how this work can be extended towards a full
%   verification framework.
We present a framework integrating  imperative and declarative views
for structured communications. Starting from languages for the
specification of services, we provide a modal logic characterisation
of the interactions occurring in a system, both from a global
standpoint and at the level of the individual participants. The framework copes
with two aims: exhibiting logical guarantees about the presence of an
interaction, and model generation from logical specifications
% \footnotemark \footnotetext{ This work
%   is an extended version of Chapter \ref{chap:logic4chor}. In particular, sections
%   \ref{Logic4Struct:sec:globalCalc} -- \ref{Logic4Struct:sec:proofSys}
%   contain simplified versions of the results in Chapter \ref{chap:logic4chor}, and readers
%   can refer to such chapter for its full explanation.
% }
.

%\keywords{Choreography, Logic, Session Types, Web Services}
% \end{abstract}


%%% Local Variables: 
%%% mode: latex
%%% TeX-master: "../../Thesis"
%%% End: 
