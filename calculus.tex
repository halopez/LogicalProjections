\section{The Global Calculus}\label{Logic4Struct:sec:globalCalc}

The Global Calculus (GC) \cite{CHY:dcm2006,carbone7scc} originates
from the Web Service Choreography Description Language (WS-CDL)
\cite{kavantzas2004web}, a description language for web services
developed by W3C. Terms in GC describe choreographies as interactions
between participants by means of message exchanges. The description of
such interactions is centred on the notion of \emph{session}, in
which two interacting parties first establish a private connection via
some public channel and then interact through it, possibly interleaved
with other sessions. More concretely, an interaction between two
parties starts by the creation of a fresh session identifier, that
later will be used as a private channel where meaningful interactions
take place. Each session is fresh and unique, so each communication
activity will be clearly separated from other interactions. In this
section, we provide an operational semantics for GC in terms of a
label transition system (LTS) \cite{plotkin81structural} describing
how global descriptions evolve, and the type discipline that describes
the structured sequence of message exchanges between participants from
\cite{carbone7scc}.

\subsection{Syntax}

Let $\chor, \chor', \ldots$ denote {\em terms} of the calculus, often
called {\em interactions} or {\em choreographies}; $A, B, C, \ldots$
range over {\em participants}; $k,k', \ldots$ are {\em linear
  channels}; $a,b,c,\ldots$ {\em shared channels}; $v,w,\ldots$
variables; $X,Y,\ldots$ process variables; $l,l_i,\ldots$ labels for
branching; and finally $e, e', \ldots$ over unspecified arithmetic and
other first-order expressions. We write $e@A$ to denote that the
expression $e$ is evaluated using the variable related to participant
$A$ in the store.

\begin{definition}\label{Logic4Struct:definition:globalCalc}
  The syntax of the global calculus \cite{CHY:dcm2006} is given by the
  following grammar:
  \begin{alignat}{3}
    \chor ::= \ & \phantom{{}\mid{}}
    &\ & \INACT                          \label{Logic4Struct:eq:Ginaction} \tag{inaction} \\
    &\mid && \init{A}{B}{a}{k}\pfx \chor \label{Logic4Struct:eq:Ginit} \tag{init} \\
    &\mid && \interact{A}{B}{k}{e}{y}\pfx\chor  \label{Logic4Struct:eq:Gcom} \tag{com} \\
    &\mid && \choice{A}{B}{k}{l}{\chor}   \label{Logic4Struct:eq:Gchoice} \tag{choice} \\
    &\mid && \chor_1 \pp \chor_2          \label{Logic4Struct:eq:Gpar} \tag{par} \\
    &\mid && \itn{e@A}{\chor_1}{\chor_2}  \label{Logic4Struct:eq:Gcond} \tag{cond}\\
    &\mid && X                            \label{Logic4Struct:eq:Grecvar} \tag{recvar} \\
    &\mid && \mu X\pfx C                  \label{Logic4Struct:eq:Grec} \tag{recursion}
  \end{alignat}
\end{definition}
\NI Intuitively, the term (\ref{Logic4Struct:eq:Ginaction}) denotes a
system where no interactions take place. (\ref{Logic4Struct:eq:Ginit})
denotes a session initiation by $A$ via $B$'s service channel $a$,
with a fresh session channel $k$ and continuation $\chor$. Note that
$k$ is bound in $\chor$.  (\ref{Logic4Struct:eq:Gcom}) denotes an
in-session communication of the evaluation (at $A$'s) of the
expression $e$ over a session channel $k$. In this case, $y$ does not
bind in $\chor$ (our semantics will treat $y$ as a variable in the
store of $B$).  (\ref{Logic4Struct:eq:Gchoice}) denotes a labelled
choice over session channel $k$ and set of labels $I$. In
(\ref{Logic4Struct:eq:Gpar}), $\chor_1 \pp \chor_2$ denotes the
parallel product between $\chor_1$ and $\chor_2$.
(\ref{Logic4Struct:eq:Gcond}) denotes the standard conditional
operator where $e@A$ indicates that the expression $e$ has to be
evaluated in the store of participant $A$.  In
(\ref{Logic4Struct:eq:Grec}), $\mu X\pfx\chor$ is the minimal fix
point operation for recursion, where the variable $X$ of
(\ref{Logic4Struct:eq:Grecvar}) is bound in $\chor$.  The free and
bound session channels, names, and term variables are defined in the
usual way, and are denoted as $fsc(\chor), bsc(\chor), fn(\chor),
fv(\chor)$.  

\begin{definition}[Structural congruence in GC]
GC is equipped with a standard structural
congruence $\equiv$, defined as the minimal congruence relation on
interactions $\chor$, such that $\equiv$ is a commutative monoid with
respect to $\pp$ and $\INACT$, it is closed under alpha equivalence
$\equiv_\alpha$ of terms, and it is closed under the recursion
unfolding, i.e., $ \mu X. \chor \equiv \chor \MSUBS{\mu X. \chor}{X}$.
\end{definition}

\begin{remark}[Differences with the approach in \cite{carbone7scc}]
  \label{Logic4Struct:remark::one}
  The syntax in Definition \ref{prelim:definition:globalCalc} presents
  a simplified version of the global calculus without restriction,
  summation and local assignments. In its original presentation, restriction is used only during session
  initiation. We capture the requirement of fresh identifiers by using
  the operational rules in Figure \ref{Logic4Struct:table:global:semantics}.
  Excluding the lack of local assignment, we argue that our version of
  GC is, to some extent, as expressive as the one originally reported
  in \cite{carbone7scc}.  In particular, the interaction
  process $\interact{A}{B}{k}{\mathsf{op},e}{y}$  as originally defined
  captures both selection and message passing which are instead
  disentangled in our case (mainly for reasons of clarity). The absence
  of $\mathsf{op}$ in the interaction process
  $\interact{A}{B}{k}{e}{y}$ can be easily encoded with the existing
  operators. In fact, $\interact{A}{B}{k}{\mathsf{op},e}{y} \pfx
  \chor'$ can be decomposed into $\choice{A}{B}{k}{op}{\chor'} \pfx
  \interact{A}{B}{k}{e}{y} \pfx \chor'$ with unary $I$ (although we
  lose atomicity).
\end{remark}



\subsection{Semantics}

We give the operational semantics in terms of configurations $(\sigma,
\chor)$, where $\sigma$ represents the state of the system and $\chor$
the choreography actually being executed. The state $\sigma$ contains
a set of variables labelled by participants. As described in the
previous subsection, a variable $x$ located at participant $A$ is
written as $x@A$.  The same variable name labelled with different
participant names denotes different variables (hence $\sigma(x@A$) and
$\sigma(x@B)$ may differ).  Formally, the operational semantics is
defined as a labelled transition system (LTS). A transition
$(\sigma,\chor) \action{\ell} (\sigma',\chor')$ says that a
choreography $\chor$ in a state $\sigma$ executes an action (or label)
$\ell$ and evolves into $\chor'$ with a new state $\sigma'$. Actions
are defined as
$\ell \in \{\initF{A}{B}{a(k)},\comF{A}{B}{k},\branchF{A}{B}{k}{l_i}\}$,
denoting initiation, in-session communication and branch selection,
respectively.  We write $(\sigma, \chor) \action{} (\sigma', \chor')$
when $\ell$ irrelevant, and $\action{}^*$ denotes the transitive
closure of $\action{}$. The transition relation $\action{}$ is defined
as the minimum relation on pairs state/interaction satisfying the
rules in Figure~\ref{Logic4Struct:table:global:semantics}.

\begin{myfigure}{t}
{\small
  \begin{align*}
&\myrule{    G-Init}
{ h \text{ fresh} } { ({\sigma,\init
        {A}{B}{a}{k}\pfx \chor}) \action{\initF{A}{B}{a(k)}}
      ({\sigma,\chor[h/k]}) }
\quad \myrule{    G-Struct}
{\chor\equiv \chor'' \quad
      ({\sigma,\chor}) \action{\ell} ({\sigma', \chor'}) \quad
      \chor'\equiv \chor'''} {({\sigma,\chor''}) \action{\ell}
      ({\sigma',\chor'''})}
   \\&
\myrule{    G-Choice}
{}
    {({\sigma,\choice{A}{B}{k}{l}{\chor}})
      \action{\branchF{A}{B}{k}{l_i}} ({\sigma,\chor_i})}
\quad 
\myrule{    G-Par}
{({\sigma,\chor_1}) \action{\ell}
      ({\sigma',\chor_1'})} {({\sigma,\chor_1\pp \chor_2})
      \action{\ell} ({\sigma',\chor_1'\pp \chor_2})}
    \\&
\myrule{    G-IfF}
{\sigma(e@A)\Downarrow \false \quad
      (\sigma, \chor_2) \action{\ell} (\sigma', \chor'_2)} { (\sigma,
      \itn{e@A}{\chor_1}{\chor_2}) \action{\ell}(\sigma',\chor'_2)}
\quad
\myrule{    G-IfT}
{\sigma(e@A)\Downarrow \true \quad
      (\sigma, \chor_1) \action{\ell} (\sigma', \chor'_1)} { (\sigma,
      \itn{e@A}{\chor_1}{\chor_2}) \action{\ell} (\sigma', \chor'_1) }
\end{align*} \vspace{-0.5cm}\begin{align*}
\myrule{    G-Com}
{\sigma(e@A)\Downarrow v} {
      ({\sigma,\interact{A}{B}{k}{e}{x}\pfx \chor})
      \action{\comF{A}{B}{k}} ({\sigma[x@B \mapsto v],\chor}) }
 \end{align*}
}
  \caption{Operational Semantics for the Global Calculus}
  \label{Logic4Struct:table:global:semantics}
\end{myfigure}

Intuitively, transition \Did{G-Init} describes the evolution of a
session initiation: after $A$ initiates a session with $B$ on service
channel $a$, $A$ and $B$ share the fresh channel $h$ locally.
% This will be denoted by a restriction of the session channel $k$
% over the continuation $C$, as denoted by $\new{k} C$.  The
% initiation channel $a$ will play an important role for typing and
% the end-point projection later.
\Did{G-Com} describes the main interaction rule of the calculus: the
expression $e$ is evaluated into $v$ in the $A$-portion of the state
$\sigma$ and then assigned to the variable $x$ located at $B$
resulting in the new state $\sigma [x@B \mapsto v]$. \Did{G-Choice}
chooses the evolution of a choreography resulting from a labelled
choice over a session key $k$. \Did{G-IfT} and \Did{G-IfF} show the
possible paths that a deterministic evolution of a choreography can
produce. \Did{G-Par} and \Did{G-Struct} behave as the standard rules
for parallel product and structural congruence, respectively.

\begin{remark}[Global Parallel]
  Parallel composition in the global calculus differs from the notion
  of parallel found in standard concurrency models based on
  input/output primitives \cite{milner:99:cmspc}. In the latter, a
  term $P_1\pp P_2$ may allow {\em interactions} between $P_1$ and
  $P_2$. However, in the global calculus, the parallel composition of
  two choreographies $\chor_1\pp \chor_2$ concerns two parts of the
  described system where {\em interactions} may occur in $\chor_1$ and
  $\chor_2$ but never across the parallel operator $\pp$. This is
  because an interaction $A\rightarrow B\ldots$ abstracts from the
  actual end-point behaviour, i.e., how $A$ sends and $B$ receives. In
  this model, dependencies between two choreographies can be expressed
  by using variables in the state $\sigma$.
\end{remark}

In its original presentation \cite{carbone7scc}, GC comes equipped
with a reduction semantics (presented for completeness in Figure
\ref{Logic4Struct:table:global:reduction-semantics}).
% unlike the one presented in Figure
% \ref{Logic4Struct:table:global:semantics}. 
Our LTS semantics has the advantage of
allowing us to observe the changes in the behaviour of the system by
presenting labels at each point of evolution. This  will
prove useful when relating to the logical characterisation in Section
\ref{Logic4Struct:sec:globalLogic}. 


\begin{myfigure}{t}
{\small
  \begin{align*}
&\myruleg{G-RInit}
{ h \text{ is fresh }  } { ({\sigma,\init {A}{B}{a}{k}\pfx \chor})  
          -->
      ({\sigma, \chor[h/k]}) }
\quad \myruleg{G-RStruct}
{\chor\equiv \chor'' \quad
      ({\sigma,\chor})  -->  ({\sigma', \chor'}) \quad
      \chor'\equiv \chor'''} {({\sigma,\chor''})  --> 
      ({\sigma',\chor'''})}
   \\[0.3cm]&
% \myruleg{G-RSum}
% { (\sigma, \chor_i) --> (\sigma', \chor' ) \quad i \in I}
%     {(\sigma, \sum_{i \in I} \chor_i)
%       --> ({\sigma',\chor'})}
% ~ 
\qquad \qquad \myruleg{G-RRec}
{ (\sigma, \chor [\rec X. \chor/\chor]) --> (\sigma', \chor' ) }
    {(\sigma, \rec X. \chor)
      --> ({\sigma',\chor'})}
\qquad 
\myruleg{G-RPar}
{({\sigma,\chor_1}) --> 
      ({\sigma',\chor_1'})} {({\sigma,\chor_1\pp \chor_2})
       -->  ({\sigma',\chor_1'\pp \chor_2})}
% ~
% \myruleg{G-RRes}
% {({\sigma,\chor}) --> ({\sigma',\chor'})}
%       {({\sigma, \new{k}\chor})
%        -->  ({\sigma', \new{k}\chor'})}
   \\[0.3cm]&
\myruleg{G-RIfT}
{\sigma(e@A)\Downarrow \true} { (\sigma,
      \itn{e@A}{\chor_1}{\chor_2})  --> (\sigma, \chor_1) }
\quad
\myruleg{G-RIfF}
{\sigma(e@A)\Downarrow \false } { (\sigma,
      \itn{e@A}{\chor_1}{\chor_2})  --> (\sigma,\chor_2)}
\end{align*} \vspace{-0.5cm}\begin{align*}
\myruleg{G-RCom}
{\sigma(e@A)\Downarrow v} {
      ({\sigma,\interact{A}{B}{k}{\mathsf{op},e}{x}\pfx \chor})
       -->  ({\sigma[x@B \mapsto v],\chor}) }
 \end{align*}}
  \caption{Reduction Semantics for the Global Calculus \cite{carbone7scc}}
  \label{Logic4Struct:table:global:reduction-semantics}
\end{myfigure}


% We conjecture that our proposed LTS semantics
% and the reduction semantics of the global calculus originally
% presented in \cite{carbone7scc} coincide (taking into account the
% considerations in Remark~\ref{Logic4Struct:remark::one}).




As it is expected, the labelled transition semantics above presented
and the reduction semantics of the global calculus originally
presented in \cite{carbone7scc}  coincide (taking into account the
considerations in Remark~\ref{Logic4Struct:remark::one}).

\begin{lemma} \label{struct:gc-opsem} if $\chor \equiv \chor''$ and
  $(\sigma,\chor) \action{\ell} (\sigma',\chor')$ then
  $(\sigma,\chor'') \action{\ell}(\sigma',\chor')$
  \begin{proof}
    It follows by induction on the structural congruence rules in
    $\equiv$ and second induction on the height of the derivation tree
    for $(\sigma,\chor) \action{\ell} (\sigma',\chor')$.
  \end{proof}
\end{lemma}

\begin{proposition}[Reduction and LTS semantics coincide in GC]
  Given $\chor, \chor'$ processes, $->$ the reduction relation between
  global calculus processes 
%  in \cite{carbone7scc}  (Included in
%   Appendix \ref{Logic4Struct:appendix:GlobalC-Rsemantics})
  and
  $\action{\ell}$ the labelled transition relation in Figure
  \ref{Logic4Struct:table:global:semantics}. We can say that:

\begin{description}
  \item [Soundness]: If $(\sigma, \chor) -> (\sigma', \chor')$, then
    $\exists \ell $ s.t. $ (\sigma, \chor) \action{\ell}^*
    (\sigma', \chor'')$ and $\chor' \equiv \chor''$.
  \item [Completeness]: For any $\ell$, if $(\sigma, \chor) \action{\ell} (\sigma',
    \chor')$ then $(\sigma, \chor) -> (\sigma',  \chor')$. 
\end{description}
\begin{proof} \hfill
\begin{description}

\item[On (Soundness):] \hfill \\ 
The proof proceeds by induction on the height of the derivation tree
for $(\sigma,\chor) -> (\sigma',\chor')$,
with a case analysis on the last applied rule. 
% The proof proceeds by induction on the length of
% derivation in $->$.
 \begin{description}
   \item[Case \Did{RG-Init}]: There must be the case that $\chor =
     \init{A}{B}{a}{k} \pfx \chor'$ and $(\sigma, \chor) ->(\sigma',
     %\new{k}
      \chor')$. One can easily show that for $\chor$ and $\ell=
     \initF{A}{B}{a(k)}$, $(\sigma, \chor)->(\sigma',\chor')$ which is
     structurally congruent to $%\new{k} 
     \chor'$ with fresh $k$ on
     $\chor'$.

   \item[Case \Did{RG-Comm}:] There must be the case that $\chor =
     \interact{A}{B}{k}{\mathsf{op}, e}{y} \pfx \chor'$ can be
     decomposed into $\choice{A}{B}{k}{op}{\chor'} \pfx
     \interact{A}{B}{k}{e}{y} \pfx \chor'$ (Remark \ref{Logic4Struct:remark::one}),
     so the reduction of $\chor$ will be given by $(\sigma, \chor) ->
     ^*(\sigma, \chor')$.

     By \Did{RG-Comm} one get that $(\sigma, \chor )->(\sigma [x@B |->
     v], \chor' )$, and one can easily show that for such a $\chor$,
     there is a transition $(\sigma, \chor) \action{\ell} ^* (\sigma,
     \chor')$, with $\ell$ variable depending on whether $\mathsf{op}$
     is used or not: 
     \begin{itemize}
       \item If $\mathsf{op} \not = \emptyset$, then we know that the behaviour
         maps to a deterministic choice, therefore $\ell =
         \branchF{A}{B}{k}{l}$ and $(\sigma,\chor)
         \action{\ell}(\sigma'', \chor'') $ $\action{\comF{A}{B}{k(x)}} (\sigma' [x@B |->
     v], \chor')$.
       \item If $\mathsf{op} = \emptyset$, then we know that the behaviour
         maps to message passing, therefore $\ell =
         \comF{A}{B}{k(x)}$ and $(\sigma,\chor)
         \action{\branchF{A}{B}{k}{l}}(\sigma'', \chor'') $ $\action{\ell} (\sigma' [x@B |->
     v], \chor')$.
      \end{itemize}
    \item[Case \Did{RG-IfT}:] We can assume that for $\chor = \itn{
        e@A }{ \chor_1 }{ \chor_2 }$ then $\sigma |- e@A \Downarrow
      \true$ and $(\sigma,\chor) -> (\sigma',\chor_1)$, the converse
      case is symmetric.
      
        From the application of $\Did{G-IfT}$ along with the induction
        hypothesis $(\sigma,\chor_1)\action{\ell}(\sigma',\chor'_1)$
        we get that $ (\sigma,\itn{e@A}{\chor_1}{\chor_2}) \action{\ell}
        (\sigma',\chor'_1)$ which is what we had to show.
      
        
        
        \item[Case \Did{RG-Par} and \Did{RG-Rec}]: Follows directly
          by simple rule induction.
          
        \item[Case \Did{RG-Struct}:] Given $(\sigma, \chor'') ->
          (\sigma,\chor''')$ then we can assume that $\chor \equiv
          \chor''$ and $(\sigma,\chor) -> (\sigma',\chor') \land
          \chor' \equiv \chor'''$. We need to show that
          $(\sigma,\chor'') \action{\ell}^* (\sigma,\chor''')$. From
          Lemma \ref{struct:gc-opsem} and the inductive hypothesis
          $\chor \equiv \chor'' \land (\sigma,\chor)
          \action{\ell}^*(\sigma',\chor') \land \chor' \equiv \chor'''$
          then $(\sigma,\chor'') \action{\ell}^* (\sigma',\chor''')$.
 \end{description}


\item[On (Completeness):] \hfill \\
The proof proceeds by induction on the height of the derivation tree for $(\sigma,\chor) \action{\ell} (\sigma',\chor')$,
with a case analysis on the last applied rule. 
% The proof proceeds by
% %  rule induction on the
% %   lenght of the derivations in $\action{\ell}$.
% case analysis of the labels
% in $\ell$ over the transitions in $(\sigma,C) \action{\ell}
% (\sigma',C')$. 

\begin{description}
  \item[Case $\ell = \initF{A}{B}{a(k)}$:] We have that $(\sigma,
    \chor) \action{\ell} (\sigma', \chor')$ with $\chor =
    \init{A}{B}{a}{k} \pfx \chor'$. After the application of
    $\Did{RF-Init}$ we get that $(\sigma,\chor)->(\sigma,\chor''[h/k])$
    with fresh $h$, and $\chor''[h/k] \equiv \chor'$, hence we are
    done.
    \item[Case $\ell = \comF{A}{B}{k(x)}$:] We have that
      $(\sigma,\chor) ->(\sigma[x@B \mapsto v], \chor')$ with $\chor =
      \interact{A}{B}{k}{e}{x} \pfx \chor'$. After the application of
      $\Did{RG-Comm}$ with $\mathsf{op} = \emptyset$ we get that
      $(\sigma,\chor)->(\sigma[x@B \mapsto v], \chor')$.
     \item[Case $\ell = \branchF{A}{B}{k}{l}$:] We have that
       $(\sigma,\chor)\action{\ell}(\sigma,\chor_i)$ with $\chor =
       \choice{A}{B}{k}{l}{\chor}$. After the application of
       $\Did{RG-Comm}$ with $\mathsf{l_i}$ we get that $(\sigma,\chor)
       ->(\sigma[x@B \mapsto v], \chor_i)$ with $x$ an irrelevant
       variable in $B$ (extra and not used elsewhere).
\end{description}

\end{description}
\end{proof}
\end{proposition}


\begin{example}[Online Booking]\label{Logic4Struct:ex:onlinebooking}
  We consider the example presented in the introduction, i.e., a
  simplified version of the on-line booking scenario presented in
  \cite{LOP-places09}. Here, the customer (Cust) establishes a session
  with the airline company (AC) using service (on-line booking,
  shorted as ob) and creating session keys $k_1,k_2$. Once sessions
  are established, the customer will request the company about a
  flight offer with his booking data, along the session key $k_1$. The
  airline company will process the customer request and will send a
  reply back with an offer using the session key $k_2$. The customer
  will eventually accept the offer, sending back an acknowledgment to
  the airline company using $k_1$. The following specification in the
  global calculus represents the protocol:
  \begin{align} \label{Logic4Struct:example:syntax} \chor_{\mathsf{OB}} = {} &
    \init{\text{Cust}}{\text{AC}}{\text{ob}}{k_1,k_2} \pfx
    \interact{\text{Cust}}{\text{AC}}{k_1}{\text{booking}}{x} \pfx \tag{OB} \\
    & \interact{\text{AC}}{\text{Cust}}{k_2}{\text{offer}}{y} \pfx
    \interact{\text{Cust}}{\text{AC}}{k_1}{\text{accept}}{z} \pfx
    \INACT \notag \, .
  \end{align}
\end{example}


\subsection{Session Types for the Global Calculus}
\label{Logic4Struct-GlobalTypes}

The Global Calculus comes accompanied wth a type discipline that
ensures the proper control flow among interactions. It is built as a
generalisation of session types \cite{honda1998lpa} for global
interactions, first presented in \cite{carbone7scc}.  Here we
informally describe
their use through examples, and direct the reader  to their original presentation for
a more formal view. 

Session types in
GC are used to structure sequence of message exchanges in a session.
Their syntax is as follows:
\begin{align}\label{Logic4Struct:eq:sessiontypes}
  \theta =& ~~\mathtt{bool} ~|~ \mathtt{ int}  ~|~ \ldots  \notag\\
  \alpha = & ~~
  \uparrow(\theta).\alpha
  ~|~
  \downarrow(\theta).\alpha
  ~|~
  \&\{ l_i: \alpha_i\}_{i\in{I}}
  ~|~ \oplus\{ l_i: \alpha_i\}_{i\in{I}}  ~|~ \alpha_1 \pp \alpha_2 ~|~ \endT ~|~ \mu \mathbf{
    t }\pfx \alpha ~|~  \mathbf{ t }
\end{align}

Here, $\theta$ range over standard data types $\mathtt{bool, string, int,
  \dots}$ and $\alpha$ describe session types.  We describe the forms
of $\alpha$. 
% The first four types are associated with the various
% communication operations:
\begin{itemize}
\item $\downarrow(\theta).\alpha$ and $\uparrow(\theta).\alpha$ are
  the input and output types and describe the reception (resp.
  emission) of a message with data type $\theta$ followed by a
  continuation $\alpha$.
\item Similarly, $\&\{ l_i: \alpha_i\}_{i\in{I}}$ is the branching
  type while $\oplus\{ l_i: \alpha_i\}_{i\in{I}}$ is the selection
  type.
 \item The type $\alpha_1 \pp \alpha_2$ is a parallel composition of
   session types $\alpha_1$ and $\alpha_2$. 
\item The type $\endT$ indicates session termination and is often
  omitted.
\item $\mu \mathbf{t} \pfx \alpha$ indicates a recursive type with
  $\mathbf{t}$ as a type variable. $\mu \mathbf{t} \pfx \alpha$ binds
  the free occurrences of $\mathbf{t}$ in $\alpha$. We take an
  \emph{equi-recursive} view of types, not distinguishing between $\mu
  \mathbf{t} \pfx \alpha$ and its unfolding $\alpha [\mu \mathbf{t}
  \pfx \alpha / \mathbf{t}]$.
\end{itemize}

Typing judgments in GC have the form $\typerule{\typeEnv}{}{\chor \ccc
  \Delta}$, where $\typeEnv$ is a type environment describing
\emph{services},  and $\Delta$  the type environment describing 
\emph{sessions}. Typically, $\typeEnv$
contains a set of type assignments of the form $a@A: \alpha$, which
say that a service $a$ located at participant $A$ may be invoked and
run a session according to type $\alpha$. $\Delta$ contains type
assignments of the form $k[A,B]: \alpha$ which say that a session
channel $k$ identifies a session between participants $A$ and $B$ and
has session type $\alpha$ when seen from the viewpoint of
$A$. There is no particular reason why one has to choose a
  strict direction when considering interactions, and one may as well
  consider $k[A,B]: \alpha$ from the viewpoint of $B$. We return to
the specification (\ref{Logic4Struct:example:syntax}) in
Example~\ref{Logic4Struct:ex:onlinebooking}
  to see how some of the typing rules work.  One possible assignment
  for $\Delta$ is:

 \begin{equation*}
       k_1, k_2 [Cust, AC]: k_1 \downarrow
       \text{booking}(\mathtt{string}) \pfx k_2 \uparrow
       \text{offer}(\mathtt{int}) \pfx k_1 \downarrow \text{accept}(\mathtt{string}) \pfx \endT 
\end{equation*}

Describing that $k_1$ and $k_2$ are names corresponding to the same
session between participants $Cust$ and $AC$, and corresponds to the
session type $\alpha = k_1 \downarrow
       \text{booking}(\mathtt{string}) \pfx k_2 \uparrow
       \text{offer}(\mathtt{int}) \pfx k_1 \downarrow
       \text{accept}(\mathtt{string}) \pfx \endT $ when seeing it from the
       point of view of $Cust$.

%  \begin{equation*}
%    \text{ob}@\text{AC}:(k_1,k_2) \pfx k_1 \downarrow
%     \text{booking}(\mathtt{string}) \pfx k_2 \uparrow
%     \text{offer}(\mathtt{int}) \pfx k_1 \downarrow \text{accept}
%     (\mathtt{int}) \pfx \endT \, .
% \end{equation*}
% the service type of
%   the airline company at channel $ob$ can be described as:


We provide some examples of the typing rules for the GC.
The full set of
typing rules derived from the original work are attached in Appendix
\ref{Logic4Struct:appendix:global:typing}.
First, we comment the rule \Did{G-TInit}, which types the establishment of a
new session between two participants. 
\begin{align*}
\myrule{    G-TInit}
    { \typerule{\typeEnv, a@B : (\vec{k})\alpha  }{}{
        \chor  \ccc \Delta \cdot \vec{k}[B,A]: \alpha \quad A \neq B} } { \typerule{\typeEnv, a@B : (\vec{k})\alpha  }{}{
        \init{A}{B}{a}{\vec{k}}\pfx \chor  \ccc \Delta}}
\end{align*} 

Here, the typing rule dictates some requirements on the structure of
the choreography: first, the initialisation of a session between
participants in $\init{A}{B}{a}{\vec{k}}\pfx \chor$ requires that 
sessions names in $\vec{k}$ correspond to a session type in the
premise. Moreover, it checks that the service channel $a@B :
(\vec{k})\alpha$ is declared in the service typing $\typeEnv$.
The rule $\Did{G-TCom}$ describes communication between
participants: 

\begin{align*}
    \myrule{    G-TCom}
    { \typerule{\typeEnv  }{}{
        \chor   \ccc \Delta \cdot \vec{k} [A,B] :\alpha}
      \quad \typerule{\typeEnv}{}{e@A:\theta}
      \quad \typerule{\typeEnv}{}{x@B:\theta}
      \quad k \in \vec{k}
      \quad A \neq B 
    } { \typerule{\typeEnv  }{}{
        \interact{A}{B}{k}{e}{x} \pfx \chor \ccc \Delta \cdot
        \vec{k}[A,B]: k \uparrow \theta \pfx \alpha}}
\end{align*}

Here, the interaction $ \chor = \interact{A}{B}{k}{e}{x} \pfx \chor'$
will be 
typable with a session type $\Delta \cdot
        \vec{k}[A,B]: k \uparrow \theta \pfx \alpha $ provided that:
        1) The evaluation of the expression $e$ at $A$ and its
        recipient variable $x$ at $B$ correspond to the same value
        type, 2) the communication is performed between different
        participants $A$ and $B$, and 3) the continuation
        $\chor$ contains a session type  between $A$ and $B$ such
        that its session names in $\vec{k}$ contain $k$. In the
        conclusion, we use an output type $k \uparrow \theta \pfx
        \alpha$ describing the emission of value from the point of view
        of $A$. It is clear, that we could use a complementary rule to
        type the input of values from the point of view of $B$.





\begin{assumption}[Well-typedness]
Henceforth we only consider well-typed terms for the Global calculus,
unless otherwise specified.  
% We hereafter assume In the sequel, we only consider choreographies that satisfy the
%   typing discipline.
\end{assumption}



%%%%%%%%%%%%%%%%%%%%%%%%%%%%%%%%%%%%%%%%%%%%%%%%%%%%
%%% Local Variables: 
%%% mode: latex
%%% TeX-master: "../Thesis"
%%% End: 
%%%%%%%%%%%%%%%%%%%%%%%%%%%%%%%%%%%%%%%%%%%%%%%%%%%%

%%%%%%%%%%%%%%%%%%%%%%%%%%%%%%%%%%%%%%%%%%%%%%%%%%%%
%%% Local Variables: 
%%% mode: latex
%%% TeX-master: "../Thesis"
%%% End: 
%%%%%%%%%%%%%%%%%%%%%%%%%%%%%%%%%%%%%%%%%%%%%%%%%%%%
