
\subsection{Session Types for the Global Calculus}
\label{Logic4Struct-GlobalTypes}

The Global Calculus comes accompanied wth a type discipline that
ensures the proper control flow among interactions. It is built as a
generalisation of session types \cite{honda1998lpa} for global
interactions, first presented in \cite{carbone7scc}.  Here we
informally describe
their use through examples, and direct the reader  to their original presentation for
a more formal view. 

Session types in
GC are used to structure sequence of message exchanges in a session.
Their syntax is as follows:
\begin{align}\label{Logic4Struct:eq:sessiontypes}
  \theta =& ~~\mathtt{bool} ~|~ \mathtt{ int}  ~|~ \ldots  \notag\\
  \alpha = & ~~
  \uparrow(\theta).\alpha
  ~|~
  \downarrow(\theta).\alpha
  ~|~
  \&\{ l_i: \alpha_i\}_{i\in{I}}
  ~|~ \oplus\{ l_i: \alpha_i\}_{i\in{I}}  ~|~ \alpha_1 \pp \alpha_2 ~|~ \endT ~|~ \mu \mathbf{
    t }\pfx \alpha ~|~  \mathbf{ t }
\end{align}

Here, $\theta$ range over standard data types $\mathtt{bool, string, int,
  \dots}$ and $\alpha$ describe session types.  We describe the forms
of $\alpha$. 
% The first four types are associated with the various
% communication operations:
\begin{itemize}
\item $\downarrow(\theta).\alpha$ and $\uparrow(\theta).\alpha$ are
  the input and output types and describe the reception (resp.
  emission) of a message with data type $\theta$ followed by a
  continuation $\alpha$.
\item Similarly, $\&\{ l_i: \alpha_i\}_{i\in{I}}$ is the branching
  type while $\oplus\{ l_i: \alpha_i\}_{i\in{I}}$ is the selection
  type.
 \item The type $\alpha_1 \pp \alpha_2$ is a parallel composition of
   session types $\alpha_1$ and $\alpha_2$. 
\item The type $\endT$ indicates session termination and is often
  omitted.
\item $\mu \mathbf{t} \pfx \alpha$ indicates a recursive type with
  $\mathbf{t}$ as a type variable. $\mu \mathbf{t} \pfx \alpha$ binds
  the free occurrences of $\mathbf{t}$ in $\alpha$. We take an
  \emph{equi-recursive} view of types, not distinguishing between $\mu
  \mathbf{t} \pfx \alpha$ and its unfolding $\alpha [\mu \mathbf{t}
  \pfx \alpha / \mathbf{t}]$.
\end{itemize}

Typing judgments in GC have the form $\typerule{\typeEnv}{}{\chor \ccc
  \Delta}$, where $\typeEnv$ is a type environment describing
\emph{services},  and $\Delta$  the type environment describing 
\emph{sessions}. Typically, $\typeEnv$
contains a set of type assignments of the form $a@A: \alpha$, which
say that a service $a$ located at participant $A$ may be invoked and
run a session according to type $\alpha$. $\Delta$ contains type
assignments of the form $k[A,B]: \alpha$ which say that a session
channel $k$ identifies a session between participants $A$ and $B$ and
has session type $\alpha$ when seen from the viewpoint of
$A$. There is no particular reason why one has to choose a
  strict direction when considering interactions, and one may as well
  consider $k[A,B]: \alpha$ from the viewpoint of $B$. We return to
the specification (\ref{Logic4Struct:example:syntax}) in
Example~\ref{Logic4Struct:ex:onlinebooking}
  to see how some of the typing rules work.  One possible assignment
  for $\Delta$ is:

 \begin{equation*}
       k_1, k_2 [Cust, AC]: k_1 \downarrow
       \text{booking}(\mathtt{string}) \pfx k_2 \uparrow
       \text{offer}(\mathtt{int}) \pfx k_1 \downarrow \text{accept}(\mathtt{string}) \pfx \endT 
\end{equation*}

Describing that $k_1$ and $k_2$ are names corresponding to the same
session between participants $Cust$ and $AC$, and corresponds to the
session type $\alpha = k_1 \downarrow
       \text{booking}(\mathtt{string}) \pfx k_2 \uparrow
       \text{offer}(\mathtt{int}) \pfx k_1 \downarrow
       \text{accept}(\mathtt{string}) \pfx \endT $ when seeing it from the
       point of view of $Cust$.

%  \begin{equation*}
%    \text{ob}@\text{AC}:(k_1,k_2) \pfx k_1 \downarrow
%     \text{booking}(\mathtt{string}) \pfx k_2 \uparrow
%     \text{offer}(\mathtt{int}) \pfx k_1 \downarrow \text{accept}
%     (\mathtt{int}) \pfx \endT \, .
% \end{equation*}
% the service type of
%   the airline company at channel $ob$ can be described as:


We provide some examples of the typing rules for the GC.
The full set of
typing rules derived from the original work are attached in Appendix
\ref{Logic4Struct:appendix:global:typing}.
First, we comment the rule \Did{G-TInit}, which types the establishment of a
new session between two participants. 
\begin{align*}
\myrule{    G-TInit}
    { \typerule{\typeEnv, a@B : (\vec{k})\alpha  }{}{
        \chor  \ccc \Delta \cdot \vec{k}[B,A]: \alpha \quad A \neq B} } { \typerule{\typeEnv, a@B : (\vec{k})\alpha  }{}{
        \init{A}{B}{a}{\vec{k}}\pfx \chor  \ccc \Delta}}
\end{align*} 

Here, the typing rule dictates some requirements on the structure of
the choreography: first, the initialisation of a session between
participants in $\init{A}{B}{a}{\vec{k}}\pfx \chor$ requires that 
sessions names in $\vec{k}$ correspond to a session type in the
premise. Moreover, it checks that the service channel $a@B :
(\vec{k})\alpha$ is declared in the service typing $\typeEnv$.
The rule $\Did{G-TCom}$ describes communication between
participants: 

\begin{align*}
    \myrule{    G-TCom}
    { \typerule{\typeEnv  }{}{
        \chor   \ccc \Delta \cdot \vec{k} [A,B] :\alpha}
      \quad \typerule{\typeEnv}{}{e@A:\theta}
      \quad \typerule{\typeEnv}{}{x@B:\theta}
      \quad k \in \vec{k}
      \quad A \neq B 
    } { \typerule{\typeEnv  }{}{
        \interact{A}{B}{k}{e}{x} \pfx \chor \ccc \Delta \cdot
        \vec{k}[A,B]: k \uparrow \theta \pfx \alpha}}
\end{align*}

Here, the interaction $ \chor = \interact{A}{B}{k}{e}{x} \pfx \chor'$
will be 
typable with a session type $\Delta \cdot
        \vec{k}[A,B]: k \uparrow \theta \pfx \alpha $ provided that:
        1) The evaluation of the expression $e$ at $A$ and its
        recipient variable $x$ at $B$ correspond to the same value
        type, 2) the communication is performed between different
        participants $A$ and $B$, and 3) the continuation
        $\chor$ contains a session type  between $A$ and $B$ such
        that its session names in $\vec{k}$ contain $k$. In the
        conclusion, we use an output type $k \uparrow \theta \pfx
        \alpha$ describing the emission of value from the point of view
        of $A$. It is clear, that we could use a complementary rule to
        type the input of values from the point of view of $B$.





\begin{assumption}[Well-typedness]
Henceforth we only consider well-typed terms for the Global calculus,
unless otherwise specified.  
% We hereafter assume In the sequel, we only consider choreographies that satisfy the
%   typing discipline.
\end{assumption}



%%%%%%%%%%%%%%%%%%%%%%%%%%%%%%%%%%%%%%%%%%%%%%%%%%%%
%%% Local Variables: 
%%% mode: latex
%%% TeX-master: "../Thesis"
%%% End: 
%%%%%%%%%%%%%%%%%%%%%%%%%%%%%%%%%%%%%%%%%%%%%%%%%%%%
