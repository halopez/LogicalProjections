\section{\texorpdfstring{\LL:}{LL:}  A logic for End Points}
\label{Logic4Struct:sec:localLogic}
% \subsection{Syntax}
In this section, we introduce a simple logic for
orchestrations. Having close resemblance with the global logic,
$\mathcal{LL}$ expresses properties of end-points processes at the
\emph{local level}.  This is possible by presenting a logic featuring
assertions for modalities for locations
\cite{Cardelli2006Ambient-Logic}, equality, value/name
passing and  
 spatial operators \cite{caires2001spatial}
%  and nonce creation.
% It is inspired by the modal logic for session types presented in
% \cite{Berger2008Completeness-an}.  
%fixed point formulas  and.  
The grammar of assertions is given in Figure
\ref{Logic4Struct:table:LocalLogic}.
\begin{myfigure}{t}
$$
\begin{array}{cc}

 \begin{array}{rll}
    \psi, \omega \ ::=\ &\phantom{{}~ | ~{}}
%     \true 	& \text{(true)} \\
\locatedActionF{A}{\emm}\pfx \psi      & \text{(Located action)}\\
    &~ | ~ (e_1=e_2)@A    & \text{(equality)}\\
    &~ | ~ \psi \land \omega               & \text{(conjunction)}\\
    &~ | ~ \neg \psi                 & \text{(neg)}\\
%     &~ | ~  \psi => \omega                 & \text{(implication)}\\
    &~ | ~ \exists \var\pfx \psi     & \text{(exists)}\\
    &~ | ~ \may \psi                  & \text{(may)}\\
%        &~ | ~ \mu X\pfx \phi             & \text{(rec)}\\
%        &~ | ~ X                      & \text{(recVar)}\\
%     &~ | ~ \new{ x} \pfx \psi         & \text{(new)}\\
    &~ | ~ \psi ~|~ \omega                    & \text{(parallel)}\\
%     &~ | ~ \psi \wand \omega               & \text{(wand)}\\
    &~ | ~ \endF               & \text{(inaction)}
% \end{array}
% \begin{array}{rll}
\\\\
    \emm\ ::=\ & \phantom{{}~ | ~{}} \asynchServiceInputF{a}{k}&
    \text{(Service Init Input)}\\
    &~ | ~ \asynchServiceOutF{a}{k}&    \text{(Service Init Output)}\\
    &~ | ~ \asynchInputF{k}{x}&    \text{(Input)}\\
    &~ | ~ \asynchOutputF{k}{x}&    \text{(Output)}\\
    &~ | ~ \asynchBranch{k}{l}&    \text{(Branching)}\\
    &~ | ~ \asynchSelection{k}{l}&    \text{(Selection)}\\
%     &~ | ~  m \oplus m &    \text{(Internal choice)}\\
%     &~ | ~ \tau& \text{(Synch)}
       \end{array}
\end{array}
$$
\caption{Syntax of  \LL }
\label{Logic4Struct:table:LocalLogic}
\end{myfigure}


The logic consists of the standard FOL operators $\land$, $\lnot$,
and the existential quantifier $\exists$. In
$\exists\var\pfx\psi$, the variable $\var$ is meant to range over
service and session channels and participants.  Accordingly, it works
as a binder in $\psi$.  In addition to the standard operators, we
include an unspecified (decidable) equality on expressions
$(e_1=e_2)$.  Our operators depend on the labels of the labelled
transition system of the end point calculus: $\locatedActionF{A}{\emm}
\pfx \psi $ represents the execution of an action $\emm$ located at
participant $A$, followed by the assertion $\psi$ execution of a
labelled action $\emm$ followed by the assertion $\psi $; $\nextOp
\psi$ and $\may \psi$ denote the standard next and eventually
operators from Linear Temporal Logic.  The spatial operator in $\psi |
\chi$ denotes composition of formulae where $\psi$ and $\chi$ do not
share variables, 
% Similarly,  $\psi \wand \chi$ denotes the
%adjoint operation for $|$ and is used to describe the  contribution of
%the context in the logic, 
as we can see in the definition of its
semantics below.

\begin{remark}[Derived Operators]
  We can get the full set of operators in \LL by standard derivation
  from the above presented set of operations:

  \begin{alignat*}{3}
    \true & = (0 = 0) &\qquad\qquad
    \false  & = (0 = 1) &\qquad
    (e_1 \neq e_2)  & = \neg (e_1 = e_2) \\
    \nextOp \psi & = \exists \ell \pfx \exists A \pfx
    \locatedActionF{A}{\emm} \pfx \psi  &
   \psi \land \omega  & = \lnot ( \lnot \psi  \lor \lnot \omega) &
    \forall x \pfx \psi   & = \neg \exists x  \pfx \lnot \omega \\
%     \phi => \chi & = \lnot \phi \lor \chi \\
    [] \psi   & = \lnot <<>> \neg \psi &
    [\locatedActionF{A}{\emm}] \pfx \psi & = \neg
    \locatedActionF{A}{\emm} \pfx \neg \psi &
    \psi => \omega & = \lnot \psi \lor \omega 
   % (\nu X(\vec{x})\pfx \phi ) & = \neg (\mu X\pfx \neg \phi )
  \end{alignat*}
\end{remark}


\subsection{Examples of formulae in \texorpdfstring{\LL}{LL}}

\paragraph{Request - Reply}

For a classical request-reply system in the End Point Calculus: 
\begin{align*}
P ~&::=  \repInitIn {p}{k_1,k_2}\pfx \send{k_1}{e_1} 
\receive{k_2}{y} \INACT \\
B ~&::=  \initOut{p}{k_1,k_2} \pfx \receive{k_1}{x} 
\send{k_2}{e_2} \INACT \\
System ~&::=  A[P]_\sigma \pp B[Q]_{\delta}
\end{align*}

We can describe some of the \LL formulae that hold for the above
system. At the participant level, we can describe a formula enforcing
an eventual response after a given message exchange:

\begin{align*}
System &|= \locatedActionF{A}{\asynchServiceOutF{a}{(k_1,k_2)}} \pfx
\locatedActionF{A}{\asynchOutputF{k_1}{e_1}} \pfx \may
\locatedActionF{A}{\asynchInputF{k_2}{x}} \pfx \endF
\end{align*}

Dually, we now look at $B$. On its composition, we can use both the parallel
product or the magic wand operation to denote the fact that both
participants should be present in the interaction providing
complementary actions. A partial description leaving out session
initiation constructions is given below:

\begin{align*}
System &|= (\locatedActionF{A}{\asynchOutputF{k_1}{e_1}} \pfx \may
\locatedActionF{A}{\asynchInputF{k_2}{x}} \pfx \endF) ~|~
%\wand
 (\locatedActionF{B}{\asynchInputF{k_1}{x}} \pfx \may
\locatedActionF{B}{\asynchOutputF{k_2}{e_2}} \pfx \endF)
\end{align*}

%The choice between $\pp$ and $\wand$ depends on how much information
%one have from the system. By $\pp$ one fixes the system to a given
%topology with two interacting parts, whereas the magic wand operation
%allows for more flexibility in the system (for instance, allowing a
%participant $C$ to be present in the system).
%\CommentHugo{Change here $\wand$ for $\pp$}


\subsection{Semantics of  \texorpdfstring{\LL}{LL}}
 \label{Logic4Struct:locallogic-assertions}
 
 We now give a formal meaning to \LL by providing a set of assertions
 describing the semantics of each of the operators of the logic with
 respect to the LTS semantics of the end point calculus described in
 the previous section.  In particular we introduce the notion of
 satisfaction.
  % Let $\sigma$ be a mapping from % participants in a choreography to
  % variable assignments.  
 We write $\network |= \psi$ whenever a network $\network$ satisfies a
 formula $\psi$ in \LL.  The relation $|=$ is the maximum relation
 satisfying the rules given in
 Figure~\ref{Logic4Struct:table:local:assertions}.  \begin{myfigure}{t}
   { \begin{displaymath} \begin{array}{llllllllllllll}
%          \network |=   \true    & \defSym &       \true         \\
        \network |= \locatedActionF{A}{\emm}\pfx \psi & \defSym &
         \begin{array}{r} 
           \network \equiv \pr{A}{P}{\sigma} \pp M \land \exists
         Q,\sigma'. \pr{A}{P}{\sigma} \action{\emm} \pr{A}{Q}{\sigma'}
         \land \pr{A}{Q}{\sigma'} \pp M |= \psi 
%          ( \emm \neq \emm_1
%          \oplus \emm_2)
         \end{array}\\
%         \network |= \locatedActionF{A}{\emm}\pfx \psi & \defSym &
%          \begin{array}{r}
%            \network \equiv \pr{A}{P_1 \oplus P_2}{\sigma} \pp M \land
%          ( \pr{A}{P_1}{\sigma} |=  \locatedActionF{A}{\emm_1}\pfx \psi
%          \lor  \pr{A}{P_2}{\sigma} |= \locatedActionF{A}{\emm_2}\pfx
%          \psi) 
% %          ( \emm = \emm_1
% %          \oplus \emm_2)
%          \end{array}\\
%          \network |= (e_1=e_2)@A & \defSym & \network \equiv
%          \pr{A}{P}{\sigma} \pp M ~ \land ~ \sigma(e_1) = \sigma(e_2)  \\
         \network |= (e_1=e_2)@A & \defSym & \network \equiv
         \pr{A}{P}{\sigma} ~ \land ~ \sigma(e_1) = \sigma(e_2)  \\
         \network |= \psi\land \rho & \defSym & \network |= \psi
         \text{ and } \network |= \rho \\
%          \network |= \psi => \omega & \defSym &\network | \text{ If } \network |=
%          \psi \text{ then } \network |= \omega \\
         \network |= \neg \psi & \defSym & \network \not|= \psi \\
         \network |= \exists\var\pfx\psi & \defSym & \network
         |= \psi[w/\var]\quad(\text{for some appropriate } w)      \\
         \network |= \may \psi & \defSym & \network
         -->^*M \ \text{ and } M |= \psi     \\
%          \network |= \new{k} \pfx \psi & \defSym & \network \equiv (\nu k)  M \;\text{ and } M |= \psi \\
         \network |= \psi \pp \rho & \defSym & \network \equiv M \pp
         M'
         \text{ s.t. } M |= \psi \text{ and } M' |= \rho    \\
%          \network |= \psi \wand \rho & \defSym & \exists M. M |= \rho\ 
%          \land\ M \pp N  |= \psi\\
%  This operator can be encoded with
%          the existential quantifier, according to davide.
         \network |= \endF & \defSym & \network \equiv \INACTNW
    \end{array}
  \end{displaymath}  }
  \caption{Assertions of the Local logic}
  \label{Logic4Struct:table:local:assertions}
\end{myfigure}


\begin{definition}[Satisfiability, Validity and Logical Equivalence in
  \LL]\
  \begin{itemize}
  \item A formula $\psi$ is \emph{satisfiable} if there exists a
    network under which it is true, that is, $\network |= \psi$ for some
    \network.
  \item A formula $\psi$ is \emph{valid} if it is true in every
    network \network, that is, $\network |= \psi$ for every
    \network.
  \item A formula $\psi$ is a \emph{logical consequence} of a formula
    $\rho$ (or $\psi$ \emph{logically implies} $\rho$), denotes with
    an abuse of notation as $\psi |= \rho$, if every network
    $\network$ that makes $\psi$ true also makes $\rho$ true up to alpha-renaming
  \item We say that a formula $\psi$ is \emph{logical equivalent} to
    a formula $\rho$, written $\psi \logicEquiv \rho$, if $\psi |= \rho$ iff
    $\rho |= \psi$.
    \item Given a set of formulae $\Psi$ and $\logicEquiv$, the
      equivalence class of $\phi \in \Psi$ is the subset of all
      elements in $\Psi$ such that are logically equivalent to $\psi$:
      \[
      [ \psi ] = \{ x \in \Psi | x \logicEquiv \psi \}
      \]
  \end{itemize}
\end{definition}


We provide some auxiliary lemmas describing the relation of the logic
with the behaviours evidenced in end-point processes. The main result
here is the preservation of satisfiability in type-bisimilar
processes. That is, if two processes are prunable (the type
bisimilarity introduced in the End Point Projection), then they
satisfy the same formula in \LL.


\begin{lemma}[Structural congruence preserves satisfiability]
  \label{Logic4Struct:lemma:StructuralSatisfiability-local} If $M\equiv\network$ and
  $M |= \psi$, then $\network |= \psi$.
\end{lemma}
\begin{proof}(Sketch)
  It follows by simple case analysis over the rules in $\equiv$. The
  proof is similar to the one in Lemma \ref{Logic4Struct:lemma:StructuralSatisfiability}.
\end{proof}

%\begin{proposition}%[Properties of $[ \phi ]$]
%  It holds that:
%     \begin{enumerate}
%       \item $\phi \in [ \phi ] $.
%        \item For any two equivalence classes $[ \phi ], [ \chi ]$, either
%          $[ \phi ]=[ \chi ]$ or $[ \phi ]$ and $[ \chi ]$ are
%          disjoint.
%          \item The set of all equivalence classes of $\Phi$ form a
%            partition of $\Phi$.
%            \item $\chi \logicEquiv \phi$ iff $[ \chi ] = [ \phi ]$.
%       \end{enumerate}
%
%\begin{proof}
%TBD
%\end{proof}
%\end{proposition}

\begin{lemma}\label{Logic4Struct:lemma:LL:implication}
   if $\network |= \psi $ and $\exists M. M \action{\emm} \network$
   and $\emm \not = \tau$,
   then $M |= \rho \land \rho => \psi$.
\begin{proof}
It follows by induction on the transitions leading to $\network$ in $M \action{\emm}
\network$ and the definition of $\network  |= \psi$.

We show the case for $\emm = \asynchServiceInputF{a}{\VEC{k}}$; all other cases
follow similarly.

If $M \action{\asynchServiceInputF{a}{\VEC{k}}} \network$ then $M
\equiv \pr{A}{\repInitIn{a}{k} \pfx P \pp P'}{\sigma} \pp M'$ and
$\pr{A}{P \pp P'}{\sigma'} \pp M'$. Assume w.l.o.g. $\rho =
\actionF{\asynchServiceInputF{a}{\VEC{k}}} \psi$. We have to show that
$M |= \rho$ and $\rho => \psi$.

From the definition of $|=$, we have that:
\begin{align}
   \pr{A}{\repInitIn{a}{\VEC{k}} \pfx P \pp P'}{\sigma} \pp M' |=
   \actionF{\asynchServiceInputF{a}{\VEC{k}}} \psi \text{ iff } M \equiv
   \pr{A}{\repInitIn{a}{\VEC{k}} \pfx P}{\sigma} \pp M''  \notag\\
   \land \exists
   Q',\sigma'; \pr{A}{Q'}{\sigma'} \land \pr{A}{Q'}{\sigma'} \pp M''
   |= \psi 
\end{align}

After the application of structural congruence rules, then $\pr{A}{\repInitIn{a}{\VEC{k}}
  \pfx P \pp P'}{\sigma} \equiv \pr{A}{\repInitIn{a}{\VEC{k}}
  \pfx P }{\sigma} \pp \pr{A}{P'}{\sigma}$. Finally, we have that $M'' = M \pp
\pr{A}{P'}{\sigma}$ and $\pr{A}{\repInitIn{a}{\VEC{k}}
  \pfx P }{\sigma} \pp M'' |=
\actionF{\asynchServiceOutF{a}{\VEC{k}}} \psi$ and
$\actionF{\asynchServiceOutF{a}{\VEC{k}}} \psi => \psi$, which is what we
had to show.
\end{proof}
\end{lemma}



\begin{definition}[Input-Output Correspondence] \label{Logic4Struct:def:io-correspondence}
  We say that a network \network~ is input-output correspondent if
  it contains no dangling inputs. That is for each $x \in
  channels(\network)$, then $\bar{x} \in channels(\network)$.
\end{definition}

% \begin{proposition}[Logical Characterisation of $\prune$]
%   Let $M,\network$ input-output correspondent networks, then $M
%   \prune \network$ if and only if $M |= \psi$ implies $\network |=
%   \psi$. 
%   \begin{proof}
%     In \cite{milner1993modal}, the logical characterisation of early
%     bisimulation is proved for the logic constructed using equations,
%     inequations, disjunction and $\actionF{\ell}$ formula. The
%     adaptation of the present setting is easy: 
\begin{proposition}[Preservation of satisfiability in pruning] \label{Logic4Struct:prop:preservation:pruning}
  Let $M,\network$ input-output correspondent networks, then  $M |= \psi$ and $M \prune \network$ then $\network |= \psi$.
  \begin{proof}
    From the definition of pruning, we have that $M \prune \network$
    iff $ \tproves{\typeEnv}{M}{\Delta}$, $\tproves{\typeEnv,
      \typeEnv'}{\network}{\Delta}$, $\network \equiv \network_0 \pp
    !\network' $ where $\tproves{\typeEnv}{\network_0}{\Delta}$,
    $\tproves{\typeEnv}{\network'}{\Delta}$ and $M \mergeable N_0$. We need to
    prove that given $\network \equiv \network_0 \pp
    !\network' $ then $\network |= \psi$.

    From Definition \ref{Logic4Struct:def:io-correspondence} and the definition of
    pruning, we know $M \merge N_0$ and $channels(M) =
    channels(\network)$, so we are not filtering any formulae related
    to inputs that could be lost in the pruning. 

    From the definition of $|=$, we have that $\network |= \psi ~|~
    \rho $ iff $\network \equiv \network'' \pp
    \network'''$. Substituting $\network_0$ for $\network''$ and
    $!\network$ for $\network'''$, then $\network |= \psi $ holds as a
    logical consequence of $\network |= \psi ~|~ \rho$.
    \end{proof}
\end{proposition}

% \CommentHugo{Can we prove the converse?   Two networks satisfy the
%   same formulae if they are typed strong
%   bisimilar (prunable):   Let $M,\network$
%   input-output correspondent networks, then $M |= \psi$ and 
%   $\network |=   \psi$  imply $M  \prune \network$.  -- Hint: use the
%   same proof technique in \cite[Theorem 2]{milner1993modal}, where a the logical characterisation of early
%     bisimulation is proved for the logic constructed using equations,
%     inequations, disjunction and $\actionF{\ell}$ formula. The
%     adaptation of the present setting seems doable.
% }

\subsection{Translation from \texorpdfstring{\GL}{GL}  to
  \texorpdfstring{\LL}{LL} }


In the same way that we define that the end point projection relates
operational views of choreographies and end-points, we need to have a
projection between the declarative visions of choreographies and their
corresponding visions in end points. We do so by providing a mapping
between the Global Logic and the Local Logic, as expressed below:


\begin{proposition}[\GL Normal form]\label{Logic4Struct:prop:nf}
  For all \GL formulae $\phi$, there exists $\phi'$ such that $\phi
  \logicEquiv \exists \vec{t}\pfx \exists
  \vec{B}\pfx \phi'$, where $\phi'$ is $\exists$ free.
%  For all choreography $\chor$, there exists $\chor'$ such that $\chor \equiv (\nu
%  \vec{s})\pfx \exists \vec{t}\pfx \exists \vec{B}\pfx \chor'$, where $\chor'$
%  is $\nu$ and $\exists$ free.
%
%  For all \GL formulas $\Phi$, there exists $\Phi'$ such that $\Phi
%  \logicEquiv (\nu \vec{s})\pfx \exists \vec{t}\pfx \exists
%  \vec{B}\pfx \Phi'$, where $\Phi'$ is $\nu$ and $\exists$ free.
\end{proposition}
\begin{proof}[Sketch]
  It follows from structural induction over $\phi$ 
\end{proof}

\begin{definition}[Participants in a formula]
Let be $\phi$ a generic \GL
formula. The set  $\parts(\phi)$ denotes the  set of participants in
$\phi$, inductively defined as follows:

\begin{align*}
\parts(\ell) & = \{A,B\} && \text{ If } \ell = \initF A B {a(k)} \\
& = \{A,B\} && \text{ If } \ell = \comF ABk \\
& = \{A,B\} && \text{ If } \ell = \branchF ABkl \\\\
%      & \{A,B\} & \phi = e_1@A = e_2@B \\
\parts(\phi)      &=  \parts(\ell) \cup \parts(\phi') & &\text{ If }\phi = \langle\ell\rangle \phi' \\
\parts(\phi)      &=  \parts(\phi') \cup \parts(\chi) &&\text{ If } \phi = \phi' \land \chi \\
\parts(\phi)      &=  \parts(\phi') \cup \parts(\chi) &&\text{ If } \phi = \phi' \mid \chi \\
\parts(\phi)      &=  \parts(\phi') &&\text{ If } \phi = \neg \phi'\\
\parts(\phi)      &=  \parts(\phi') &&\text{ If } \phi = \exists \var\pfx \phi' \\
\parts(\phi)      &=  \parts(\phi') &&\text{ If } \phi = \may \phi'  \\
\parts(\phi)      &=  \emptyset && \text{ otherwise}
\end{align*}
\end{definition}


\begin{definition}[Logical
  Projection] \label{Logic4Struct:def::logicalprojection}

We define a translation operator $[| \cdot |]$ from the formulae of \GL to
the formulae of \LL. To this end we will generate a set of formulae for every
process involved into the global formula. Additionally, let 
$\phi =  \exists \vec{t}. \exists
\vec{B}. \phi$
%$\psi = (\nu \vec{s}). \exists \vec{t}. \exists
%\vec{B}. \Phi$
 be its normal form in virtue of
Proposition~\ref{Logic4Struct:prop:nf}, then $[| \cdot |]$ is defined as follows.
\begin{gather*}
  [| \exists \vec{t}\pfx \exists \vec{B}\pfx \Phi|] =
   \exists \vec{t}\pfx \exists \vec{B}\pfx
  (\prod_{A \in \parts(\Phi)} [|\Phi|]_A) \\[1ex]
   \begin{align*}
&     [|\true|]_A  = \true \\
&     [|\actionF{\ell} \pfx \Phi|]_A  =     [|l|]_A \pfx [|\Phi|]_A \quad \text{if } A \in \parts(l) \\
&     [| \actionF{\ell} \pfx \Phi|]_A  =     [|\Phi|]_A \qquad\quad\ \, \text{if } A \notin \parts(l) \\
&     [|<<>>\Phi|]_A  = <<>> [|\Phi|]_A \vee [|\Phi|]_A\\
&     [|\Phi \wedge \Psi|]_A  =[|\Phi|]_A \wedge [|\Psi|]_A \\
&     [|\Phi \mid \Psi|]_A  = [|\Phi|]_A \mid [|\Psi|]_A \\
&     [|\neg \Phi|]_A  = \neg [|\Phi|]_A \\
&     [|\endF|]_A  = \endF      \\
% &     [| e_1 = e_2 |]_A  = \ (e_1 = e_2)@A\\[0.5cm]
&     [| e_1@A = e_2@B |]_A  = \ (e_1 = e_2)@A\\\\
     \text{Where } [|l|]_A \text{ is defined as:}\\[0.2cm]
&     [| \initF{A}{B}{a(k)} |]_A  = A[ \asynchServiceOutF{a}{k} ] \\
&     [| \initF{A}{B}{a(k)} |]_B  = B[ \asynchServiceInputF{a}{k} ] \\     
&     [| \comF{A}{B}{k(x)} |]_A  = A[ \asynchOutputF{k}{x}]      \\
&     [|\comF{A}{B}{k(x)} |]_B  = B[ \asynchInputF{k}{x}] \\ 
&     [| \branchF{A}{B}{k}{\op} |]_A  = A[ \asynchBranch{k}{op} ]          \\
&     [| \branchF{A}{B}{k}{\op} |]_B  = B[ \asynchSelection{k}{op} ]
   \end{align*}
 \end{gather*}

\end{definition}

The logical projection takes a formula $\phi$ in expressed \GL and generates the
corresponding \LL formula for each of the participants
contained in $\phi$. The mapping is standard, therefore we focus our
description in the case where $\phi$ corresponds to the action formula
$\actionF{\ell} \phi'$. The projection is defined for each
participant, so there are corresponding parts for sender and receiver
end-points in an action formula. For example, in the case $\phi =
\actionF{\initF{A}{B}{a(k)}} \pfx \phi'$, we have that $\phi$ projects
to the parallel composition of formulae for participants $A$ and $B$,
which are defined as $A[ \asynchServiceOutF{a}{k} ] \pfx [ \phi']_A$
and $B[ \asynchServiceInputF{a}{k} ] \pfx [ \phi' ]_B$
respectively. Analogous are the cases for in-session communication and
label selection.

We finish this section  by providing proof of the soundness of the
mapping between logics. Theorem
\ref{Logic4Struct:theorem:lp:soundness} states the correspondence
between end-point projections and logical projections between formulae.



\begin{lemma}[ End-point projections preserve $\equiv_{rec}$]\label{Logic4Struct:lemma:epp-equiv}

If $\chor \equiv_{rec} \chor'$ then $\epp{\CAL{A}}{\sigma}
\equiv_{rec} \epp{\CAL{A'}}{\sigma}$ where $\CAL{A},\CAL{A'}$ are
consistent annotated interactions of $\chor, \chor'$.
\begin{proof}
  By induction on the structure of  $\chor$, and Definition
%   \ref{Logic4Struct:def:cons-annotated-ints}, 
\ref{Logic4Struct:def:epp}.
\end{proof}
\end{lemma}




% \begin{lemma}\label{Logic4Struct:lemma:well-defined-TP}
%   If $\chor$ is a choreography and $\CAL{A}$ is the consistent
%   annotation of $\chor$, then $TP(\CAL{A},t_i)$ is well defined for
%   $\CAL{A}$ and $\chor \action{} \chor'$, then $TP(\CAL{A'},t_i)$ is
%   well defined, for all $t_i \in \chor$.
% \end{lemma}


\begin{theorem}[Preservation of satisfaction over logical translation] \label{Logic4Struct:theorem:lp:soundness}
Let $\chor$ a choreography and $\CAL{A}$ a consistent annotation of
$\chor$. We can conclude that $\epp{\CAL{A}}{\sigma} |= [| \phi |]$
if $\chor |=_\sigma \phi$.
\end{theorem}

\begin{proof}
  Follows by structural induction over $\chor |=_\sigma \phi $,
  definitions \ref{Logic4Struct:def:epp} and \ref{Logic4Struct:def::logicalprojection}.

Let $parts(\chor)$ a function returning the set of participants
involved in a choreography. We have the following cases:

\begin{description}


\item[Case $\chor |=_\sigma \endF$:]

  If $\chor |=_\sigma \endT$ then $\chor \equiv \INACT$, then $\CAL{A}
  = \INACT$.
  Using Definition \ref{Logic4Struct:def:epp}, $\epp{\INACT}{\sigma} = \prod_{A\ \in parts(\INACT)}
\pr{A}{\prod _{[t]} \bigcup_{t' \in [t]} TP(\INACT,
  t')}{\sigma @A} = \INACTNW$.

  On the logical side, we have that $[| \endT |] = \endT$ from
  definition \ref{Logic4Struct:def::logicalprojection}. Then $\INACTNW |= \endT$
  holds from the definition of $|=$.\\

\item[Case $\chor |=_\sigma \lnot \phi$:]

  If $\chor |=_\sigma \phi$ then $\chor \not |=_\sigma \phi$. We have
  to show $\epp{\CAL{A}}{\sigma} \not |= [| \phi |]$.

  We proceed by contradiction:  Take $\epp{\CAL{A}}{\sigma} |= [|
  \phi|]$, then $\chor |=_\sigma \phi$, which is a contradiction to
  $\chor |=_\sigma \lnot \phi$.

\item[Case $\chor |=_\sigma \phi \land \chi$:] 

  From  the definition of $|=_\sigma$, $\chor
  |=_\sigma  \phi \land \chi$ iff $\chor |=_\sigma \phi \land 
  \chor |=_\sigma \chi$. 
  
  From the induction hypothesis we have
  $\epp{\CAL{A}}{\sigma} |= [| \phi |] $ and $\epp{\CAL{A}}{\sigma} |=
  [| \chi |] $.

  Then $\epp{\CAL{A}}{\sigma} |= [| \phi |] \land [|
  \chi |]$ from the definition of $|=$. 

\item[Case $\chor |= \phi \pp \chi$:]

  From  the definition of $|=_\sigma$, $\chor |=_\sigma \phi \pp \chi$ iff $\chor \equiv \chor_1 \pp \chor_2$
  such that $\chor_1 |=_\sigma \phi $ and $\chor_2 |=_\sigma
  \chi$.

  From the induction hypothesis we know that,
  $\epp{\CAL{A}_1}{\sigma} |= [| \phi |]$ and $\epp{\CAL{A}_2}{\sigma}
  |= [| \chi |]$. 

  From Lemma
  \ref{Logic4Struct:lemma:StructuralSatisfiability} we know that $\chor_1 \pp
  \chor_2 |=_\sigma \phi$.

 From the definition of the end-point projection and 
  Lemma \ref{Logic4Struct:lemma:epp-equiv}, we know that there exists
  $\network$ such that $\network \equiv_{rec} \epp{\CAL{A}_1}{\sigma}
  \pp \epp{\CAL{A}_2}{\sigma}$. 

  Then $\network |= [| \phi |] \pp [|
  \chi |]$ follows from the definition of $|=$.


\item[Case $C |=_\sigma \actionF{\ell} \phi$:]

 From the definition of $|=_\sigma$, $\chor |=_\sigma \actionF{\ell}
  \phi$ iff $<. \sigma, \chor.> \action{\ell} <. \sigma', \chor' .>
  \land \chor' |=_{\sigma'} \phi$.  We have to show that
  $\epp{\CAL{A}}{\sigma} |= [| \actionF{\ell} \phi |]$.

From induction hypothesis, we have that:

\begin{equation} \label{Logic4Struct:eq:action-IH}
  \epp{\CAL{A}'}{\sigma'} |= [| \phi |] \land
  \epp{\CAL{A}'}{\sigma'} \prune \network
\end{equation}

From (completeness with action labels) in Theorem \ref{Logic4Struct:theorem:epp} we
have that, given $<. \sigma, \chor.>
\action{\ell} <. \sigma', \chor' .>$ then:
  \begin{equation} \label{Logic4Struct:eq:action1}
    \epp{\CAL{A}}{\sigma} \action{\emm} \network \land
    \epp{\CAL{A'}}{\sigma'} \prune \network.
  \end{equation}
Now, we only have to show that $\network |= [| \phi |]$, which holds
after using  Proposition \ref{Logic4Struct:prop:preservation:pruning} in
$\epp{\CAL{A}'}{\sigma'} \prune \network $ from Equation \ref{Logic4Struct:eq:action-IH}.


%   from the definition of $|=_\sigma$, $\chor |=_\sigma \actionF{\ell}
%   \phi$ iff $<. \sigma, \chor.> \action{\ell} <. \sigma', \chor' .>
%   \land \chor' |=_{\sigma'} \phi$.  From (completeness with action
%   labels) in Theorem \ref{Logic4Struct:theorem:epp} we have that
%   \begin{equation} \label{Logic4Struct:eq:action1}
%     \epp{\CAL{A}}{\sigma} \action{\emm} \network \land
%     \epp{\CAL{A'}}{\sigma'} \prune \network.
%   \end{equation}

%   From the induction hypothesis, we have:
%   \begin{equation} \label{Logic4Struct:eq:action2}
% %     \epp{\CAL{A'}}{\sigma'} |= [| \phi |],
%     \network |= [| \phi |],
%     \end{equation}
% %     with $\CAL{A'}$ the consistent annotation of $\chor'$.
%     Then from Lemma
%     \ref{Logic4Struct:prop:preservation:pruning}, $ \epp{\CAL{A'}}{\sigma'} |= [|
%     \phi |]$. Finally, from Equations \ref{Logic4Struct:eq:action1} and \ref{Logic4Struct:eq:action2} we get:

%     \begin{equation}
%        \epp{\CAL{A}}{\sigma} |= [| \actionF{\ell} \phi |]
%       \end{equation}

%       Which is what we had to show.


% \item[Case $C |=_\sigma \actionF{\ell} \phi$:]

%   from the definition of $|=_\sigma$, $\chor |=_\sigma \actionF{\ell}
%   \phi$ iff $<. \sigma, \chor.> \action{\ell} <. \sigma', \chor' .>
%   \land \chor' |=_{\sigma'} \phi$. By induction hypothesis we have
%   that $\epp{\CAL{A'}}{\sigma'} |= [| \phi |]$, where $\CAL{A'}$ is the
%   consistent annotation of $\chor'$. We have to show
%   $\epp{\CAL{A}}{\sigma} |= \actionF{\ell} [| \phi |]$.

%   We proceed by case analysis of $\ell$:

%   \begin{itemize}

%     \item{Case $\ell = \initF{A}{B}{a(k)}$:} then $\chor =
%       \init{A}{B}{a}{k} \pfx \chor'$, and the consistent annotation of
%       $\chor$ is $\CAL{A} = \init{A^{t_1}}{B^{t_2}}{a}{k} \pfx
%       \CAL{A'}$. Then''
      
%       \begin{equation}\label{Logic4Struct:epp:init-proj}
%       \epp{\CAL{A}}{\sigma} = \pr{A}{\initOut{a}{k} \pfx
%         TP(\CAL{A'},t_1)}{\sigma @ A} \pp \pr{B}{\repInitIn{a}{k} \pfx
%         TP(\CAL{A'},t_2)}{\sigma @ B}
%       \end{equation}

%       and by Lemma \ref{Logic4Struct:lemma:well-defined-TP} we have that
%       $TP(\CAL{A'},t_1), TP(\CAL{A'},t_2)$ are well defined.

%       On the logic side:
      
%       \begin{equation}
%         [| \actionF{\initF{A}{B}{a(k)}} \phi |] =
%         (\pr{A}{\asynchServiceOutF{a}{k}} \pfx [| \phi |] \pp \pr{B}{\asynchServiceInputF{a}{k}} \pfx [| \phi |])
%         \end{equation}

%         From the assertion semantics for \LL:

%         \[
%         \pr{A}{\asynchServiceOutF{a}{k}} \pfx [| \phi |] \pp
%         \pr{B}{\asynchServiceInputF{a}{k}} \pfx [| \phi |] \iff
%         \network   
%         \]

%     \end{itemize}

%   By definition of
%   $|=_\sigma$, $C |=_\sigma \actionF{\ell} \phi \iff C \action{\ell}
%   C' \land C' |=_\sigma \phi$, and by induction hypothesis $N' =
%   EPP(C',\sigma')$ such that $N' |= \phi$. We have to show that $EPP(C,\sigma) |= [|
%   \actionF{\ell} \phi|]$.
  
%   We proceed by case analysis of $\ell$, here we present the case for
%   $\ell = \initF{A}{B}{a(k)}$, and the others follow similarly.

%   If $\ell = \initF{A}{B}{a(k)}$, then we know that $C \action{\ell}
%   C'$ and $C'' \equiv C'$. Assuming 
%   $[|  \exists \var\pfx \exists B \actionF{\ell} \phi |] = 
%   \exists \var \pfx \exists B \prod_{A \in \parts(\phi)} [|
%   \actionF{\ell} \phi|]$,
%   then by Definition \ref{Logic4Struct:def::logicalprojection}, we have that:
% \[
%       \actionF{\ell}  \phi = [| l |]_A \pfx [| \phi |]_A \text{ if
% } A \in \parts(l) \text{ or } [| \phi |]_A \text{ otherwise}
%   \]
%   \begin{itemize}
%   	\item (Case  $A \not \in \parts(\ell)$): follows naturally, as $
%   \exists \var \pfx \exists B \prod_{A \in \parts(\phi)} [| \phi |]_A
%   $ can be deduced from the induction hypothesis $N' |= \phi $.
  
%  	\item (Case $A \in \parts(\ell)$): then \[ [| \actionF{\ell} \phi |]_A =
%    \locatedActionF{A}{\asynchServiceOutF{a}{k}}\pfx [| \phi |]_A, \] Similarly for
%    $B$: \[ [| \actionF{\ell} \phi |]_B =
%    \locatedActionF{B}{\asynchServiceInputF{a}{k}} \pfx [| \phi |]_B,\]
%    then: \[ [|  \exists \var \pfx \exists B  \actionF{\ell}
%    \phi |] = \exists \var \pfx \exists B \big(
%    \locatedActionF{A}{\asynchServiceOutF{a}{k}}\pfx [| \phi |]_A  \pp
%    \locatedActionF{B}{\asynchServiceInputF{a}{k}} \pfx [| \phi |]_B
%    \big) .\]

%    Also,  
%    	\begin{align*}
%    		N = & EPP ( C, \sigma) && =  \prod_{A \in parts(C)}
%   \pr{A}{\prod _{[\tau]} \bigcup_{\tau' \in [\tau]} TP(C,
%     \tau')}{\sigma @A} \\ 
%     & &&  =\pr{A}{\initOut{a}{k} \pfx TP (C',A)}{\sigma} \pp
%   \pr{B}{\repInitIn{a}{k} \pfx TP(C',B)}{\sigma}
%   	\end{align*} 
% 	We denote the
%   thread projections $TP (C',A), TP (C',B)$  as $N_A', N_B'$ respectively.

%   By definition of the assertions in $|=$, we have that:
%   \[
%   N |=  \exists \var \pfx \exists B \big(
%    \locatedActionF{A}{\asynchServiceOutF{a}{k}}\pfx [| \phi |]_A  \pp
%    \locatedActionF{B}{\asynchServiceInputF{a}{k}} \pfx [| \phi |]_B
%    \big)
%   \]

%   Which is true for $N \equiv N' \pp M$ and the inductive hypotheses
%   $\pr{A}{N_A'}{\sigma'} |= [| \phi_A|]$ and   $\pr{B}{N_B'}{\sigma'} |= [|
%   \phi_B |]$, so we are done.
% 	\end{itemize}
% \item{$C |=_\sigma \phi ~|~ \chi$:} 

%   By definition of $|=_\sigma$, $C
%   \equiv C' \pp C''$, $C' |=_\sigma \phi$ and $C'' |=_\sigma \chi$. By
%   induction hypothesis $EPP( C', \sigma) = N$,  $EPP( 
%   C'', \sigma) = N'$,  $N |= \phi$ and $N' |= \chi$, so we can get $N
%   \pp N' |= \phi \pp \chi $ directly by the definition of $|=$.

\item[Case $\chor |=_\sigma \may \phi$:]

  From  the definition of $|=_\sigma$, $\chor
  |=_\sigma \may \phi$ iff $(\sigma, \chor) -->^* (\sigma', \chor')$ and
  $\chor' |=_\sigma \phi$. Assume $(\sigma, \chor) -->^* (\sigma', \chor')$
  as a finite sequence of transitions $(\sigma, \chor) -->^n
  (\sigma', \chor')$. We proceed by induction on $n$.

  \begin{description}
   \item[Case $n=1$:] then $\chor |=_\sigma \may \phi$ iff $(\sigma, \chor) -->^1
  (\sigma', \chor') \land \chor' |=_{\sigma'} \phi$, which is the same case as
  for $\chor |=_\sigma \actionF{\ell} \phi$ for any $\ell$. As $\chor
  |=_\sigma \actionF{\ell} \phi \implies \epp{\CAL{A}}{\sigma} |=
  [| \actionF{\ell} \phi |]$ then we are done.

   \item[Case $n>1$:] then $\chor |=_\sigma \may \phi$ iff $(\sigma, \chor)
  -->^{n-1} (\sigma'', \chor'') --> ^1 (\sigma',\chor') \land \chor' |=_{\sigma'}
  \phi$.

  From the induction hypothesis we get $(\sigma, \chor)
  -->^{n-1} (\sigma'', \chor'')$ and $\chor ''|=_{\sigma''} \phi$, then $\chor
  |=_\sigma \may \phi$. 

  Moreover, $\epp{\CAL{A''}}{\sigma''} =N$
  and $N |= [| \phi |]$, with $\CAL{A''}$ a consistent annotation of
  $\chor''$, then $\epp{\CAL{A}}{\sigma} |= \may \nextOp [| \phi |]$.

  Finally, using subsumption we know that $\may [| \phi |] \implies
  \may \nextOp  [| \phi |]$. Then $ \epp{\CAL{A}}{\sigma} |= \may [| \phi |]$, which is what
  we had to show.
  \end{description}
%  Also, from the induction hypothesis we get that $(C', \sigma ') = N'$ and $N'
%   |= \phi$. We have to show that $EPP(\new{k} C, \sigma) |= [| \may \phi|]$.

%   From the semantics of $|=_\sigma$, we know that $C |=_\sigma \phi $ iff
%   $(\sigma, C) \action{}^* (\sigma', C')$ and $C' |=_{\sigma'}
%   \phi$.
%  Applying $[| \cdot |]$, $[| \may \phi |]_A = \may [| \phi
%   |]_A \lor [| \phi |]_A$ which holds true by the inductive
%   hypothesis.

  \item[Case $\chor |=_\sigma e_1@A = e_2@B$:] Then $\sigma(e_1@A) \Downarrow v \land
    \sigma(e_2@B) \Downarrow v$. 

    From $[| \cdot |]$, we know $[| e_1@A = e_2@B |] = (e_1 = e_2)@A
    \pp (e_1 = e_2)@B$.

    We know that $\epp{\CAL{A}}{\sigma} |= (e_1 = e_2)@A
    \pp (e_1 = e_2)@B$ iff $\network \equiv \pr{A}{P}{\sigma} \pp
    \pr{B}{Q}{\sigma'} \pp M$ and $ \sigma(e_1) = \sigma(e_2) \land
    \sigma'(e_1) = \sigma'(e_2)$, which follows directly from 
    Definition \ref{Logic4Struct:def:epp}.

\end{description}

\end{proof} 



%%% Local Variables: 
%%% mode: latex
%%% TeX-master: "../Thesis"
%%% End: 
