\subsection{\texorpdfstring{\LL}{LL} : Proof System}


Similarly to \GL, \LL is equipped with an inference system to deduce
whether an end point $\network$ respects a given formula $\psi$. We
write $ \network |- \psi $ for the provability judgement where
$\network$ is a network and $\psi$ is a formula in \LL.

\begin{definition}[Exhibition - End Points]
  We say that a network $\network$ \emph{exhibits} a formula $\psi$,
  written $\network |- \psi$, iff the assertion $\network |- \psi$ has
  a proof in the proof system given in
  Figure~\ref{Logic4Struct:table:Local:proofSys}.
\end{definition}


\begin{myfigure}{t}
  \begin{align*}
    &
    \myruleg{L.P_{init-out}}{\typerule{A[P]_\sigma}{}{\psi}}
    {\typerule{A[ \initOut{a}{k} \pfx P ]_\sigma }
      {}{\locatedActionF{A}{\asynchServiceOutF{a}{k} } \pfx \psi}}
   ~
    \myruleg{L.P_{init-in}}{\typerule{A[P]_\sigma}{}{\psi}}
    {\typerule{A[ \repInitIn{a}{k} \pfx P ]_\sigma }
      {}{\locatedActionF{A}{\asynchServiceInputF{a}{k} } \pfx \psi}} 
    ~
    \myruleg{L.P_{may1}}{ \typerule{\network}{}{\psi}}
    {\typerule{ \network }{}{\may \psi}}
  ~
    \myruleg{L.P_{may2}}{ \typerule{\network}{}{\psi \lor \nextOp \may \psi}}
    {\typerule{ \network }{}{\may \psi}}
   \\[1mm]
    &
    \myruleg{L.P_{bra}}{\typerule{\forall i\in I.\ A[P_i]_\sigma}{}{\psi_i}}
    {\typerule{  A [\branching{k}{\sum_i \mathsf{l_i} \pfx
          P_i}] }{}{\bigwedge_{i\in I}
        \locatedActionF{A}{\asynchBranch{k}{l_i}} \pfx \psi_i}}
    ~
    \myruleg{L.P_{sel}}{\typerule{ A[P]_\sigma }{}{\psi}}
    {\typerule{  A [\selection{k}{ \mathsf{l} } P] }{}{
        \locatedActionF{A}{\asynchSelection{k}{l}} \pfx \psi}}
    ~
        \myruleg{L.P_{end}}{ }
    {\typerule{ \INACT }{}{\endF}}
  \\[1mm]
    &
    \myruleg{L.P_{send}}{
      \typerule{A[P]_\sigma}{}{\psi}
    }
    {\typerule{ 
        A[ \outputP{k}{x} \pfx P ]_\sigma }
      {}{ \locatedActionF{A}{\asynchOutputF{k}{x}} \pfx \psi}}
    ~
    \myruleg{L.P_{rcv}}{
      \typerule{A[P]_\sigma}{}{\psi}
    }
    {\typerule{ 
        A[ \inputP{k}{x} \pfx P ]_\sigma }
      {}{ \locatedActionF{A}{\asynchInputF{k}{x}} \pfx \psi}}
    ~
%     \myruleg{P_{or1}}{\typerule{\network}{}{\psi_1} }
%     {\typerule{\network}{}{\psi_1 \lor \psi_2}}
%     ~
    \myruleg{L.P_{and}}{\typerule{\network}{}{\psi_1} \quad \typerule{\network}{}{\psi_2}}
    {\typerule{\network}{}{\psi_1 \land \psi_2}}
  \\[1mm]
    &
    \myruleg{L.P_{exp}}{\network \equiv A[P]_\sigma ~~ \sigma(e_1) \Downarrow v ~~ \sigma(e_2) \Downarrow v}
    {\typerule{\network}{}{ (e_1 = e_2)@A }}
    ~
    \myruleg{L.P_{struct}}{\typerule{\network}{}{\psi} \quad \network\equiv \network '}
    {\typerule{\network '}{}{\psi}}
%     ~
%     \myruleg{P_{or2}}{\typerule{\network}{}{\psi_2} }
%     {\typerule{\network}{}{\psi_1 \lor \psi_2}}
    ~
    \myruleg{L.P_{neg}}{ \network \not |- \psi }
    {\typerule{\network}{}{ \lnot \psi}}
  \\[1mm]
    &
    \myruleg{L.P_{ifT}}{\sigma(e) \Downarrow \true \quad
      \typerule{A[P_1]_\sigma}{}{\psi}}
    {\typerule{ A[\itn {e}{P_1}{P_2}]_\sigma }{}{\psi}}
    ~
    \myruleg{L.P_{ifF}}{\sigma(e) \Downarrow \false \quad
      \typerule{A[P_2]_\sigma}{}{\psi}}
    {\typerule{ A[\itn {e}{P_1}{P_2}]_\sigma }{}{\psi}}
    ~
    \myruleg{L.P_{par}}{\typerule{\network_1}{}{\psi_1} \quad
      \typerule{\network_2}{}{\psi_2}}
    {\typerule{\network_1 \pp \network_2}{}{ \psi_1 \pp \psi_2}}
   \\[1mm]
    &
    \myruleg{L.P_{\exists}}{\exists  w\in fn(\network) \cup \{k\}.\
      \typerule{\network}{}{\psi[w/\var]} ~\text{ $k$ fresh}}
    {\typerule{\network}{}{\exists \var \pfx \psi}}
    ~
    \myruleg{L.P_{oplus1}}{\typerule{A[P_1]_\sigma}{}{\psi} }
    {\typerule{ A[ P_1 \oplus P_2 ]_\sigma}{}{\psi}}
    ~
    \myruleg{L.P_{oplus2}}{\typerule{A[P_2]_\sigma}{}{\psi} }
    {\typerule{ A[P_1 \oplus P_2 ]_\sigma}{}{\psi}}
%     ~
%     \myruleg{P_{rest}}{\typerule{ \network}{}{\psi} ~~ x \text{ is fresh }}
%     {\typerule{ \new{x} \pfx \network }{}{\new{x} \pfx \psi}}
\\[1mm]
&
%     \myruleg{L.P_{synch}}{\network \equiv M \pp M' ~~ \typerule{M}{}{
%         \locatedActionF{A}{\asynchOutputF{s}{k}}} \pfx \psi ~~ \typerule{M'}{}{
%         \locatedActionF{B}{\asynchInputF{s}{k}}} \pfx \omega}  
%     {\typerule{ \network }{}{\psi [\tau] \omega}}
%     ~
% \\[1mm]
% &
%     \myruleg{L.P_{synch-k}}{\network \equiv M \pp M' ~~ \typerule{M}{}{
%         \locatedActionF{A}{\asynchServiceOutF{s}{k}}} \pfx \psi ~~ \typerule{M'}{}{
%         \locatedActionF{B}{\asynchServiceInputF{s}{k}}} \pfx \omega}  
%     {\typerule{ \network }{}{\psi [\tau_k] \omega}}
%     ~
   % &
    % \myruleg{P_{var}}{ \mathbf{-} } {\typerule{ X }{}{\phi}}
    % \qquad
    % \myruleg{P_{rec}}{\typerule{C\MSUBS{\mu X.C}{X}}{}{\phi} }
    % {\typerule{ \mu X\pfx C }{}{ \phi }}
 \end{align*}
 \caption{Proof system for the End Point Calculus.}
 \label{Logic4Struct:table:Local:proofSys}
\end{myfigure}
Let us now describe some of the inference rules of the proof system.
The rule $\mathsf{L.P_{end}}$ relates the inaction terms with the
termination formula. The rules $\mathsf{L.P_{and}}$ and
$\mathsf{L.P_{neg}}$ denote rules for conjunction and negation in
classical logic, respectively. The rule for parallel composition is
represented in $\mathsf{L.P_{par}}$; it does not indicate the behaviour
of a given end-point, but relates interacting end-points with their
corresponding formulae: $\mathsf{L.P_{par}}$ juxtaposes the behaviour of two
processes and combines their respective formulae by the use of a
separation operator. The correspondence between a network and a may
formula is given by a formula relating each of the possible labels it
can contain.: 
$\mathsf{L.P_{bra}}$ can be explained as follows: suppose we are given a
process $\network = \pr{A}{\branching{s}{\Sigma_{i} l_i} \pfx P_i}{\sigma}$, a set of branch labels
$\{l_i \, | \, i \in I\} $ (determined by typing) and we are given a
proof that each $\pr{A}{\sigma}{P_i}$ satisfies $\psi_i$, then we certainly have a
proof saying that every derivation of $\network$ should satisfy a guard $l_i$
followed by a formula $\psi_i$. 
$\mathsf{L.P_{init-out}}$ and $\mathsf{L.P_{init-in}}$
describe the session initiation formulae, $\mathsf{L.P_{bra}}$ and
$\mathsf{L.P_{sel}}$ describe label branching and selection, and
$\mathsf{L.P_{rcv}}$ and $\mathsf{L.P_{send}}$ data communication. 
 Similarly,
the rule
$\mathsf{P_\exists}$ says that in order to satisfy an $\exists t\pfx
\psi$, it is sufficient to find a value $w$ for $t$ in the free names
used by the network $\network$ or in the free names used by the
formula $\psi$. Rule $\mathsf{P_{exp}}$ denotes
evaluation of local  expressions.
As can be noted by $\mathsf{L.P_{may1}}$ and
$\mathsf{L.P_{may2}}$, the proof system encodes the eventual operator as
the unfolding recursion of $<<>> \psi$. 
$\mathsf{P_{struct}}$ proves
that processes that are structurally congruent bear correspondence
with the same logical formula. Conditional and internal choice rules
$\mathsf{L.P_{ifT}}$, $\mathsf{L.P_{ifF}}$, $\mathsf{L.P_{oplus1}}$
and $\mathsf{L.P_{oplus2}}$ are also standard. 


% \CommentHugo{Fixme: Do something with the rules in
%   $\mathsf{L.P_{synch}, L.P_{synch-k}}$, originally we introduced them
%   to reason about synchronization within networks, but now they are
%   not included in \LL, although they seem a bit important for this
%   setting. Marco, what do you think?} 



\begin{theorem}[Soundness]\label{Logic4Struct:thm:LL:soundness}
  For any 
  recursion-free network \network, and every formula $\psi$, if $
  \network |- \psi$  then $ \network |= \psi$.
\end{theorem}
\begin{proof}
  It follows by induction on the derivation of $|-$.
 \begin{description}

  \item[Case $\mathsf{P_{end}}$:] trivial.

  \item[Case $\mathsf{L.P_{init-out}}$:] We have that $\network =
    \pr{A}{\initOut{a}{k} \pfx P}{\sigma} $ and $\network |-
    \locatedActionF{A}{ \asynchServiceOutF{a}{k}} \pfx \psi$. By
    $\mathsf{L.P_{init-out}}$, we have that $\pr{A}{P}{\sigma} |-
    \psi$. From induction hypothesis we get that $\pr{A}{P}{\sigma} |=
    \psi$. We have to show that  $\pr{A}{\initOut{a}{k} \pfx
      P}{\sigma}   |= \locatedActionF{A}{ \asynchServiceOutF{a}{k}}
    \pfx \psi$. From the assertion semantics, we have that $\network' |=
    \locatedActionF{A}{\asynchServiceOutF{a}{k}} \pfx \psi$ iff
    $\network' \equiv \pr{A}{Q}{\sigma} \pp M \land \exists
    R,\sigma'. \pr{A}{Q}{\sigma} \action{\asynchServiceOutF{a}{k}}
    \pr{A}{R}{\sigma'} \land \pr{A}{R}{\sigma'} \pp M |= \psi$, which
    holds immediately from the selection of $Q = \initOut{a}{k} \pfx P$
    and the induction hypothesis.

    \item[Cases $\mathsf{L.P_{init-in}}, \mathsf{L.P_{bra}},
      \mathsf{L.P_{sel}}, \mathsf{L.P_{send}}, \mathsf{L.P_{rcv}}$:]
      Analogous to case $\mathsf{L.P_{init-out}}$.

  \item[Case $\mathsf{L.P_{par}}$:] We have that $\network  =
    \network_1 \pp \network_2$ and $\network |- \psi_1
    ~\pp~ \psi_2$ and by $\mathsf{L.P_{par}}$ we know that $\network_1
    |- \psi_1$ and $\network_2 |- \psi_2$. From the induction
    hypothesis we know that $\network_1 |= \psi_1$ and $\network_2 |=
    \psi_2$. We have to show $\network |= \psi_1 \pp \psi_2$, which
    follows directly from the assertion semantics for $\psi_1 \pp
    \psi_2$ and the induction hypothesis.

    \item[Cases $\mathsf{L.P_{and}}$, $\mathsf{L.P_{exp}}$:] Analogous to
      $\mathsf{P_{par}}$.


    \item[Case $\mathsf{L.P_{neg}}$:] We have that $\network |- \lnot
      \psi$, so by $\mathsf{L.P_{neg}}$ we get $\network \not |-
      \psi$. By induction hypothesis we have that $\network \not
      |= \psi$, which is necessary condition to deduce
      $\network |= \lnot \psi$.


  \item[Case $\mathsf{L.P_{\exists}}$:] We have that $\network |-
    \exists t. \psi$ and by $\mathsf{L.P_{\exists}}$ we have that
    $\exists w \in fn(\network) \cup \{ k \}$ and  $\network |-
    \psi [w/t]$. By induction hypothesis we know that $\network |=
    \psi [w/t]$. Take $w \in fn(\network)$ or a fresh name, then we know
    that $\network |= \psi[w/t]$ with appropriate $w$, and then
    $\network |= \exists t. \psi$ follows from the definition of
    the assertion semantics. 



%     \item[Case $\mathsf{P_{may}}$:] We have that $\chor |-_\sigma
%       \may \phi$ and by $\mathsf{P_{may}}$ then $\chor' |-_{\sigma'}
%       \phi$ and $(\sigma', \chor') \in $Reachable($\sigma,
%       \chor$). From the induction hypothesis we have that $\chor'
%       |=_{\sigma} \phi$, then we have to show that $\chor |=_\sigma
%       \may \phi$. From the assertion semantics we know that $C
%       |=_\sigma \may \phi \iff (\sigma, \chor') \action{} ^* (\sigma',
%       \chor')$ and $\chor' |=_{\sigma'} \phi$, which holds immediately
%       by the selection of $(\sigma', \chor') \in $Reachable($\sigma,
%       \chor$) and the induction hypothesis.


  \item[Case $\mathsf{L.P_{may1}}$:] We have that $\network |- \may
    \psi$ and by $\mathsf{L.P_{may1}}$ then $\network |- \psi$. By
    induction hypothesis we have $\network |= \psi$. We have to
    show that $\network |= \may \psi$, which follows immediately
    from $\psi => \may \psi$, so $\network |= \may \psi$.

  \item[Case $\mathsf{L.P_{may2}}$:] We have that $\network |- \may
    \psi$ and by $\mathsf{L.P_{may2}}$ then $\network |- \psi \lor \nextOp
    \may \psi$. By induction hypothesis we have that $\network |=
    \psi \lor \nextOp \may \psi$. We have to prove that $\network |= \may
    \psi$. By definition of the assertion semantics, $\network |=
    \may \psi$ iff $\network \action{ }^* M. \; M |= \psi $. We can
    express $ \network \action{ }^*  M $ as a finite sequence of
    transitions $\network \action{ }^n M$. We
    have to show that there exists $0 \leq i \leq n$ such that
    $\network \action{}^i \network' \action{}^j
    M$ and $\network' |= \may \psi$. We
    proceed by second induction on $j-i$:
    
    \begin{description}

    \item[1. ($j-i = j$):] Then $ \network \action{}^0
      \network' \action{}^n M$ and $M
      |= \psi$, which is the same case than the one for
      $\mathsf{L.P_{may1}}$, hence $N |= \may \psi$.

        \item[2.($j-i=1$):] Then $\network \action{}^1 \network' \action{}^{n-1} M$ and $M
          |= \psi$, so $\network |= \nextOp \may \psi$, then $\network |= \may  \psi$
          holds true by direct application of the induction hypothesis.

        \item[3.($ j-i = k \land 1 < k \leq n$):] Then we have 
          that $ \network \action{}^i \network' \action{}^j M$ and $M
          |= \psi$. By second inductive hypothesis we have that
          $\network' |= \may \psi$. We can decompose
          $\network \action{}^i \network'$ as:
            \begin{equation}\label{Logic4Struct:eq::may2}
              N \action{}^1 N''
              \action{}^{i-1} N'
            \end{equation}
            The combination of the second inductive hypothesis and the
            assertion semantics for equation
            \ref{Logic4Struct:eq::may2} leads to $\network |= \nextOp \may
            \may \psi$, which reduces to $\network |= \nextOp
            \may \psi$ using standard formula equivalences from
            LTL \cite{emerson1991temporal}.
      \end{description}


    \item[Case $\mathsf{L.P_{struct}}$:] It follows by direct
      application of lemma \ref{Logic4Struct:lemma:StructuralSatisfiability-local}.

    
    \item[Case $\mathsf{L.P_{ifT}, L.P_{ifF}}$:] We take only the proof for
    $\mathsf{L.P_{ifT}}$, the other works similarly.
   
    We have that $\network = \pr{A}{\itn{e}{P_1}{P_2}}{\sigma}$, by
    $\mathsf{L.P_{ifT}}$ we have that $\sigma(e) \Downarrow \true $
    and $\pr{A}{P_1}{\sigma} |- \psi$, and by induction hypothesis we have
    that $\pr{A}{P_1}{\sigma} |= \psi$. 
    We have to show that $\network |= \psi.$
    Assume a $\sigma$ s.t. $\sigma ( e )
    \Downarrow \true$ (The other case is symmetric),
    from the assertion semantics, we get that $\network ' |= (e_1 =
    e_2)@A$ iff $\network' \equiv \pr{A}{P}{\sigma}$ and $\sigma(e_1)
    = \sigma(e_2)$, which holds true from Lemma
    \ref{Logic4Struct:lemma:epp-equiv} and the induction hypothesis,
    therefore $\network |= \psi$.
%     Assume a $\sigma$ s.t. $\sigma ( e )
%     \Downarrow \true$ (The other case is symmetric), and $P_1|=
%     \psi$ referring to the property expressed in the transition
%     $(\sigma, \chor_1) \action{\ell} (\sigma, \chor_1')$. Then from
%     \Did{G-IfT} we know that $(\sigma, \chor) \action{\ell} (\sigma,
%     \chor_1')$ and $\chor |=_\sigma \phi$, which is what we had to
%     show.

    \item[Case $\mathsf{L.P_{oplus1}}$, $\mathsf{L.P_{oplus2}}$: ]
      It follows by direct application of lemma \ref{Logic4Struct:lemma:LL:implication}.
 \end{description}
\end{proof}

% \CommentHugo{Fixme: Adapt the following theorems}

% \begin{theorem}[Completeness]\label{Logic4Struct:thm:LL:completeness}
%   For any 
%   recursion-free network \network, and every formula $\psi$, if
%   $\network |= \psi$   then $\network |- \psi$.
% \end{theorem}






%%% Local Variables: 
%%% mode: latex
%%% TeX-master: "../Thesis"
%%% End: 
