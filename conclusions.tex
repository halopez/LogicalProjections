\section{Conclusion and Related Work}
\label{Logic4Struct:sec:conclusion}
% \CommentHugo{Fixme: Adapt the following conclusion to the full
%   framework, and connect the research as the foundation of model
%   checking of structured communications}

This ongoing work aims at establishing the relations between imperative and declarative views of structured
communications, and it  constitutes just the first step towards a
verification framework for communication-centred programs. Summarising,
this work argues that one can have more flexible specifications in a declarative
(logical) of communication-centred programs than in an
imperative one, and it presents ways of verifying the correspondence
of imperative views with respect to their declarative ones, in terms of
proof systems for each of the levels of abstraction here considered
(choreographies and end-points). Similarly, we establish a connection
between the methodology used for describing communication-centred
programs imperatively (the end-point projection) and a logical
projection between logics, and prove that the end-points generated
from a global specification comply with the projections of global
formulae in the local logic. Some further development of the ideas
here exposed involve the proof about the completeness of \LL along the same lines as that for \GL, and exploring the termination of the
proof checking algorithm. These results will pave the way for an approach to verifying structured communications, and one foresees further
implementation of model checking techniques where the connections
between declarative and imperative specifications can be exploited.




Our work has led to further questions concerning the appropriate set
of operators to be included in a logic for structured
communications. In this document we explored derivations of
Hennessy-Milner Logics, where the main properties of interest involved
action and may formulae both at the level of choreographies and
end-points. The may operator provides important information about the
existence of an evolution where a property is fulfilled, but sometimes
it can fall short of this  by allowing other evolutions of the system that do
not comply to the property. In \cite{Carbone2011Open-Mixed-Refi} we
started studies on stronger versions of the may modality, where one is
allowed to express that a property is fulfilled in all possible
executions in an eventual state, and their implementation as part of
the operators in \GL is foresee.  Other improvements to the logics
proposed include the use of fixed points, essential for describing
state-changing loops, and auxiliary axioms describing structural
properties of a choreography.



\paragraph{Related Work}
The connections between logics and session types have been explored in
different works. Here we comment on some of the most representative
exponents, namely \cite{CD08, caires2010session,
  Bocchi2010A-theory-of-des,gordon2009principles,Berger2008Completeness-an}.
In \cite{CD08}, a calculus combining notions of concurrent constraint
programming and name passing is proposed. The resulting calculus
treats sessions as constraint formulae representing the requirements
to be satisfied in a client-server communication, using an approach similar to that of the CC-Pi calculus. As communications are
represented as constraints, the type discipline takes the relationship
between processes and constraints into account and guarantees that 
processes and constraints are related, guaranteeing that
communications follow a structured communication as in
\cite{honda1998lpa}.

The relationship between session types and linear logics has been
explored in \cite{caires2010session}, where the authors establish a
bidirectional correspondence between the session types and (dual)
intuitionistic linear logic formulae. The correspondence is tight, and
relates the existence of a simulation between reductions in session
types and proof reductions in dual intuitionistic
linear logic, and vice versa.  In \cite{Perez2012Linear-Logical-}, the
authors make use of the linear logic interpretation of session types
to describe a theory of logical relations for session types, allowing
one to study properties like termination of well-typed interactions,
and behavioural characterisations of session-typed isomorphisms as
linear logic equivalences.

Type and effect systems have been used to study structured
communications. In \cite{gordon2003typing}, the \mipi- calculus is
extended with labelled assertions describing progress in their
communication steps. Assertions have complementary operations, and one
can ensure that the communication is safe if all specified assertions
have their correspondent begin-end operations present in the run of a
protocol. In \cite{bonelli2005correspondence}, the theory of session
types with correspondence assertions is studied, providing stronger
guarantees for session types, in the sense that correspondence
assertions allow one to keep track of the changes on the data
transmitted over sessions and the way data is propagated across
multiple parties.


Relations between types and logics can also give more information
about the nature of structured communications. In
\cite{Bocchi2010A-theory-of-des}, authors proposed the integration of
typed-based signatures with logical predicates as a method to
guarantee finer grained properties about the information in transit in
structured interactions. The proposed a methodology (\emph{Design by
  contract}), constitutes an extension of multiparty session types
\cite{DBLP:conf/popl/HondaYC08} with global assertions, describing
global constraints on processes' interactions in terms of predicate
logic formulae. In this way, types not only describe causal relations
between the inter-process communications, but they also fulfil
constraints regarding the values in transit.

In \cite{Berger2008Completeness-an}, a proof system characterising
May/Must testing pre-orders and bisimilarities over typed
\mipi-calculus processes is presented. The connection between types
and logics in this system comes in handy to restrict the shape of the
processes one might be interested, allowing us to consider such work
as a suitable proof system for calculi describing the communication of
end points.

In the context of security, the work on F7 \cite{gordon2009principles} has explored the
integration of dependent and refinement types in a suite of functional
programming languages, with the aim of statically checking assertions
about data and state, in order to enforce security policies.


%%%%%%%%%%%%%%%%%%%%%%%%%%%%%%%%%%%
%% Places 2010 Version
%%%%%%%%%%%%%%%%%%%%%%%%%%%%%%%%%%%


% \section{Future work}

% \label{Logic4Struct:sec:futurework}


% The work here presented constitutes just the first from the initial
% steps towards a verification framework of structured
% communications. Our main concerns relate to (i) establish a
% completeness relation between the choreography logic and its proof
% system, (ii) the ability of integrate the proof system here presented
% with other logical frameworks for the specification of sessions, and
% (iii) the ability of reasoning about partial information within the
% framework. In \cite{Berger2008Completeness-an}, Berger et
% al. presented proof systems characterizing May/Must testing preorders
% and bisimilarities over typed \mipi-calculus processes. The connection
% between types and logics in such system comes in handy to restrict the
% shape of the processes one might be interested, allowing us to
% consider such work as a suitable proof system for the calculus of end
% points. Moreover, a connection between the global logic here presented
% and the modal logic used in typed \mipi-specifications results
% necessary, allowing us to project a property satisfied in a
% choreography into sets of properties in \mipi-calculus
% specifications. Other improvements to the system proposed include the
% use of fixed points, essential for describing state-changing loops,
% and auxiliary axioms describing not only structural properties of a
% choreography, but also expected results given by the interaction of
% one or more choreographies.


%%% Local Variables: 
%%% mode: latex
%%% TeX-master: "../Thesis"
%%% End: 
