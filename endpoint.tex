\section{A calculus for orchestrations: The End-point Calculus }
\label{Logic4Struct:sec:epc}

\subsection{Syntax}
The end-point calculus (EPC) \cite{carbone7scc} is the $\pi$-calculus
\cite{milner:99:cmspc} extended with sessions \cite{honda1998lpa} as
well as locations \cite{hennessy2007distributed} and store
\cite{DBLP:conf/fsttcs/CarboneNS04}.  Below, $P, Q, \ldots$ denote {\em processes}, $M,
N, \ldots$ {\em networks}.
%\begin{table}[ht!]
\[\label{Logic4Struct:endpointsyntaxlable}
\begin{array}{ll}
%\begin{align*}
\begin{array}{rllll}
  P ::=
  &\phantom{{}\mid\quad}  \repInitIn {a}{\VEC{k}}\pfx P & \text{(initin)} 
  &\mid \quad \initOut{a}{\VEC{k}}\pfx P                  & \text{(initout)}\\
  &\mid\quad \send{k}{e} P                    &\text{(send)} 
  &\mid\quad \receive{k}{x} P                &\text{(receive)}\\
  &\mid\quad \selection{k}{\mathsf{l}} P  &\text{(label selection)} 
  &\mid\quad \branching{k}{\sum_i \mathsf{l_i} \pfx P_i}  &\text{(label branching)}\\
%   &\mid \quad\inputBranch{s}{i}{op}{{y}}{P}           & \text{(branch)} \\ 
%   &\mid\quad\outputP{s}{op}{{e}}\pfx P                & \text{(out)}\\
%  &\mid \quad x:=e\pfx P                                 & \text{(assign)} 
  &\mid \quad P_1\oplus P_2                              & \text{(plus)}    
  &\mid\quad  P_1\pp P_2                                 &
  \text{(par)}\\  
  &\mid \quad \rec X\pfx P                              & \text{(rec)}   
  &\mid\quad  X                                         & \text{(recvar)}\\
 &\mid\quad\itn{e}{P_1}{P_2}                     & \text{(cond)} 
%   &\mid \quad \new s P                                  & \text{(new)}     
  &\mid\quad  \INACT                                    & \text{(inact)}\\
\end{array}
\\\\ 

\begin{array}{rll}
  N
  ::=
  &\phantom{{}\mid\quad}   \pr AP\sigma
  &\text{(participant)}\\
  &\mid\quad  N_1\pp N_2 &\text{(parnet)}\\
%   &\mid \quad \new s N  & \text{(newnet)}    \\
  &\mid\quad  \INACTNW  &\text{(inactnet)}
\end{array}
\end{array}
\]
% \caption{End Point Calculus: Syntax }
% \end{table}
%\end{align*}


(initin) and (initout) are dual operations for describing session
initiation: $\repInitIn{a}{\VEC{k}} \pfx P$ denotes a process
offering a replicated (available in many copies) service $a$ with
session channels $\VEC{k}$ while $\initOut{a}{\VEC{k}} \pfx P$ denotes a
process requesting a service $a$ with session channels $\VEC{k}$. In
both cases, $P$ is the continuation. The next two processes denote
standard in-session communications (where $y_i$ in the first
construct, the branching input, is not bound in $P_i$, and $\{l_i\}$
should be pairwise distinct). 
%Next, $x := e \pfx P$ assigns the value
%of $e$ to $x$ in the store then continues as $P$.  
The term $\Did{plus}$
denotes internal choice. The rest is standard. Networks are parallel
composition of participants, where a participant has the shape $\pr
AP\sigma$, with $A$ being the name of the participant, $P$ its
behaviour, and $\sigma$ its local state, now interpreted as a local
function from variables to values. We often omit $\sigma$ when
irrelevant. The free session channels, free term variables and service
channels are defined as usual over processes and networks and,
similarly to the global calculus, are denoted by $fsc(P/N), fv(P/N)$ and
$channels(P/N)$ respectively. The syntax here presented differs from
its original presentation in the absence of the local assignments 
and restriction of networks and processes.

\subsection{Semantics}

Similarly to the Global Calculus, the EPC is equipped with a
structural congruence relation over networks and thread processes.
\begin{definition}[Structural Congruence in EPC]
The structural congruence relation $\equiv$ in EPC is the least
congruence on processes and networks such that $(\LITEQ, \INACT,
\oplus)$, 
% $(\LITEQ, \INACT, +)$, 
$(\LITEQ, \INACT, \pp)$ and $(\LITEQ,
\epsilon, \pp)$
% , $(\pp,\INACTNW)$ 
are commutative monoids and such
that 
% $ a) \new{s}\INACT \LITEQ \INACT$, 
% $ b) \new{s_1}\new{s_2}P
% \LITEQ \new{s_2}\new{s_1}P$, 
% $ c) \new{s}P\pp Q \LITEQ \new{s}(P\pp Q)
% ~(\text{for }s\not\in\fsc{Q})$, 
% $ d) \pr{A}{P}{\sigma} \LITEQ
% \pr{A}{Q}{\sigma} ~ (\text{for }P \LITEQ Q)$, 
% $ e)\pr{A}{\new{s}P}{\sigma} \LITEQ \new{s}(\pr{A}{P}{\sigma})$, 
% $ f)\new{s_1}\new{s_2}M \LITEQ \new{s_2}{\new{s_1}}M$, 
% $ g)\new{s}\INACTNW \LITEQ \INACTNW$, 
% $ h) \new{s}M\pp N \LITEQ
% \new{s}(M\pp N) ~(\text{for }s\not\in\fsc{N})$, 
% $ i)\pr{A}{\INACT}{\sigma} \equiv \INACTNW$.
$ \pr{A}{P}{\sigma} \LITEQ
\pr{A}{Q}{\sigma} ~ (\text{for }P \LITEQ Q)$, and
$ \pr{A}{\INACT}{\sigma} \equiv \INACTNW$.
\end{definition}

We give an operational semantics in terms of  configurations $ N
\action{\emm} N' $, where $N$ and $N'$ are networks and $\emm$ belongs
to the sets of labels $\{ \tau,\,\asynchServiceOutF{s}{k},\,
\asynchServiceInputF{s}{k},\,\asynchOutputF{k}{x},\,
\asynchInputF{k}{x},\, \asynchBranch{k}{l},\, \asynchSelection{k}{l}
\} $.  Its labelled transition semantics follows the $\pi$-calculus
and is defined by the rules given in
Figure~\ref{Logic4Struct:table:endpoint:sos}. Note that symmetric rules are omitted.
\begin{myfigure}{t!}
  \begin{align*} &
\myruleg{    E-S.Init.O} 
{    }
    { {\initOut{a}{\VEC{k}}\pfx P}
      \action{\asynchServiceOutF{a}{\VEC{k}}} P  }
~
\myruleg{    E-S.Init.I} 
    {    }
    { {\repInitIn{a}{\VEC{k}}\pfx P}
      \action{\asynchServiceInputF{a}{\VEC{k}}}
      P \pp \repInitIn{a}{\VEC{k}}\pfx P    }
~
\myruleg{    E-M.Out} 
{ e\Downarrow v }{
      {\outputP{k}{e} \pfx P} \action{\asynchOutputF{k}{v}}
      {P} }
~
\myruleg{    E-M.In} 
{ }{
      \inputP kx\pfx P
      \action{\asynchInputF{k}{x}}
      P }
    \\&
\myruleg{    E-L.Sel} 
{ }{
      \selection{k}{\mathsf{l}} P
      \action{\asynchSelection{k}{l}}
      P }
~
\myruleg{    E-L.Branch} 
{ 1\leq j\leq i }{
      \branching{k}{\sum_i \mathsf{l_i} \pfx P_i}
      \action{\asynchBranch{k}{l_j}}
      P_j }
~
\myruleg{    E-Sum }
{ i\in\{1,2\} }
    {  {P_1\oplus P_2} \action{\tau} {P'_i}    }
~
\myruleg{    E-Par.P}
{  P \action{\emm}  P' } 
    {  P\pp Q  \action{\emm} P'\pp Q}
\end{align*}
\caption{End Point Calculus: LTS semantics for Processes}
\label{Logic4Struct:table:endpoint:sosP}
\end{myfigure}


\begin{myfigure}{t!}
  \begin{align*} &
\myruleg{ E-Part}
    { P \action{\emm}  P' 
      ~
      \emm \neq \asynchInputF{k}{x} }
    {
      \pr A P\sigma
      \action{\emm}
      \pr A {P'}\sigma
    } 
 \myruleg{E-Par.N}
{ M \action{\emm}  M' }  {
      M \pp N \action{\emm} M' \pp N } 
~
\myruleg{ E-Part.In}
{ P \action{\asynchInputF{k}{x}}  P'  }
    {
      \pr A P\sigma 
      \action{\asynchInputF{k}{v}}
      \pr A {P'}{\sigma[x\mapsto v]}
    } 
\\&
\myruleg{ E-Com} 
{  N \action{\emm} N'
      ~
      M \action{\ol\emm} M'
      ~
      \emm \text { not init}
%      \emm\not\in\{\asynchServiceInputF {ch}{s},\asynchServiceOutF {ch}{s}\}
    }
    { N \pp M
      \action{\tau}
      N' \pp M'
    }
~
%     \Did{E-Res}\
%     &
%     \Rule{ M \action{\emm}   M'
%       \qquad
%       s\not\in\fn (m)  } {
%       \new{s}M   \action{\emm}    \new{s}M'   }
%     &
\myruleg{E-Init} 
{ 
      N \action{\asynchServiceOutF {a}{\VEC{k}}} N'
      ~
      M \action{\asynchServiceInputF {a}{\VEC{k}}} M' 
      % ~
%       \VEC{k} \not \in fn(N') \cup fn(M') 
      \quad \VEC{h} \text{ is fresh}
    }
    { N \pp M
      \action{\tau}
      (N' \pp M')[\VEC{h}/\VEC{k}]
    }
\\& 
\myruleg{E-IfT }{
  \sigma\vdash e \converges \true \qquad 
  \pr{A}{P_1}{\sigma}  \action{\emm}   \pr{A}{P'_1}{\sigma}
} 
{  \pr{A}{\itn{e}{P_1}{P_2}}{\sigma}
  \action{\emm}
  \pr{A}{P'_1}{\sigma}} 
~
\myruleg{E-IfF }
{
  \sigma\vdash e \converges \false \qquad 
  \pr{A}{P_2}{\sigma}  \action{\emm}   \pr{A}{P'_2}{\sigma}
} 
{
  \pr{A}{\itn{e}{P_1}{P_2}}{\sigma}
  \action{\emm}
  \pr{A}{P'_2}{\sigma}
}
\end{align*}
  \caption{End Point Calculus: LTS semantics for Networks}
  \label{Logic4Struct:table:endpoint:sos}
\end{myfigure}

%%%%%%%%%%%%% HUGO's VERSION
% \begin{table}[ht!]
% {\scriptsize \begin{align*} 
%     {\tiny \Did{E-S.Init.O}} & 
%     \Rule{
%       \VEC{s} \not \in \fsc{P'}}
%     {
%       \pr{A}{\initOut{ch}{\VEC{s}}\pfx P \pp P'}{\sigma}
%       \action{\asynchServiceOutF{ch}{\VEC{s}}}
%       \pr{A}{P \pp P'}{\sigma} 
%     }&
%     {\tiny     \Did{E-Par.N} }&
%     \Rule{
%       M \action{\emm}  M' } 
%     {
%       M \pp N \action{\emm} M' \pp N } 
% \end{align*}\vspace{-0.4cm}
% \begin{align*} 
%     {\tiny \Did{E-S.Init.I} }&
%     \Rule{ 
%       \pr{A}{P}{\sigma}
%       \action{\asynchServiceOutF{ch}{\VEC{s}}} \pr{A}{P'}{\sigma} \qquad
%       \VEC{s} \not \in \fsc{Q'} }
%     { \pr{A}{P}{\sigma} \pp \pr{B}{ \repInitIn{ch}{\VEC{s}}\pfx Q \pp Q'}{\sigma'}
%       \action{\asynchServiceInputF{ch}{\VEC{s}}}
%       \new{\VEC{s}} \pr{A}{P'}{\sigma} \pp \pr{B}{\repInitIn{ch}{\VEC{s}} \pfx Q \pp
%         Q \pp Q'}{\sigma'}
%     }
% \end{align*}\vspace{-0.4cm}
% \begin{align*} 
%     {\tiny     \Did{E-M.Out} }& 
%     \Rule{
%       \sigma \vdash e@A\Downarrow v
%     }{
%       \pr{A}{\outputP{s}{e} \pfx P}{\sigma} \action{\asynchOutputF{s}{v}}
%       \pr{A}{P}{\sigma}
%     }
%     &{\tiny     \Did{E-L.Sel} }& 
%     \Rule{
%     }{
%       \pr{A}{\selection{s}{\mathsf{op}} P}{\sigma} \action{\asynchSelection{s}{op}}
%       \pr{A}{P}{\sigma}
%     }
% \end{align*}\vspace{-0.4cm}
% \begin{align*} 
%     {\tiny     \Did{E-M.In} }& 
%     \Rule{
%       \pr{A}{P}{\sigma} \action{\asynchOutputF{s}{v}}
%       \pr{A}{P'}{\sigma} \qquad \sigma''=\sigma[x@B\mapsto v]}
%     {
%       \pr{A}{P}{\sigma} \pp \pr{B}{\inputP{s}{x} \pfx Q }{\sigma'} \action{\asynchInputF{s}{x}}
%       \pr{A}{P'}{\sigma} \pp \pr{B}{Q }{\sigma''} }\\[0.05cm]
%     {\tiny     \Did{E-L.Branch} }& 
%     \Rule{
%       \pr{A}{P}{\sigma} \action{\asynchSelection{s}{op_i}}
%       \pr{A}{P'}{\sigma} }
%     {
%       \pr{A}{P}{\sigma} \pp \pr{B}{\branching{s}{\sum_i \mathsf{op_i} \pfx Q_i} }{\sigma'} \action{\asynchBranch{s}{op_i}}
%       \pr{A}{P'}{\sigma} \pp \pr{B}{Q_i }{\sigma''} }
% \end{align*}\vspace{-0.4cm}
% \begin{align*} 
% {\tiny \Did{E-IfT} }&
% \Rule
% {
%   \sigma\vdash e \converges \true \qquad 
%   \pr{A}{P_1}{\sigma}  \action{\emm}   \pr{A}{P'_1}{\sigma}
% } 
% {
%   \pr{A}{\ifthenelse{e}{P_1}{P_2}}{\sigma}
%   \action{\emm}
%   \pr{A}{P'_1}{\sigma}
% } &
%     {\tiny \Did{E-IfF} }&
% \Rule
% {
%   \sigma\vdash e \converges \false \qquad 
%   \pr{A}{P_2}{\sigma}  \action{\emm}   \pr{A}{P'_2}{\sigma}
% } 
% {
%   \pr{A}{\ifthenelse{e}{P_1}{P_2}}{\sigma}
%   \action{\emm}
%   \pr{A}{P'_2}{\sigma}
% }
% \\[0.05cm]   
%     {\tiny \Did{E-Par.P} }&
%     \Rule{ 
%       \pr{A}{P}{\sigma}
%       \action{\emm} \pr{A}{P'}{\sigma'}}
%     { \pr{A}{P \pp Q}{\sigma} 
%       \action{\emm}
%       \pr{A}{P' \pp Q}{\sigma'}
%     }
% &
%     {\tiny     \Did{E-Sum} }&
%     \Rule{
%       \pr{A}{P_i}{\sigma} \action{\emm} \pr{A}{P'_i}{\sigma}\quad i\in\{1,2\}
%     }{
%       \pr{A}{P_1\oplus P_2}{\sigma} \action{\emm} \pr{A}{P'_i}{\sigma} }
% \end{align*}\vspace{-0.4cm}
% \begin{align*}
% %     {\tiny   \Did{E-Rec} }&
% %   \Rule {
% %     \pr A {P\MSUBS{\mu X.P}{X}}{\sigma} \pp N 
% %     \action{\emm}   N' } 
% %   {
% %   \pr{ A }{\mu X.P}{\sigma} \pp N \action{\emm}  N' } &
%     {\tiny    \Did{E-Assign}\ }&
%    \Rule {
%      \sigma\proves e\converges v \quad 
%      \pr{A}{P}{\sigma} \action{\emm} \pr{A}{P'}{\sigma'}
%    } {
%      \pr{A}{x:=e\pfx P}{\sigma} 
%      \action{\emm}
%      \pr{A}{P'}{\sigma'[x\mapsto v]}}\\[0.05cm]
%     {\tiny    \Did{E-Res} }&
%    \Rule{
%      M \action{\emm}   M'
%    } {
%      \new{s}M   \action{\emm}    \new{s}M'} &
%    {\tiny \Did{E-Struct} }& 
%    \Rule{
%      M\equiv M'\quad M' \action{\emm} N'\quad N'\equiv N
%    } { 
%      M \action{\emm} N}
% \end{align*}
% }
% \caption{End Point Calculus: LTS semantics}
% \label{Logic4Struct:table:endpoint:sos}
% \end{table} 



Rules in the transition semantics for EPC treat processes and networks
differently. \Did{E-S.Init.O} and \Did{E-S.Init.I} describe 
session initiation from the point of view of the requester and
provider, respectively. Here, $\repInitIn{a}{k}.P$ denotes a
replicated service. Message passing communication over sessions are
described by\Did{E-M.Out} and \Did{E-M.In}, where $e\Downarrow v$
describes the evaluation of expression and $P[v/x] $ the substitution
of variables $v$ by $x$ in $P$. Label
selection/branching is given by \Did{E-L.Sel} and
\Did{E-L.Branch}. \Did{E-Par.P} and \Did{E-Sum} are standard rules
representing parallel composition and internal choice. Rules
\Did{E-Part} allows transition labels to travel out from processes to
networks without modifying the store. \Did{E-Part.In} modifies the
store of a participant after having exhibited an input
behaviour. \Did{E-Com} describes synchronisation of transition labels,
used for message passing and label selection between
networks. \Did{E-Init} represents session initiation between different
networks, here $w$ acts as a ``fresh'' variable, used to represent
the creation of a new session between $\network$ and $M$. Finally,  
\Did{E-Par.N} describes the parallel composition between networks


As it is expected, both the  labelled transition semantics for EPC above presented
and the reduction semantics presented in
Figure \ref{Logic4Struct:table:endpoint:reduction-semantics} coincide.


\begin{myfigure}{ht!}
\begin{align*}
&\myruleg{E-RInit}
  {     k_i\not\in\fsc{P'}\cup\fsc{Q'}   \quad \tilde{h} \text{ is
      fresh}}
  {   \pr {A}{\repInitIn{a}{\tilde k}\pfx P\pp P'}\sigma\pp
   \pr {B}{\initOut{a}{\tilde k}\pfx Q\pp Q'}{\sigma'}    \to \ 
   % \new  {\tilde k}
   (
   \pr {A} {\repInitIn{a}{\tilde k}\pfx P\pp P\pp P'} {\sigma}\pp \pr
   {B}{Q\pp Q'} {\sigma'}
   )[\tilde{h}/\tilde{k}]
  }   
\\[0.5cm]
& \qquad \myruleg{ E-RCom}
  {   \sigma\proves e\converges v   } 
  {  \pr{A}{\receive{k}{x} P \,|\, P'}{\sigma}
  \,|\, 
  \pr{B}{\send{k}{x} Q \,|\, Q'}{\sigma'}
  \to \ 
  \pr{A}{P\pp P'}{\sigma[x\mapsto v]}
  \pp 
  \pr{B}{Q\pp Q'} {\sigma'}  }
\\[0.5cm]
&\myruleg{ E-RSel}
  {   j \in I   } 
  {  \pr{A}{ \branching{k}{\sum_i \mathsf{l_i} \pfx P_i} \,|\, P'}{\sigma}
  \,|\, 
  \pr{B}{\selection{k}{l_j} Q \,|\, Q'}{\sigma'}
  \to \ 
  \pr{A}{P_j\pp P'}{\sigma}
  \pp 
  \pr{B}{Q\pp Q'} {\sigma'}  }
\\[0.5cm]
&\myruleg{ E-RIfT}
{  \sigma\vdash e \converges \true } 
{  \pr{A}{\itn{e}{P_1}{P_2}\,|\, P'}{\sigma}
  \;\to\; 
  \pr{A}{P_1\pp P'}{\sigma}}
\qquad \qquad 
\myruleg{ E-RPar}
{  M   \;\to\;    M' } 
{   M|N   \;\to\;    M'|N} 
% \quad 
% \myruleg{ E-RRes}
% {   M   \;\to\;    M' } 
% {  \new{k}M   \;\to\;    \new{k}M' } 
\\[0.5cm]
&\myruleg{ E-RIfF}
{  \sigma\vdash e \converges \false } 
{   \pr{A}{\itn{e}{P_1}{P_2}\,|\, P'}{\sigma}   \;\to\;    \pr{A}{P_2\pp P'}{\sigma}}
\quad
\myruleg{ E-RSum}
{ i\in\{1,2\} }  {   A[P_1\oplus P_2|R]_\sigma   \;\to\;
  A[P_i|R]_\sigma }   
\end{align*} 
\begin{align*}
% \Did{E-Res1} \ &
% \Rule
% {
%   \pr{A}{P}{\sigma}  
%   \;\to\; 
%   \pr{A}{P'}{\sigma'}  
% } 
% {
%   \pr{A}{\new{s}P}{\sigma}
%   \;\to\; 
%   \pr{A}{\new{s}P'}{\sigma}
% } 
% &
\myruleg{ E-RRec}
{
  \pr A {P\MSUBS{\mu X.P}{X}\pp Q}{\sigma}\pp N
  \;\to\; 
  N'
} 
{
  \pr A {\mu X.P\pp Q}{\sigma}\pp N
  \;\to\; 
  N'
}
&
% \myruleg{ E-RAssign} 
% {
%   \sigma\proves e\converges v
% } 
% {
%   \pr{A}{x:=e\pfx P\,|\, P'}{\sigma}
%   \;\to\; 
%   \pr{A}{P\pp P'}{\sigma[x\mapsto v]}
% }
% \\\\
% &
\qquad \myruleg{ E-RStruct}{M\equiv M'\quad M'\to N'\quad N'\equiv N} { M\to N}
\end{align*}
% \begin{align*}
% &
% % \Did{E-Par2} \ &
% % \Rule
% % {
% %   \pr{A}{P_1\,|\, R}{\sigma}  
% %   \;\to\; 
% %   \pr{A}{P'_1\,|\, R}{\sigma'}  
% % } 
% % {
% %   \pr{A}{P_1\,|\,  P_2\,|\, R}{\sigma}  
% %   \;\to\; 
% %   \pr{A}{P'_1\,|\,  P_2\,|\, R}{\sigma'}  
% % } 
% \end{align*}
\caption{Reduction Relation for the End-Point Calculus \cite{carbone7scc}}
\label{Logic4Struct:table:endpoint:reduction-semantics}
% \end{align*}

\end{myfigure}


\begin{lemma} \label{struct-cong-epc}
If $\network \equiv M$ and $\network \action{\emm} \network'$  then $M
\action{\emm} \network'$.

\begin{proof}
It follows by induction on the structural congruence rules in
    $\equiv$ and second induction on the height of the derivation tree
    for $\network \action{\emm} \network'$.
\end{proof}
\end{lemma}

\begin{proposition}[Reduction and LTS semantics coincide in EPC]
Given $\network, \network'$ networks, $->$ the reduction relation
between networks  
% in \cite{carbone7scc} (given for readability in
% Appendix \ref{Logic4Struct:appendix-semanticsEPC})
 and $\action{\emm}$ the
transition relation between networks in EPC.
%  given in Figure \ref{Logic4Struct:table:endpoint:sos}.
 We
can say that:

\begin{description}
  \item[Soundness]: If $ \network -> \network'$, then
    $\exists \emm $ s.t. $ \network \action{\emm} \network'$.
  \item[Completeness]: If $ \network \action{\tau} 
    \network'$ then $\network ->  \network'$. 
\end{description}
\begin{proof}\hfill
\begin{description} 
  \item[(On Soundness):] \hfill \\
The proof proceeds by induction on the height of the derivation tree
for $\network -> \network'$, with a case analysis on the last applied
rule. We have the following cases:
% The proof proceeds by induction on the length of the
%  reductions in $->$.
 
 \begin{description}
   \item[Case $\Did{E-RInit}$:] We have that $\network =
     \pr{A}{\repInitIn{a}{\VEC{k} \pfx P} \pp P'}{\sigma} \pp
     \pr{B}{\initOut{a}{\VEC{k}} \pfx Q \pp Q'}{\sigma'}$ and we have
     that for $k_i \in \VEC{k}$, $k_i \in fsc(P') \cup fsc(Q')$ and
     $h$ is fresh, and $\network -> (\pr{A}{\repInitIn{a}{\VEC{k}}
       \pfx P \pp P'}{\sigma} \pp  \pr{B}{Q \pp Q'}{\sigma'})[\VEC{h}/
     \VEC{k}]$.

     From the application of $\Did{E-Part}$, \Did{E-Par.P} and \Did{E-S.Init.I} in
     $\network$, we have that $\network_1 =
     \pr{A}{\repInitIn{a}{\VEC{k}} \pfx P  \pp P'}{\sigma}
     \action{\asynchServiceInputF{a}{\VEC{K}}} \pr{A}{
       \repInitIn{a}{\VEC{k}} \pfx P \pp P \pp P'}{\sigma}$.

     Similarly, from the application of \Did{E-Part}, \Did{E-Par.P} and \Did{E-S.Init.O} we get
     $\network_2 =  \pr{B}{\initOut{a}{\VEC{k}} \pfx Q \pp
       Q'}{\sigma'}$ $
     \action{\asynchServiceOutF{a}{\VEC{K}}} \pr{B}{ Q \pp
       Q'}{\sigma'} $.

     Finally, after application of \Did{E-Int} we have that
      \[ \network_1 \pp \network_2 \action{\tau}
     (\pr{A}{\repInitIn{a}{\VEC{k}} \pfx P \pp P \pp P'
       }{\sigma} \pp \pr{B}{Q \pp
       Q'}{\sigma'})[\VEC{h}/ \VEC{k}], \] which is what we had to show.
 
     \item[Case \Did{E-RCom}:] We have that $\network =
       \pr{A}{\inputP{k}{x} \pfx \pp P'}{\sigma} \pp
       \pr{B}{\outputP{k}{x} \pfx Q \pp Q'}{\sigma'}$, and given
       $\sigma |- e@\Downarrow v$ then $\network -> \pr{A}{P \pp P'}{\sigma[x
         \mapsto v]]} \pp \pr{B}{Q \pp Q'}{\sigma'}$.

       After the application of \Did{E-Part}, \Did{E-Par.P} and
       \Did{E-M.Out} over $\network_2 = \pr{B}{\outputP{k}{x} \pfx Q
         \pp Q'}{\sigma'}$ then $\network_2
       \action{\asynchOutputF{k}{v}} \pr{B}{ Q
         \pp Q'}{\sigma'}$.

       Similarly, the application of \Did{E-Part.In},\Did{E-Par.P} and
       \Did{E-M.In} to $\network_1 = \pr{A}{\inputP{k}{x} \pfx P \pp
         P'}{\sigma} $ leads to $\network_1
       \action{\asynchInputF{k}{v}} \pr{A}{P \pp
         P'}{\sigma[x \mapsto v]}$.

       Finally, the application of \Did{E-Com} over $\network_1 \pp
       \network_2$ leads to $\network_1 \pp \network_2 \action{\tau}
       \pr{A}{P \pp P'}{\sigma[x \mapsto v]} \pp \pr{B}{ Q \pp
         Q'}{\sigma'}$, which is what we had to show.
       
       \item[Case \Did{E-RSel}:] We have that $\network =  \pr{A}{
           \branching{k}{\sum_i \mathsf{l_i} \pfx P_i} \,|\,
           P'}{\sigma} \,|\, \pr{B}{\selection{k}{l_j} Q \,|\,
           Q'}{\sigma'}$ and, provided $j \in I$, then $\network ->
         \pr{A}{P_j\pp P'}{\sigma}  \pp   \pr{B}{Q\pp Q'}{\sigma'} $.

         After the application of \Did{E-part}, \Did{E-Par.P} and
         \Did{E-L.Branch} over $\network_1 = \pr{A}{
           \branching{k}{\sum_i \mathsf{l_i} \pfx P_i} \,|\,
           P'}{\sigma}$ we get that $\network_1
         \action{\asynchBranch{k}{l_j}} \pr{A}{P_j \pp P'}{\sigma}$.

         Similarly,   after the application of \Did{E-part}, \Did{E-Par.P} and
         \Did{E-L.Branch} over $\network_2 = \pr{B}{\selection{k}{l_j} Q \,|\,
           Q'}{\sigma'}$ we get that $\network_2
         \action{\asynchSelection{k}{l_j}} \pr{B}{Q \pp Q'}{\sigma'}$.

         Finally, after the application of \Did{E-Com} over
         $\network_1 \pp \network_2$ we get: $\network_1 \pp
         \network_2 \action{\tau} \pr{A}{P_j \pp P'}{\sigma} \pp
         \pr{B}{Q \pp Q'}{\sigma'}$, and we are done.

         \item[Cases \Did{E-RifT} and \Did{E-RifF}:] 
           It must be the case that $\network =
           \pr{A}{\itn{e}{P_1}{P_2} \pp P'}{\sigma}$ and assume $\sigma |- e
           \Downarrow \true$ and $\pr{A}{P_1}{\sigma} -> \pr{A}{P'_1}{\sigma}$, then  $\network ->
           \pr{A}{P'_1 \pp P'}{\sigma}$. One can easily show that
           $\pr{A}{\itn{e}{P_1}{P_2} \pp P'}{\sigma} \action{\emm}
           \pr{A}{P'_1 \pp P'}{\sigma}$ after the application of rules 
           \Did{E-Par.P} and
           \Did{E-IfT} and the induction hypothesis
           $\pr{A}{P_1}{\sigma} \action{\emm} \pr{A}{P'_1}{\sigma}$ (the converse case is
           symmetrical).  
           
           \item[Case \Did{E-RPar}, \Did{E-RSum} and \Did{E-RRec}:]
             They follow directly after simple rule induction.

             \item[Case \Did{E-RStruct}:]
               Given $\network'' -> \network'''$ then we
               can assume that $\network \equiv \network''$ and
               $\network -> \network' \land \network' \equiv
               \network'''$. We need to show that $\network''
               \action{\emm} \network'''$. From Lemma
               \ref{struct-cong-epc} and the inductive hypothesis
               $\network \equiv \network'' \land \network
               \action{\emm} \network' \land \network' \equiv
               \network'''$ then $\network'' \action{\emm}
               \network'''$.

  \end{description}

 \item[On Completeness:] \hfill \\ 
The proof proceeds by induction on the height of the derivation tree
for $\network \action{\tau} \network'$, with a case analysis on the
last applied rule.  
% The proof proceeds by induction on the length of the
%  derivations in $\action{\tau}$.

 \begin{description}
   \item[Case \Did{E-Com}:] We have that $M \action{\overline \emm} M'$
     and $\network \action{\emm} \network'$ where $m$ is not part of the
     initialisation labels $\{\asynchServiceOutF{a}{k},
     \asynchServiceInputF{a}{k} \}$, then $\network \pp M
     \action{\tau} N' \pp M'$. We have to show that $\network \pp M
     -> N' \pp M'$. 

     We proceed by case analysis on $\emm$:
     \begin{itemize}
       \item {\bf ($\emm = \asynchOutputF{k}{v}$ and $\overline{\emm} =
           \asynchInputF{k}{v}$):}
         Assume $\network  = \pr{A}{\outputP{k}{e} \pfx P'}{\sigma}$
         and $M = \pr{B}{\inputP{k}{x} \pfx Q'}{\sigma'}$ and $\sigma
         |- e\Downarrow v$.
         
         On the one hand, after application of \Did{E-M.Out},
         \Did{E-M.In}, \Did{E-Part.In}, \Did{E-Part} and \Did{E-Com}
         in the LTS, we have that $M \pp \network \action{\tau}
         \pr{A}{P'}{\sigma} \pp \pr{B}{Q'}{\sigma'[x \mapsto v]}$.

         On the other hand, after the application of \Did{E-RCom}  to
         $M \pp \network$ we get that $M \pp \network ->
         \pr{A}{P'}{\sigma} \pp \pr{B}{Q'}{\sigma'[x \mapsto v]}$, so
         we are done.

         \item {\bf ($\emm = \asynchSelection{k}{l}$ and $\overline{\emm}
             = \asynchBranch{k}{l}$:)} 
           Assume w.l.o.g. $\network =
           \pr{A}{\selection{k}{l} \pfx P'}{\sigma}$ and $M =
           \pr{B}{\branching{k}{\sum_i \mathsf{l_i} \pfx
               Q_i}}{\sigma'}$. From the application of \Did{E-L.Sel},
           \Did{E-L.Branch}, \Did{E-Part} and \Did{E-Com} in the LTS
           semantics, we get that $ \network \pp M \action{\tau}
           \pr{A}{P'}{\sigma} \pp \pr{B}{Q_i}{\sigma'}$.
           Similarly, from the application of \Did{E-RSel} in the
           reduction semantics we get that $\network \pp M ->
           \pr{A}{P'}{\sigma} \pp \pr{B}{Q_i}{\sigma'}$ so we are done.
       \end{itemize}

       \item[Case \Did{E-Init}:] We have that $M
         \action{\asynchServiceOutF{a}{\VEC{k}}} M'$ and $\network \action{\asynchServiceInputF{a}{\VEC{k}}}
         \network'$. Then applying \Did{E-Init} in the LTS semantics
         we get $\network \pp M \action{\tau} (\network' \pp
         M')[\VEC{h}/\VEC{k}]$. We have  to show that $\network \pp M -> (\network' \pp
         M')[\VEC{h}/\VEC{k}]$.

         Assume w.l.o.g. $M = \pr{A}{\repInitIn{a}{\VEC{k} \pfx P
           }}{\sigma}$ and $\network =
         \pr{B}{\initOut{a}{\VEC{k}}}{\sigma'}$. After the application
         of \Did{E-S.Init.O}, \Did{E-Part}, \Did{E-S.init.I} and
         \Did{E-Init} in the LTS, we have that $M \pp \network
         \action{\tau} ( \pr{A}{ \repInitIn{a}{\VEC{k}} \pfx P \pp
             P}{\sigma} \pp
           \pr{B}{Q}{\sigma'})[\VEC{k}/\VEC{h}]$. Similarly, after  applying
           \Did{E-RInit} to $M \pp \network$ we get that $M \pp
           \network -> ( \pr{A}{ \repInitIn{a}{\VEC{k}} \pfx P \pp
             P}{\sigma} \pp  \pr{B}{Q}{\sigma'})[\VEC{k}/\VEC{h}]$, which is what we had to
           show.

           \item[Case \Did{E-Sum}]: It follows immediately after the
             application of \Did{E-RSum}.
   \end{description}

\end{description}
\end{proof}
\end{proposition}



\subsection{Session Types for the End-Point Calculus}
\label{Logic4Struct-EndPointTypes}
Session types for the EPC use the syntax of session types in
equation \ref{Logic4Struct:eq:sessiontypes}. Basically, the type
discipline of the EPC stems from the Global Calculus, but assigns
session types to every single participant instead of the whole
choreography. In this way, the session typing in the EPC describes the
end-point behaviour.
% \begin{definition}[
% End-Point Typing Judgements
  An {\it end-point typing judgment} contains judgements for processes in  the form $\tprovesA{\Gamma}{P}{\Delta}$ (where $P$ is
    typed as a behaviour for $A$) and judgements for networks
    $\tproves{\Gamma}{N}{\Delta}$. In both,  mappings $\Gamma$ and $\Delta$ are {\em service} and {\em session typings} respectively.
%    triple on processes or
%   networks:\\
%   \begin{tabular}{ll}
%     (Processes) & $\tprovesA{\Gamma}{P}{\Delta}$ (where $P$ is
%     typed as a behaviour for $A$)\\
%     (Networks) & $\tproves{\Gamma}{N}{\Delta}$.  
%   \end{tabular}\\
%   where the mappings $\Gamma$ and $\Delta$ are the {\em service} and the {\em session typing} respectively.
% \end{definition}
Here, $\Gamma$ 
% (service typing)
 and $\Delta$ 
% (session typing) 
are defined
as:
\begin{align*}
&\Gamma  \ \; ::=\;\ \emptyset
                  \;\ |\;\ \Gamma, a@A\!:\!(\VEC{k})\alpha
                  \;\ |\;\ \Gamma, \ol{a}@A\!:\!(\VEC{k})\alpha
                  \;\ |\;\ \Gamma, x@A\!:\!\theta
                  \;\ |\;\ \Gamma, X\!:\!\Delta
%\qquad
\\
&\Delta  \ \; ::=\;\ \emptyset
              \;\ |\;\ \Delta, \VEC{k}@A\!:\!\alpha
              \;\ |\;\ \Delta, \VEC{k}\!:\!\bot
\end{align*}

Above, $a@A\!:\!(\VEC{k})\alpha$ indicates the service located at $A$
which is invoked with fresh session channels $\VEC{k}$ and offers
service of the shape $\alpha$, while $\ol{a}@A\!:\!(\VEC{k})\alpha$
indicates the type abstraction for the dual invocation, i.e. a client
of a service offered by $A$ which invokes with fresh channels $\VEC{k}$ and
engages in interactions abstracted as $\alpha$. Note $@A$ indicates
the location of a service in both forms. 
As before, $\VEC{k}$ should be a vector of pairwise distinct session
channels which should cover all session channels in $\alpha$, and
$\alpha$ does not contain free type variables. $(\VEC{k})$ binds
occurrences of session channels in $(\VEC{k})$ in $\alpha$, which
induces the standard alpha-equality. A central concept in this type
discipline is the notion of duality for session types, which is defined as:
\[
\ol{\!(\VEC{k})\alpha}@A = ? (\VEC{k}) \ol{\alpha}@A ~~~~~~
\ol{? (\VEC{k}) \alpha@A} = \! (\VEC{k}) \ol{\alpha}@A 
\]
where the notion of duality $\alpha$ of $\alpha$ remains the same.




The typing rules are almost identical as the ones from the original
presentation of the EPC \cite{carbone7scc}, where the only difference
lies on the separation between input-output types and
selection-branching types as originally presented in
\cite{honda1998lpa}. Here we only comment some examples on the typing
rules, and the full type system can be found in Appendix
\ref{Logic4Struct:appendix:EPC-Types}. As in our account of the type
system for the Global Calculus, we here devote our attention to the
rules for session initiation and communication.  The two rules
\Did{E-TInit.In},\Did{E-TInit.Out} describe session initiation
primitives:


\begin{align*}
  \myrule{ E-TInit.In} 
  { \typeruleE{\typeEnv }{A}{ P \ccc
      \VEC{k}@A:\alpha} 
    \quad a \not\in dom(\typeEnv) 
    \quad    client(\typeEnv) 
  } { 
    \typeruleE{\typeEnv, a @ A: (\VEC{k}) \alpha }{A}{
      !a(\VEC{k}) \pfx P \ccc \emptyset}
  }
  \quad 
  \myrule{ E-TInit.Out} 
  { \typeruleE{\typeEnv, a@B:(\VEC{k})\alpha}{A}{ P \ccc \Delta \cdot
      \VEC{k}@A:\alpha}  } 
  { \typeruleE{\typeEnv, a @B: (\VEC{k})\alpha }{A}{ \overline{a}
      <. \VEC{k} .> P \ccc \Delta}}
\end{align*}

In \Did{E-TInit.In}, the premise only allows for typings of session
channels involved in the session initialisation of service $a$, that
is,  only the channels in $\VEC{k}$. This linearity condition blocks
free session channels from occurring during a replicated input. The
condition $a \not\in dom(\typeEnv)$ prevents from self-calls and
ensures that the type assignment occurs at the side of the client.
Requirements for the complementary typing rule \Did{E-TInit.In} are
analogous, although the linearity condition is removed. Communication
rules are standard for session types, for instance, the rule
\Did{E.TOut} is used to type message outputs:

\begin{align*}
  \myrule{    E.TOut}
    { 
      k \in \VEC{k} \quad
      \typeruleE{\typeEnv  }{A}{ 
        P  \ccc \Delta \cdot \VEC{k}@A:\alpha} \quad 
      \typeruleE{\typeEnv  }{}{ e : \theta} 
    } 
    { \typeruleE{\typeEnv  }{A}{
        \send{k}{e}  P  \ccc \Delta
      \cdot \VEC{k}@A: k \uparrow \theta
    \pfx \alpha}}
\end{align*}

Here, process $\send{k} {e}  P$ types after evaluation that the typing
of $e$ corresponds to a correct value type and that the continuation
$P$ behaves as established by the session type  in $\Delta \cdot
\VEC{k}@A:\alpha$. Analogous requirements hold for typing the input
process $ \receive{k}{x}  P$.

% \begin{theorem}\label{theorem:epc:sr}\

%   \begin{enumerate}
    
%   \item (Subject Congruence) If $\tproves{\Gamma}{M}{\Delta}$ and
%     $M\LITEQ N$ then $\tproves{\Gamma}{N}{\Delta}$;

%   \item (Subject Reduction) If $\Gamma\vdash N\rhd\Delta$ and
%     $N\rightarrow N'$ then $\Gamma\vdash N'\rhd\Delta$.

%   \end{enumerate}

% \end{theorem}


%%%%%%%%%%%%%%%%%%%%%%%%%%%%%%%%%%%%%%%%%%%%%%%%%%%%
%%% Local Variables: 
%%% mode: latex
%%% TeX-master: "../Thesis"
%%% End: 
%%%%%%%%%%%%%%%%%%%%%%%%%%%%%%%%%%%%%%%%%%%%%%%%%%%%


\subsection{End Point Projection}
\label{Logic4Struct:sec:epp-types}


The relation between global and local views at the specification of
communication protocols is given at the level of types. The central
idea is that one can \emph{project} the behaviour (type) of a
global specification given in terms of choreography into a parallel
composition of the behaviours of end-points. The mapping is far from
trivial, and needs to preserve causal relations between messages and
threads, namely \emph{connectedness}, \emph{well-threadedness} and
\emph{coherence}. The next subsection
presents a recap from the work at \cite{carbone7scc}. We will use
these definitions (specially Theorem \ref{Logic4Struct:theorem:epp}
and Definition \ref{Logic4Struct:definition:pruning}) in order to
relate the work on end-point projections with their corresponding
logical counterpart.
In order to give the formal definition of end point projection, we
first annotate global specifications with identifiers for threads.

% \begin{definition}[Annotated Interactions]
  An annotated interaction is an annotation of a choreography with
  $t$'s denoting each thread in play. Annotated interactions are
  written $\mathcal{A,A', ...}$, and they are given by
  the following grammar:

  \begin{align*} 
    \CAL{A} ::= & ~~~\init{A^{t_1}}{ B ^{t_2}}{a}{k} \pfx \CAL{A}
    && |~~\CAL{A}_1 |^t \CAL{A}_2 \\
    & |~~\interact{A^{t_1}}{ B ^{t_2}}{k}{e}{y} \pfx \CAL{A}
    && |~~\mu^t X^A \pfx \CAL{A}\\
    & |~~\choice{A^{t_1}}{B ^{t_2}}{k}{l}{\CAL{A}}
    && |~~ X^A_t \\
    & |~~\itn{e@A^{t}}{\CAL{A}_1}{\CAL{A}_2}
    && |~~\INACT
\end{align*}
where each $t$ is a natural number. We call $t,t',\cdots$ occurring in
an annotated interaction, \emph{threads}. Each $\CAL{A}$ can be
regarded as an abstract syntax built from a constructor in its root
(either a prefix or a parallel product), if the tree is originated
from a single thread, or a pair of threads if the interaction involves
an interaction (session initiation, message communication or
selection/branching). The following is the consistent annotation of
(\ref{Logic4Struct:example:syntax}).


\begin{align}
      & \init{\text{Cust}^1}{\text{AC}^2}{\text{ob}}{k_1,k_2} \pfx
    \interact{\text{Cust}^1}{\text{AC}^2}{k_1}{\text{booking}}{x} \pfx \tag{$OB_{\CAL{A}}$} \\
     & ~~~~~~\interact{\text{AC}^2}{\text{Cust}^1}{k_2}{\text{offer}}{y} \pfx
    \interact{\text{Cust}^1}{\text{AC}^2}{k_1}{\text{accept}}{z} \pfx
    \INACT \notag 
\end{align}

This annotation, though simple, would become more complicated if there
more than one session initiation was involved in the choreography.
 Take for instance the case where
$\init{\text{Cust}}{\text{AC}}{\text{ob}}{k_1,k_2}$ is decomposed by
the sequence of processes
$\init{\text{Cust}}{\text{AC}}{\text{ob}}{k_1}\pfx$ $
\init{\text{AC}}{\text{Cust}}{\text{ob}}{k_2}$ We can have different
annotations for Cust and AC. The sequence:
$\init{\text{Cust}^1}{\text{AC}^2}{\text{ob}}{k_1}\pfx
\init{\text{AC}^2}{\text{Cust}^3}{\text{ob}}{k_2}$ generates a valid
annotation as it places each session initiation between the customer
and the AC in different threads, and any other annotation would be
invalid.
% \end{definition}


A choreography $\chor$ is \emph{connected}, if in communication
actions, only the message reception leads to activity (at the
receiving participant), and that such activity should immediately
follow the reception of messages. 
Informally, for each participant $A$ in the set of
participants of a choreography $\chor$, a communication activity
originated by $A$ should have been immediately preceeded by a communication
activity where $A$ had acted as a receiver, or been preceded by a
self-contained action (evaluation of expressions, for instance).

% With respect to threads, it is convenient to introduce some notation: $t_1$
%     (resp. $t_2$) is the \emph{active thread} of $\CAL{A}$ by $B$ (resp.
%     the \emph{passive thread} of $\CAL{B}$ by $C$) if the root of $\CAL{A}$
%     is initialisation/communication from 
%     $B$ to $C$ and it is annotated by $(t_1 , t_2 )$. We define
%     predecessors and successors in annotated interactions as normally
%     defined in ASTs: A direct predecessor is a predecessor which has no
%     intermediate predecessor. Symmetrically we define successor and
%     direct successor. 
% \begin{definition}[Threads]
%   \begin{enumerate}
%   \item If the root of $\CAL{A}$ is initialisation/communication from
%     $B$ to $C$ and is annotated by $(t_1 , t_2 )$, then $t_1$
%     (resp. $t_2$) is the active thread of $\CAL{A}$ by $B$ (resp.
%     the passive thread of $\CAL{B}$ by $C$). If the root of $\CAL{A}$ is
%     another constructor then its annotation $t$ is both its active
%     thread and its passive thread.
%   \item If $\CAL{A'}$ occurs as a proper subtree of $\CAL{A}$, then
%     (the root of) $\CAL{A}$ is a predecessor of (the root of)
%     $\CAL{A'}$. A direct predecessor is a predecessor which has no
%     intermediate predecessor. Symmetrically we define successor and
%     direct successor. 
% \end{enumerate}
% \end{definition}


\paragraph{Consistent annotations} In order to provide meaningful
projections between choreographies and its end-points, we need to
define a notion of ``consistent annotation'', that is, an annotation
$\CAL{A}$ such that it respects causality conditions, and can be
realised by a projection.
% \begin{definition}[Consistently Annotated Interactions] \label{Logic4Struct:def:cons-annotated-ints}
%   A well-typed annotated connected interaction (choreography)
%   $\mathcal{A}$ is consistent if the following conditions hold for
%   each of its subtrees, say $\mathcal{A'}$:
Such conditions are: 1) Causal Consistency: if a participant annotated
with $t$ is passive in an interaction (a receiver), then the
subsequent interaction will be marked with $t$ as well, or it will be
a self-contained action, 2) Session Consistency: Two actions in
$\CAL{A}$ identified by the same session name are annotated with the
same thread, and 3) Distinctness Condition: The input of session
initiation is always given a fresh thread.

%   \begin{itemize}
%   \item[--]Freshness Condition (FC):  If $t$ is by $\mathcal{A}$ at
%     some node and by $\mathcal{B}$ at another node then $\mathcal{A}$
%       and $\mathcal{B}$ always coincide. Further, if $\mathcal{A'}$
%       starts with an initialisation, then its passive thread should be
%       fresh w.r.t. all of its predecessors (if any).
%     \item[--]Session Consistency (SC): If $\mathcal{A'}$ starts with a
%       communication between $\mathcal{B}$ and $\mathcal{C}$ via (say)
%       $k$ and another subtree $\mathcal{A''}$ of $\mathcal{A}$ starts
%       with a communication via $k$ or an initialisation which opens
%       $k$, then the thread by $\mathcal{B}$ (resp. by $\mathcal{C}$)
%       of $\mathcal{A'}$ should be equal to the thread by $\mathcal{B}$
%       (resp. by $\mathcal{C}$) of $\mathcal{A''}$.
%     \item[--]Causal Consistency (CC): If $\mathcal{A''}$ is the direct
%       successor of $\mathcal{A'}$, then the active thread of
%       $\mathcal{A''}$ should coincide with the passive thread of
%       $\mathcal{A'}$.  We also say $\chor$ is well-threaded when there
%       is a well-formed annotation $\mathcal{A}$ of $\chor$ (i.e. the
%       result of erasing the annotations from $\mathcal{A}$ coincides
%       with $\chor$).
%     \end{itemize}
%     We also say $\chor$ is well-threaded when there is a well-formed annotation $\mathcal A$ of $\chor$ (i.e.: The result of erasing the annotations from $\mathcal A$ coincides with $\chor$).
% \end{definition}


The \emph{Well-threadedness} condition ensures global specifications
are free from unrealisable dependencies among actions. We say $\CAL{A}$
is well-threaded if it is connected and it has a consistent
annotation. 



% \begin{definition}[Mergeability]
\paragraph{Mergeability}
 Annotations in a choreography allow for the extraction of threads
 directly from the global behaviour. As threads are 
 sequences of actions to be executed at each end-point, we need to ensure that
 threads generated from choreographical annotations are meaningful, in
 the sense that they project only to the required end-points, and
 threads describing the behaviour of the same end point are
 encapsulated (\emph{merged})  on a single service description. 
  Mergeability, denoted by $\mergeable$, is the smallest equivalence
  over typed terms up to $\equiv$, closed under all typed contexts and

\begin{align*}
& \myruleg{M-Sel} 
   { 
	\forall i \in (J \cap H). (P_i \mergeable Q_i) \quad \forall j \in J\backslash H. \forall h \in H\backslash J. \mathsf{l}_j \neq \mathsf{l}_h
    }
    { \branching{k}{\Sigma_{j \in J} \mathsf{l_j} \pfx P_j}  \mergeable \branching{k}{\Sigma_{h \in H} \mathsf{l_h} \pfx P_h}
    } \quad 
%  \myruleg{M-In} 
%  { 
% 	x = y \quad P \mergeable Q
%  }
%  { \receive{k}{x} P \mergeable \receive{k}{y} Q   }
 \myruleg{M-Zero} 
 { 
	fsc(P) = 0
 }
 { P \mergeable \INACT   }
\end{align*}
When $P \mergeable Q$, we say that $P$ and $Q$ are mergeable.

% \end{definition}


Above, a context is any end-point calculus process with some
holes. $\Did{M-Sel}$ is for branching and says that we can allow
differences in branches which do not overlap, but we do demand each
pair of behaviours with the same operation to be identical.


The operation $P \merge Q$ allows for merging typed processes as long
as they are mergeable according to the rules above. $P \merge Q$ is a
partial commutative binary operator on typed processes which is
well-defined iff $P \mergeable Q$. We see an example of the merging
rules, and the full set can be consulted in Appendix
\ref{Logic4Struct:appendix:EPP-merging}. The merging of two branching
processes $\branching{k}{\Sigma_{i \in I} \mathsf{l_i} \pfx P_i} $
and $\branching{k}{\Sigma_{i \in J}}$ is given as:
\begin{align*}
	\branching{k}{\Sigma_{i \in I} \mathsf{l_i} \pfx P_i}\merge \branching{k}{\Sigma_{i \in J} \mathsf{l_i} \pfx Q_i} & \DEFEQ \branching{k}{ \begin{array}{l} \Sigma_{i \in I\cap J} \mathsf{l_i} \pfx P_i \merge Q_i \\ +\Sigma_{i \in I \backslash J} \mathsf{l_i} \pfx P_i  \\ +\Sigma_{i \in  J \backslash I} \mathsf{l_i} \pfx Q_i \end{array}  }
\end{align*}
That is, the resulting merge  groups in a single session branching all the
options coming from multiple branches that have the same session key.





Given a consistent annotation, we can project each of its threads onto
an end-point process. The thread
projection $\tp{\CAL{A}}{t}$ is a partial operation that uses the
merge operator, some of the rules are given below (the full set are
included in Appendix \ref{Logic4Struct:appendix:EPP-thread-proj}):


{\small
\[
\tp{\init {A^{t_1}}{B^{t_2}}{b}{\tilde k}\pfx\mathcal A}{t}
    \DEFEQ
  \quad  \left\{ 
      \begin{array}{ll}
        \initOut{b}{\tilde k}\pfx\tp{\mathcal A}{t_1}
        &         
        \textrm{if}\ t=t_1\\
        \repInitIn{b}{\tilde k}\pfx\tp{\mathcal A}{t_2}
        & 
        \textrm{if}\ t=t_2\\
        \tp{\mathcal A}{t}
        & 
        \text{otherwise}
      \end{array}
    \right.
\]

\[\tp{\choice {A^{t_1}}{B^{t_2}}{k}{l}{\CAL{A}}}{t}
    \DEFEQ
    \quad\left\{ 
      \begin{array}{ll}
        \selection {k} {l_i}\tp{\mathcal A_i}{t}
       &                 \textrm{if}\ t=t_1
\\
        \branching k {\sum_i l_i}\pfx\tp{\mathcal A_i}{t}
        & 
        \textrm{if}\ t=t_2\\
        \tp{\mathcal A}{t}
        & 
        \text{otherwise}
      \end{array}
    \right.
\]

\[ \tp{\itn{e@A^{t'}}{\mathcal A_1}{\mathcal A_2}}{t}
    \DEFEQ
   \quad    \left\{ 
      \begin{array}{ll}
        \itn{e}{\tp{\mathcal A_1}{t'}}{\tp{\mathcal A_2}{t'}}
        &         
        \textrm{if}\ t=t'\\
        \tp{\mathcal A_1}{t}\merge\tp{\mathcal A_2}{t}
        & 
        \text{otherwise}
      \end{array}
    \right.
\]
}









\begin{definition}[Coherent Interactions]
  Given a well-threaded, consistently annotated interaction $\CAL{A}$,
  we say that $\CAL{A}$ is coherent if the following two conditions
  hold:
  \begin{enumerate}
    \item For each thread $t$ in $\CAL{A}$, $TP(\CAL{A},t)$ is
      well-defined. 
    \item For each pair of threads $t_1,t_2$ in $\CAL{A}$ with
        $t_1 \equiv_A t_2$, we have $TP(A,t_1) \mergeable
        TP(\CAL{A}, t_2)$. 
      \end{enumerate}
\end{definition}

Below, 
% we say $\chor$ is restriction-free whenever it contains no
% terms of the form $\new{a}\chor'$ as its subterm. Additionally, 
$part(\chor)$ denotes the set of participants names occurring in
$\chor$. Recall also being coherent entails being well-typed,
connected and well-threaded.


\begin{definition}[End-Point Projection]
\label{Logic4Struct:def:epp}
% Let $\chor$ be a coherent interaction such that $\chor\LITEQ \new {\tilde
%   s}\chor'$ where $\chor'$ is restriction-free. Let $\mathcal A$ be a
% consistent annotation of $\chor'$.  
Let $\chor$ be a coherent interaction, and %  such that $\chor\LITEQ \new {\tilde
%   s}\chor'$
%  where $\chor$ is restriction-free. 
 $\mathcal A$ be a
consistent annotation of $\chor$.  
% Then the {\em end point projection of
%   $\CAL{A}$ under a state $\sigma$}, denoted $\epp{\new{\tilde
%     s}\CAL{A}}{\sigma}$, is given as the following network.
Then the {\em end point projection of
  $\CAL{A}$ under a state $\sigma$}, denoted $\epp{\CAL{A}}{\sigma}$, is given as the following network.
\[
\epp{\CAL{A}}{\sigma} \DEFEQ
\Pi_{A \in part(\chor)}\ 
\pr
{A}
{\Pi_{[t]}\bigsqcup_{t'\in[t]}\tp{\mathcal A}{t'}}
{\sigma@A}
\]
\end{definition} 

The mapping given above is defined after choosing a specific
annotation of an interaction. The following result shows the map in
fact does not depend on a specific (consistent) annotation chosen, as
far as a global description has no incomplete threads, i.e. it has no
free session channels (which is what programmers/designers usually
produce).

\begin{theorem}[Soundness and Completeness of End-point Projections\cite{carbone7scc}]
  \label{Logic4Struct:theorem:epp}
  Assume $\mathcal A$ is well-typed, strongly connected, well-threaded
  and coherent. Assume further $\typerule{\typeEnv}{}{\mathcal A \ccc
    \Delta}$ and $\typerule{\typeEnv}{}{\sigma}$. Then the following
  properties hold: 

\begin{itemize}
  \item (soundness) if $\epp{\mathcal{A}}{\sigma}\action{}  \network$
    then there exists $\mathcal A'$ such that $(\sigma, \mathcal{A}) \action{}
(\sigma', \mathcal{A'})$ such that $\epp{\mathcal{A'}}{\sigma'} \prune \equiv_{rec} \network$.

 \item (completeness) If $(\sigma, \mathcal{A}) \action{}  (\sigma', \mathcal{A'})$ then there exist $\network$ such
   that $\epp{\mathcal{A}}{\sigma} \action{} \network$ and $\epp{\mathcal{A'}}{\sigma'} \prune \network$.

  \item (soundness with action labels) if $\epp{\mathcal{A}}{\sigma}\action{\emm}  \network$
    then there exists $\mathcal A'$ such that $(\sigma, \mathcal{A}) \action{\ell}
(\sigma', \mathcal{A'})$ such that $\epp{\mathcal{A'}}{\sigma'} \prune \equiv_{rec} \network$.

 \item (completeness with action labels) If $(\sigma, \mathcal{A}) \action{\ell}  (\sigma', \mathcal{A'})$ then there exist $\network$ such
   that $\epp{\mathcal{A}}{\sigma} \action{\emm} \network$ and $\epp{\mathcal{A'}}{\sigma'} \prune \network$.

\end{itemize}
\end{theorem}

Where $\equiv_{rec}$ denotes equality induced by the unfolding of
process recursion. The asymmetric relation $P \prune Q$ indicates that $P$ is the result
of cutting off ``unnecessary branches'' of $Q$, in the light of P's
own typing, is formally defined as follows:


\begin{definition}[Pruning]\label{Logic4Struct:definition:pruning}
 Let $\typeEnv |-_{A} P |> \Delta$ for $\typeEnv$ and $\Delta$
minimal and $\typeEnv, \typeEnv' |-_{A} Q |> \Delta$. If further we
have $Q \equiv Q_0 \pp !R$ where $\typeEnv |- Q_0 |> \Delta$, $\typeEnv'
|-_{A} R |> \Delta$ and $P \mergeable Q_0$, then we can write: $\typeEnv |-_A P \prune Q |>
\Delta$ or $P \prune Q$ for short, and say $P$
prunes $Q$ under $\typeEnv; \Delta$. $\prune$ is extended to networks accordingly.
\end{definition}


%%% Local Variables: 
%%% mode: latex
%%% TeX-master: "../Thesis"
%%% End: 



%%% Local Variables: 
%%% mode: latex
%%% TeX-master: "../Thesis"
%%% End: 
