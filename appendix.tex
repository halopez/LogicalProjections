% \section{Global Calculus: Reduction Semantics}
% \label{Logic4Struct:appendix:GlobalC-Rsemantics}

% The reduction semantics of the Global Calculus is defined by the rules
% in Figure \ref{Logic4Struct:table:global:reduction-semantics} 

% \begin{myfigure}{ht!}
% {\small
%   \begin{align*}
% &\myruleg{G-RInit}
% { h \text{ is fresh }  } { ({\sigma,\init {A}{B}{a}{k}\pfx \chor})  
%           -->
%       ({\sigma, \chor[h/k]}) }
% \quad \myruleg{G-RStruct}
% {\chor\equiv \chor'' \quad
%       ({\sigma,\chor})  -->  ({\sigma', \chor'}) \quad
%       \chor'\equiv \chor'''} {({\sigma,\chor''})  --> 
%       ({\sigma',\chor'''})}
%    \\[0.3cm]&
% % \myruleg{G-RSum}
% % { (\sigma, \chor_i) --> (\sigma', \chor' ) \quad i \in I}
% %     {(\sigma, \sum_{i \in I} \chor_i)
% %       --> ({\sigma',\chor'})}
% % ~ 
% \qquad \qquad \myruleg{G-RRec}
% { (\sigma, \chor [\rec X. \chor/\chor]) --> (\sigma', \chor' ) }
%     {(\sigma, \rec X. \chor)
%       --> ({\sigma',\chor'})}
% \qquad 
% \myruleg{G-RPar}
% {({\sigma,\chor_1}) --> 
%       ({\sigma',\chor_1'})} {({\sigma,\chor_1\pp \chor_2})
%        -->  ({\sigma',\chor_1'\pp \chor_2})}
% % ~
% % \myruleg{G-RRes}
% % {({\sigma,\chor}) --> ({\sigma',\chor'})}
% %       {({\sigma, \new{k}\chor})
% %        -->  ({\sigma', \new{k}\chor'})}
%    \\[0.3cm]&
% \myruleg{G-RIfT}
% {\sigma(e@A)\Downarrow \true} { (\sigma,
%       \itn{e@A}{\chor_1}{\chor_2})  --> (\sigma, \chor_1) }
% \quad
% \myruleg{G-RIfF}
% {\sigma(e@A)\Downarrow \false } { (\sigma,
%       \itn{e@A}{\chor_1}{\chor_2})  --> (\sigma,\chor_2)}
% \end{align*} \vspace{-0.5cm}\begin{align*}
% \myruleg{G-RCom}
% {\sigma(e@A)\Downarrow v} {
%       ({\sigma,\interact{A}{B}{k}{\mathsf{op},e}{x}\pfx \chor})
%        -->  ({\sigma[x@B \mapsto v],\chor}) }
%  \end{align*}}
%   \caption{Reduction Semantics for the Global Calculus}
%   \label{Logic4Struct:table:global:reduction-semantics}
% \end{myfigure}


\section{Global Calculus: Typing Rules}
\label{Logic4Struct:appendix:global:typing}

The global typing judgments are triples $\typeEnv |- \chor  : \Delta$
inductively defined by the typing rules in Figure \ref{Logic4Struct:fig:global-types}.


\begin{myfigure}{ht!}
{\small
  \begin{align*}
    &\qquad \qquad \qquad \myrule{    G-TInit}
    { \typerule{\typeEnv, a@B : (\vec{k})\alpha  }{}{
        \chor  \ccc \Delta \cdot \vec{k}[B,A]: \alpha \quad A \neq B} } { \typerule{\typeEnv, a@B : (\vec{k})\alpha  }{}{
        \init{A}{B}{a}{\vec{k}}\pfx \chor  \ccc \Delta}}
%     \quad
%     \myrule{    G-TSum}
%     { \typerule{\typeEnv  }{}{
%         \chor_1   \ccc \Delta}
%       \quad
%        \typerule{\typeEnv  }{}{
%         \chor_2   \ccc \Delta}
%     } { \typerule{\typeEnv  }{}{
%         \chor_1 + \chor_2 \ccc \Delta}}
\\ &
    \myrule{    G-TCom}
    { \typerule{\typeEnv  }{}{
        \chor   \ccc \Delta \cdot \vec{k} [A,B] :\alpha}
      \quad \typerule{\typeEnv}{}{e@A:\theta}
      \quad \typerule{\typeEnv}{}{x@B:\theta}
      \quad k \in \vec{k}
      \quad A \neq B 
    } { \typerule{\typeEnv  }{}{
        \interact{A}{B}{k}{e}{x} \pfx \chor \ccc \Delta \cdot
        \vec{k}[A,B]: k \uparrow \theta \pfx \alpha}}
\\ &
    \myrule{    G-TComInv}
    { \typerule{\typeEnv  }{}{
        \chor   \ccc \Delta \cdot \vec{k} [B,A] :\alpha}
      \quad \typerule{\typeEnv}{}{e@A:\theta}
      \quad \typerule{\typeEnv}{}{x@B:\theta}
      \quad k \in \vec{k}
      \quad A \neq B
%       \quad j \in J
    } { \typerule{\typeEnv  }{}{
        \interact{A}{B}{k}{e}{x} \pfx \chor \ccc \Delta \cdot
        \vec{k}[B,A]: k \downarrow \theta \pfx
        \alpha}}
\\ & \qquad \qquad \qquad 
    \myrule{    G-TChoice}
    { \typerule{\typeEnv  }{}{
        \chor_j   \ccc \Delta \cdot \vec{k} [A,B] :\alpha_j}
     \quad k \in \vec{k}
      \quad A \neq B 
      \quad j \in I
    } { \typerule{\typeEnv  }{}{
        \choice{A}{B}{k}{l}{\chor} \ccc \Delta \cdot
        \vec{k}[A,B]: \&\{l_i: \alpha_i \}_{i \in I}}}
\\ &\qquad \qquad \qquad 
    \myrule{    G-TChoiceInv}
    { \typerule{\typeEnv  }{}{
        \chor_j   \ccc \Delta \cdot \vec{k} [B,A] :\alpha_j}
     \quad k \in \vec{k}
      \quad A \neq B 
      \quad j \in I
    } { \typerule{\typeEnv  }{}{
        \choice{A}{B}{k}{l}{\chor} \ccc \Delta \cdot
        \vec{k}[B,A]: \oplus \{ l_i : \alpha_i \}_{i \in I} }}
\\ &
    \myrule{    G-Tpar}
    { \typerule{\typeEnv  }{}{
        \chor_1   \ccc \Delta_1}
      \quad
       \typerule{\typeEnv  }{}{
        \chor_2   \ccc \Delta_2}
    } { \typerule{\typeEnv  }{}{
        \chor_1 \pp \chor_2 \ccc \Delta_1 \bullet \Delta_2}}
    \quad 
    \myrule{    G-TIf}
    { \typerule{\typeEnv  }{}{
        e@A: \mathtt{bool}}
      \quad
       \typerule{\typeEnv  }{}{
        \chor_1   \ccc \Delta}
      \quad
       \typerule{\typeEnv  }{}{
        \chor_2   \ccc \Delta}
    } { \typerule{\typeEnv  }{}{
        \itn{e@A}{\chor_1}{\chor_2} \ccc \Delta}}
% \\ &
%     \myrule{    G-TRes1}
%     { \typerule{\typeEnv  }{}{
%         \chor   \ccc \Delta, \vec{k_1} k \vec{k_2}[A,B]:\alpha}
%     } { \typerule{\typeEnv  }{}{
%         \new{k} \chor \ccc \Delta, \vec{k_1}\vec{k_2} : \bot}}
%     \quad
%     \myrule{    G-TRes2}
%     { \typerule{\typeEnv  }{}{
%         \chor   \ccc \Delta, \vec{k_1} k \vec{k_2}:\bot}
%     } { \typerule{\typeEnv  }{}{
%         \new{k} \chor \ccc \Delta, \vec{k_1}\vec{k_2} : \bot}}
%     \quad
%     \myrule{    G-TRes3}
%     { \typerule{\typeEnv  }{}{
%         \chor   \ccc \Delta, \epsilon:\bot}
%     } { \typerule{\typeEnv  }{}{
%         \new{k} \chor \ccc \Delta, \vec{k_1}\vec{k_2} : \bot}}
\\ &
    \myrule{    G-TRec}
    { \typerule{\typeEnv \cdot X : \Delta  }{}{
        \chor   \ccc \Delta}
    } { \typerule{\typeEnv  }{}{
        \mu X \pfx \chor \ccc \Delta}}
    \quad
    \myrule{    G-TVar}
    { \typeEnv, X : \Delta \text{ well formed}  
    } { \typerule{\typeEnv, X:\Delta  }{}{
        X \ccc \Delta}}
    \quad
    \myrule{    G-TZero}
    { \typeEnv \text{ well formed}  \quad \forall i \neq j, \{k_i\}
      \cap \{k_j\} = \emptyset
    } { \typerule{\typeEnv  }{}{
        \INACT \ccc \bigcup_i \vec{k_i} [A_i, B_i]\endT}}
  \end{align*}
}
\caption{Global Calculus: Typing Rules}
\label{Logic4Struct:fig:global-types}
\end{myfigure}



% \section{End-Point Calculus: Reduction Semantics}
% \label{Logic4Struct:appendix-semanticsEPC}

% The reduction semantics for the end-point calculus follows the
% $\pi$-calculus and is defined by the rules in
% Figure~\ref{Logic4Struct:table:endpoint:reduction-semantics}. 
% \begin{myfigure}{ht!}
% \begin{align*}
% &\myruleg{E-RInit}
%   {     k_i\not\in\fsc{P'}\cup\fsc{Q'}   \quad \tilde{h} \text{ is
%       fresh}}
%   {   \pr {A}{\repInitIn{a}{\tilde k}\pfx P\pp P'}\sigma\pp
%    \pr {B}{\initOut{a}{\tilde k}\pfx Q\pp Q'}{\sigma'}    \to \ 
%    % \new  {\tilde k}
%    (
%    \pr {A} {\repInitIn{a}{\tilde k}\pfx P\pp P\pp P'} {\sigma}\pp \pr
%    {B}{Q\pp Q'} {\sigma'}
%    )[\tilde{h}/\tilde{k}]
%   }   
% \\[0.5cm]
% & \qquad \myruleg{ E-RCom}
%   {   \sigma\proves e\converges v   } 
%   {  \pr{A}{\receive{k}{x} P \,|\, P'}{\sigma}
%   \,|\, 
%   \pr{B}{\send{k}{x} Q \,|\, Q'}{\sigma'}
%   \to \ 
%   \pr{A}{P\pp P'}{\sigma[x\mapsto v]}
%   \pp 
%   \pr{B}{Q\pp Q'} {\sigma'}  }
% \\[0.5cm]
% &\myruleg{ E-RSel}
%   {   j \in I   } 
%   {  \pr{A}{ \branching{k}{\sum_i \mathsf{l_i} \pfx P_i} \,|\, P'}{\sigma}
%   \,|\, 
%   \pr{B}{\selection{k}{l_j} Q \,|\, Q'}{\sigma'}
%   \to \ 
%   \pr{A}{P_j\pp P'}{\sigma}
%   \pp 
%   \pr{B}{Q\pp Q'} {\sigma'}  }
% \\[0.5cm]
% &\myruleg{ E-RIfT}
% {  \sigma\vdash e \converges \true } 
% {  \pr{A}{\itn{e}{P_1}{P_2}\,|\, P'}{\sigma}
%   \;\to\; 
%   \pr{A}{P_1\pp P'}{\sigma}}
% \qquad \qquad 
% \myruleg{ E-RPar}
% {  M   \;\to\;    M' } 
% {   M|N   \;\to\;    M'|N} 
% % \quad 
% % \myruleg{ E-RRes}
% % {   M   \;\to\;    M' } 
% % {  \new{k}M   \;\to\;    \new{k}M' } 
% \\[0.5cm]
% &\myruleg{ E-RIfF}
% {  \sigma\vdash e \converges \false } 
% {   \pr{A}{\itn{e}{P_1}{P_2}\,|\, P'}{\sigma}   \;\to\;    \pr{A}{P_2\pp P'}{\sigma}}
% \quad
% \myruleg{ E-RSum}
% { i\in\{1,2\} }  {   A[P_1\oplus P_2|R]_\sigma   \;\to\;
%   A[P_i|R]_\sigma }   
% \end{align*} 
% \begin{align*}
% % \Did{E-Res1} \ &
% % \Rule
% % {
% %   \pr{A}{P}{\sigma}  
% %   \;\to\; 
% %   \pr{A}{P'}{\sigma'}  
% % } 
% % {
% %   \pr{A}{\new{s}P}{\sigma}
% %   \;\to\; 
% %   \pr{A}{\new{s}P'}{\sigma}
% % } 
% % &
% \myruleg{ E-RRec}
% {
%   \pr A {P\MSUBS{\mu X.P}{X}\pp Q}{\sigma}\pp N
%   \;\to\; 
%   N'
% } 
% {
%   \pr A {\mu X.P\pp Q}{\sigma}\pp N
%   \;\to\; 
%   N'
% }
% &
% % \myruleg{ E-RAssign} 
% % {
% %   \sigma\proves e\converges v
% % } 
% % {
% %   \pr{A}{x:=e\pfx P\,|\, P'}{\sigma}
% %   \;\to\; 
% %   \pr{A}{P\pp P'}{\sigma[x\mapsto v]}
% % }
% % \\\\
% % &
% \qquad \myruleg{ E-RStruct}{M\equiv M'\quad M'\to N'\quad N'\equiv N} { M\to N}
% \end{align*}
% % \begin{align*}
% % &
% % % \Did{E-Par2} \ &
% % % \Rule
% % % {
% % %   \pr{A}{P_1\,|\, R}{\sigma}  
% % %   \;\to\; 
% % %   \pr{A}{P'_1\,|\, R}{\sigma'}  
% % % } 
% % % {
% % %   \pr{A}{P_1\,|\,  P_2\,|\, R}{\sigma}  
% % %   \;\to\; 
% % %   \pr{A}{P'_1\,|\,  P_2\,|\, R}{\sigma'}  
% % % } 
% % \end{align*}
% \caption{Reduction Relation for the End-Point Calculus}
% \label{Logic4Struct:table:endpoint:reduction-semantics}
% % \end{align*}

% \end{myfigure}


\section{End-Point Calculus: Typing rules}
\label{Logic4Struct:appendix:EPC-Types}

The typing rules for the End Point Calculus are given in Figure \ref{Logic4Struct:figure:EPC-Types}, where
the compatibility operators $\COH$ and $\odot$
are defined accordingly as:

\begin{myfigure}{t!}
{\small
  \begin{align*}
    &  \myrule{ E-TInit.In} 
  { \typeruleE{\typeEnv }{A}{ P \ccc
      \VEC{k}@A:\alpha} 
    \quad a \not\in dom(\typeEnv) 
    \quad    client(\typeEnv) 
  } { 
    \typeruleE{\typeEnv, a @ A: (\VEC{k}) \alpha }{A}{
      !a(\VEC{k}) \pfx P \ccc \emptyset}
  }
    \quad
    \myrule{ E-TInit.Out} 
  { \typeruleE{\typeEnv, a@B:(\VEC{k})\alpha}{A}{ P \ccc \Delta \cdot
      \VEC{k}@A:\alpha}  } 
  { \typeruleE{\typeEnv, a @B: (\VEC{k})\alpha }{A}{ \overline{a}
      <. \VEC{k} .> P \ccc \Delta}}\\
    &\myrule{    E.TBranch}
    { 
      j  \in J \quad
      J \subseteq I \quad
      k \in \VEC{k} \quad
      \typeruleE{\typeEnv  }{A}{P_j  \ccc \Delta \cdot \VEC{k}@A:\alpha_j}
    } 
    { 
      \typeruleE{\typeEnv  }{A}{\branching{k}{\sum_{i \in I}
          \mathsf{l_i} \pfx P_i}  \ccc \Delta \cdot \VEC{k}@A:k} \rhd
      \& \{ l_i : \alpha_i \}}
    \quad
    \myrule{    E.TVar}
    { 
    } { \typeruleE{\typeEnv, X:\Delta  }{A}{
        X \ccc \Delta}}
\\
    &\myrule{    E.TSel}
    { 
      i,j  \in I \quad
      k \in \VEC{k} \quad
      \typeruleE{\typeEnv  }{A}{P_j  \ccc \Delta \cdot \VEC{k}@A:\alpha_j}
    } 
    { 
      \typeruleE{\typeEnv  }{A}{  \selection{\vec{k}}{ \mathsf{l_i}}
        P   \ccc \Delta \cdot \VEC{k}@A:k} \rhd
      \oplus \{ l_i : \alpha_i \}}
%     ~
% \myrule{    E.TSel}
%     { i  \in I \subseteq J \quad \typeruleE{\typeEnv  }{A}{ 
%         P  \ccc \Delta , \vec{k}@A:\alpha_i}} 
%     { \typeruleE{\typeEnv  }{A}{
%         \selection{\vec{k}}{ \mathsf{l_i}} P  \ccc \Delta
%       , \vec{k}@A: \sum_{j \in J} k \uparrow \mathsf{l}_j (\theta_i)
%     \pfx \alpha_i}}
~
    \myrule{    E.TRec}
    { \typeruleE{\typeEnv , X : \Delta  }{A}{
        P   \ccc \Delta}
    } { \typeruleE{\typeEnv  }{A}{
        \mu X \pfx P \ccc \Delta}}
    ~
        \myrule{    E.TInact}
    { 
    } { \typeruleE{\typeEnv  }{A}{
        \INACT \ccc \emptyset}}
\\
& 
\myrule{    E.TIn}
    { 
      k \in \VEC{k} \quad
      \typeruleE{\typeEnv  }{A}{ 
        P  \ccc \Delta \cdot \VEC{k}@A:\alpha} \quad 
      \typeruleE{\typeEnv  }{}{ x : \theta} 
    } 
    { \typeruleE{\typeEnv  }{A}{
        \receive{k}{x}  P  \ccc \Delta
      \cdot \VEC{k}@A: k \downarrow \theta
    \pfx \alpha}}
~
\myrule{    E.TOut}
    { 
      k \in \VEC{k} \quad
      \typeruleE{\typeEnv  }{A}{ 
        P  \ccc \Delta \cdot \VEC{k}@A:\alpha} \quad 
      \typeruleE{\typeEnv  }{}{ e : \theta} 
    } 
    { \typeruleE{\typeEnv  }{A}{
        \send{k}{e}  P  \ccc \Delta
      \cdot \VEC{k}@A: k \uparrow \theta
    \pfx \alpha}}
\\
    &\myrule{    E.TIf}
    { \typeruleE{\typeEnv  }{}{
        e@A: \mathtt{bool}}
      \quad
       \typeruleE{\typeEnv  }{A}{
        P   \ccc \Delta }
      \quad
       \typeruleE{\typeEnv  }{A}{
        Q   \ccc \Delta}
    } { \typeruleE{\typeEnv  }{}{
        \itn{e@A}{P}{Q} \ccc \Delta}}
    \quad
    \myrule{    E.TSum}
    { \typeruleE{\typeEnv  }{A}{
        P   \ccc \Delta}
      \quad
       \typeruleE{\typeEnv  }{A}{
        Q   \ccc \Delta}
    } { \typeruleE{\typeEnv  }{A}{
        P \oplus Q \ccc \Delta}}
\\
      &
%     \myrule{    E.TRes1}
%     { \typeruleE{\typeEnv  }{A}{
%         P   \ccc \Delta, \VEC{k_1} k \VEC{k_2}: \bot}
%     } { \typeruleE{\typeEnv  }{A}{
%         \new{k} P \ccc \Delta, \VEC{k_1} \VEC{k_2} : \bot}}
%     \,
%     \myrule{    E.TRes2}
%     { \typeruleE{\typeEnv  }{A}{
%         P   \ccc \Delta,\varepsilon: \bot}
%     } { \typeruleE{\typeEnv  }{A}{
%         \new{k} P \ccc \Delta, \VEC{k_1} \VEC{k_2} : \bot}}
%     \,
    \myrule{    E.TEnd}
    { \typeruleE{\typeEnv  }{A}{
        P   \ccc \Delta}
     \quad \{\VEC{k} \} \cap \fsc{\Delta} = \emptyset
    } { \typeruleE{\typeEnv  }{A}{
        P \ccc \Delta,  \VEC{k}@A : \endT}}
%    \\&
    \,
    \myrule{    E.TPar}
    { \typeruleE{\typeEnv_1  }{A}{
        P   \ccc \Delta_1}
      \quad
      \typeruleE{\typeEnv_2  }{A}{
        Q   \ccc \Delta_2} 
      \quad \typeEnv_1 \COH \typeEnv_2
      \quad \Delta_1 \COH \Delta_2
    } { \typeruleE{\typeEnv_1 \odot \typeEnv_2 }{A}{
        P \pp Q \ccc \Delta_1 \odot \Delta_2}}
    \\&
    \myrule{    E.TBot}
    { \typeruleE{\typeEnv  }{A}{
        P   \ccc \Delta}
     \quad \{ \VEC{k} \} \cap \fsc{\Delta} = \emptyset
    } { \typeruleE{\typeEnv  }{A}{
        P \ccc \Delta,  \VEC{k} : \bot}}
    ~
        \myrule{    E.TPart}
    { \typeruleE{\typeEnv  }{A}{
        P   \ccc \Delta}
     ~
     \typeruleE{\typeEnv  }{}{
        \sigma@A}
    } { \typeruleE{\typeEnv  }{A}{
        \pr{A}{P}{\sigma} \ccc \Delta}}
    ~
            \myrule{    E.TInactNW}
    { 
    } { \typeruleE{\typeEnv  }{A}{
        \INACTNW \ccc \emptyset}}
    \\ &
        \myrule{    E.TBotN}
    { \typeruleE{\typeEnv  }{}{
        \network   \ccc \Delta} \quad \VEC{k} \cap fsc(\Delta) = \emptyset
    } { \typeruleE{\typeEnv }{}{
        \network \ccc \Delta \cdot \VEC{k} : \bot}}
    \qquad \qquad 
    \myrule{    E.TEndN}
    { \typeruleE{\typeEnv  }{}{
        \network   \ccc \Delta}
     \, \{\VEC{k} \} \cap \fsc{\Delta} = \emptyset
    } { \typeruleE{\typeEnv  }{A}{
        \network \ccc \Delta,  \VEC{k}@A : \endT}}
    ~
%     \myrule{    E.TResN1}
%     { \typeruleE{\typeEnv  }{}{
%         \network   \ccc \Delta, \VEC{k_1} k \VEC{k_2}: \bot}
%     } { \typeruleE{\typeEnv  }{}{
%         \new{k} \network \ccc \Delta, \VEC{k_1} \VEC{k_2} : \bot}}
%     \quad
%     \myrule{    E.TResN2}
%     { \typeruleE{\typeEnv  }{}{
%         \network   \ccc \Delta,\varepsilon: \bot}
%     } { \typeruleE{\typeEnv  }{}{
%         \new{k} \network \ccc \Delta, \VEC{k_1} \VEC{k_2} : \bot}}
\\ & \qquad \qquad \qquad 
    \myrule{    E.TParN}
    { \typeruleE{\typeEnv_2  }{}{
        \network_1   \ccc \Delta_1}
      \quad
      \typeruleE{\typeEnv_1  }{}{
        \network_2   \ccc \Delta_2}
      \quad \typeEnv_1 \COH \typeEnv_2\quad \Delta_1 \COH \Delta_2
    } { \typeruleE{\typeEnv_1 \odot \typeEnv_2  }{}{
        \network_1 \pp \network_2 \ccc \Delta_1 \odot \Delta_2}}
\end{align*}
\caption{End Point Calculus: Typing rules}
\label{Logic4Struct:figure:EPC-Types}}

\end{myfigure}


  \begin{enumerate}
  \item Two service typings $\Gamma_1$ and $\Gamma_2$ are compatible
    (written $\Gamma_1\COH\Gamma_2$) if they satisfy the following
    conditions:
      \begin{enumerate}

      \item if $a@A\in\textsf{dom}(\Gamma_i)$ then
        $a@B\not\in\textsf{dom}(\Gamma_{j})$ for
        every $B$ and for $i \neq j$;

      \item if $\ol{a}@A\in\textsf{dom}(\Gamma_{i})$ and
        $\ol{a}@B\in\textsf{dom}(\Gamma_{j})$ then $A=B$ and
        $\Gamma_i(\ol{a}@A)=\Gamma_{j}(\ol{a}@B)$ for $i \neq j$ (up
        to $\alpha$-renaming of bound names);

      \item if $a@A\in\textsf{dom}(\Gamma_i)$ and
        $\ol{a}@B\in\textsf{dom}(\Gamma_{j})$ then
        $A=B$ and $\Gamma_i({a}@A)= \ol{ \Gamma_{j}( \ol{a}@B) }$ for $i \neq j$  (up to $\alpha$-renaming of bound
        names);
      \item $\Gamma_1(x)=\Gamma_2(x)$ for each $x$ in $\Gamma_{1,2}$;
      \item $\Gamma_1(X)=\Gamma_2(X)$ for each $X$ in $\Gamma_{1,2}$.
      \end{enumerate}

    \item Two session typings $\Delta_1$ and $\Delta_2$ are compatible
      (written $\Delta_1\COH\Delta_2$) if they satisfy the following
      conditions:
      \begin{enumerate}
      \item if $\VEC{k}\in\mathsf{dom}(\Delta_1)$,
        $\VEC{t}\in\mathsf{dom}(\Delta_2)$ and
        $\VEC{k}\cap\VEC{t}\not=\emptyset$ then $\VEC{k}=\VEC{t}$;
      \item if $\VEC{k}\!:\!\bot\in\Delta_i$ then
        $\VEC{k}\not\in\mathsf{dom}({\Delta_{j}})$ for $i \neq j$;
      \item if $\VEC{k}@A\!:\!\alpha_1$ in $\Delta_1$ and
        $\VEC{k}@A\!:\!\alpha_2$ in $\Delta_2$ then
        $\fsc{\alpha_1}\cap\fsc{\alpha_2}=\emptyset$;
      \item if $\VEC{k}@A\!:\!\alpha_1$ in $\Delta_1$ and
        $\VEC{k}@B\!:\!\alpha_2$ in $\Delta_2$ then
        $\alpha_1 = \ol{ \alpha_2 }$ (for $A\not=B$).
      \end{enumerate}


  \item $\Gamma_1\odot\Gamma_2$, defined whenever
    $\Gamma_1\COH\Gamma_2$, is the minimum service typing such that:
    \begin{enumerate}
    \item if ${a}@A:\alpha\in\Gamma_{i}$ % and
      % $\ol{a}@A\in\mathsf{dom}(\Gamma_{j})$
      then ${a}@A:\alpha\in \Gamma_1\odot\Gamma_2$;
    \item if $\ol{a}@A:\alpha\in\Gamma_{i}$ and
      ${a}@A:\alpha\not\in\Gamma_{j}$ then for $i \neq j$,
      $\ol{a}@A:\alpha\in \Gamma_1\odot\Gamma_2$;
    \item if $x@A:\theta\in\Gamma_i$ ($X:\Delta\in\Gamma_i$) then
      $x@A:\theta\in\Gamma_1\odot\Gamma_2$
      ($X:\Delta\in\Gamma_1\odot\Gamma_2$).
    \end{enumerate}


  \item $\Delta_1\odot\Delta_2$, defined whenever
    $\Delta_1\COH\Delta_2$, is the minimum session typing such that:
    \begin{enumerate}
    \item if $\tilde
      k@A\in\mathsf{dom}(\Delta_i)\backslash\mathsf{dom}(\Delta_{j})$
      for $i \neq j$, then $\tilde k@A:\Delta(\tilde
      k@A)\in\Delta_1\odot\Delta_2$;
    \item if $\tilde
      k\in\mathsf{dom}(\Delta_i)\backslash\mathsf{dom}(\Delta_{j})$ for $i \neq j$,
      then $\tilde k:\perp\in\Delta_1\odot\Delta_2$;
    \item if $\tilde k@A:\alpha\in\Delta_i$ and $\tilde
      k@A:\beta\in\Delta_{j}$ for $i \neq j$, then $\tilde
      k@A:\alpha\pp\beta\in\Delta_1\odot\Delta_2$;
      % for
      % ($\alpha\not=\ol\beta$).

    \item if $\tilde k@A\in\mathsf{dom}(\Delta_i)$ and $\tilde
      k@B\in\mathsf{dom}(\Delta_{j})$ for $i \neq j$, then $\tilde
      k:\perp\in\Delta_1\odot \Delta_2$.
    \end{enumerate}
  \end{enumerate}




\section{End Point Projection: Merging}
\label{Logic4Struct:appendix:EPP-merging}


\begin{definition}[Merge Operator]

  $P \merge Q$ is a partial commutative binary operator on typed
  processes which is well-defined iff $P \mergeable Q$ and satisfies
  the rules in Figure \ref{Logic4Struct:figure:EPP-merging}, where,
  in the right-hand side of each rule, we assume that each application
  of the operator to, say, $P$ and $Q$, is such that $P \mergeable Q$.


\begin{myfigure}{t!}
\begin{align*}
	\repInitIn{a}{k} \pfx P \merge \repInitIn{a}{k} \pfx Q & \DEFEQ  \repInitIn{a}{k} \pfx (P \merge Q)\\
	\initOut{a}{k} \pfx P \merge \initOut{a}{k} \pfx Q & \DEFEQ  \initOut{a}{k} \pfx (P \merge Q)\\
	\send{k}{x} P \merge \send{k}{x} Q & \DEFEQ \send{k}{x}  (P \merge Q) \\
	\receive{k}{y}P \merge \receive{k}{y} Q & \DEFEQ \receive{k}{y}  (P \merge Q)	\\
	\branching{k}{\Sigma_{i \in I} \mathsf{l_i} \pfx P_i}\merge \branching{k}{\Sigma_{i \in J} \mathsf{l_i} \pfx Q_i} & \DEFEQ \branching{k}{ \begin{array}{l} \Sigma_{i \in I\cap J} \mathsf{l_i} \pfx P_i \merge Q_i \\ +\Sigma_{i \in I \backslash J} \mathsf{l_i} \pfx P_i  \\ +\Sigma_{i \in  J \backslash I} \mathsf{l_i} \pfx Q_i \end{array}  }\\
	\selection{k}{l} P \merge \selection{k}{l} Q & \DEFEQ \selection{k}{l}(P \merge Q)\\
	\itn{e}{P_1}{P_2} \merge \itn{e}{Q_1}{Q_2} & \DEFEQ \itn{e}{(P_1 \merge Q_1)}{(P_2 \merge Q_2)}\\
	(P_1 \pp P_2) \merge (P_3 \pp P_4)  & \DEFEQ (P_1 \merge P_3) \pp (P_2 \merge P_4)\\
	(P_1 \oplus P_2) \merge (Q_1 \oplus Q_2)  & \DEFEQ (P_1 \merge Q_1) \oplus (P_2 \merge Q_2)\\
	\rec X \pfx P \merge \rec X \pfx Q  & \DEFEQ \rec X \pfx (P \merge  Q )\\
	X \merge X  & \DEFEQ X\\
	P \merge \INACT  & \DEFEQ P\\
	P \merge Q  & \DEFEQ  P' \merge Q' \qquad (P \equiv P', Q \equiv Q')
\end{align*}
\caption{End-Point Projection: Merging Rules}
\label{Logic4Struct:figure:EPP-merging}
\end{myfigure}
\end{definition}

\section{End Point Projection: Thread Projection}
\label{Logic4Struct:appendix:EPP-thread-proj}

\begin{definition}[Thread Projection]\label{Logic4Struct:def:tp}\rm 
  Given a consistently annotation $\mathcal{A}$, the
  partial operation $\tp{\mathcal{A}}{t}$ is defined as:
  {\small \begin{itemize}
\item[]
    $\tp{\init {A^{t_1}}{B^{t_2}}{b}{\tilde k}\pfx\mathcal A}{t}
    \DEFEQ$
$  \quad  \left\{ 
      \begin{array}{ll}
        \initOut{b}{\tilde k}\pfx\tp{\mathcal A}{t_1}
        &         
        \textrm{if}\ t=t_1\\
        \repInitIn{b}{\tilde k}\pfx\tp{\mathcal A}{t_2}
        & 
        \textrm{if}\ t=t_2\\
        \tp{\mathcal A}{t}
        & 
        \text{otherwise}
      \end{array}
    \right.
$\\\\

\item[] $\tp{\interact {A^{t_1}}{B^{t_2}}{k}{e}{x} \pfx\mathcal  A}{t}
    \DEFEQ$
$\quad\left\{ 
      \begin{array}{ll}
        \outputP k {e}\pfx\tp{\mathcal A}{t}
       &                 \textrm{if}\ t=t_1
\\
        \inputP k {x}\pfx\tp{\mathcal A}{t}
        & 
        \textrm{if}\ t=t_2\\
        \tp{\mathcal A}{t}
        & 
        \text{otherwise}
      \end{array}
    \right.
$\\\\
%  \end{displaymath}
%  \begin{displaymath}
\item[] $\tp{\choice {A^{t_1}}{B^{t_2}}{k}{l}{\CAL{A}}}{t}
    \DEFEQ$
$\quad\left\{ 
      \begin{array}{ll}
        \selection {k} {l_i}\tp{\mathcal A_i}{t}
       &                 \textrm{if}\ t=t_1
\\
        \branching k {\sum_i l_i}\pfx\tp{\mathcal A_i}{t}
        & 
        \textrm{if}\ t=t_2\\
        \tp{\mathcal A}{t}
        & 
        \text{otherwise}
      \end{array}
    \right.
$\\\\

\item[]     
$\tp{\itn{e@A^{t'}}{\mathcal A_1}{\mathcal A_2}}{t}
    \DEFEQ$
$\quad    \left\{ 
      \begin{array}{ll}
        \itn{e}{\tp{\mathcal A_1}{t'}}{\tp{\mathcal A_2}{t'}}
        &         
        \textrm{if}\ t=t'\\
        \tp{\mathcal A_1}{t}\merge\tp{\mathcal A_2}{t}
        & 
        \text{otherwise}
      \end{array}
    \right.
$\\
%  \end{displaymath}
%  \begin{displaymath}
% \item[] 
%     $\tp{x@A^{t'}:=e\pfx\mathcal  A}{t}
%     \DEFEQ$
% $\quad    \left\{ 
%       \begin{array}{ll}
%         x:=e\pfx\tp{\mathcal  A}{t'}
%         &         
%         \textrm{if}\ t=t'\\
%         \tp{\mathcal A}{t}
%         & 
%         \text{otherwise}
%       \end{array}
%     \right.
%  $\\

% \item[] 
% $\tp{\mathcal A_1+^{t'}\mathcal A_2}{t}
%     \DEFEQ$
% $\quad\left\{ 
%       \begin{array}{ll}
%         \tp{\mathcal A_1}{t'}\oplus\tp{\mathcal A_2}{t'}
%         &         
%         \textrm{if}\ t=t'\\
%         \tp{\mathcal A_1}{t}\merge\tp{\mathcal A_2}{t}
%         & 
%         \text{otherwise}
%       \end{array}
%     \right.
% $\\

\item[] $\tp{\mathcal A_1 |^{t'}\mathcal A_2}{t} \DEFEQ
  \tp{\mathcal A_1}{t'}\pp\tp{\mathcal A_2}{t'} $\\

%  \end{displaymath}
%  \begin{displaymath}
% \item 
%     $\tp{{\mathcal A_1}+^{t'}{\mathcal A_2}}{t}
%     \DEFEQ$

% $    \left\{ 
%       \begin{array}{ll}
%         \tp{\mathcal A_1}{t'}\oplus\tp{\mathcal A_2}{t'}
%         &         
%         \textrm{if}\ t=t'\\
%         \tp{\mathcal A_1}{t}\merge\tp{\mathcal A_2}{t}
%         & 
%         \text{otherwise}
%       \end{array}
%     \right.
% $
%  \end{displaymath}

\item[] 
  $\tp{\rec^{t':\ASET{\VEC{t_i}}} X^A\pfx\mathcal A}{t}
  \DEFEQ\rec X\pfx\tp{\mathcal A}{t}$ if
  $t\in\ASET{\VEC{t_i}}$, $\tp{\mathcal A}{t}$ otherwise.\\
\item[] 
  $\tp{X^A_{t:\ASET{\VEC{t_i}}}}{t}
  \DEFEQ X$ if $t\in\ASET{\VEC{t_i}}$, $\INACT$ otherwise.\\
\item[]
  $\tp{\INACT}{t}                                           
  \DEFEQ  \INACT$.
\end{itemize}
If $\tp{\mathcal{A}}{t}$ is undefined then we set
$\tp{\mathcal{A}}{t}=\perp$.}
\end{definition}

Above, we augment consistent annotations with a further annotation for
recursions $\rec^{t:\{t_i\}} X$
% $\rec^t X^A.\mathcal A$
 and recursion variables $X^A
_{t:\{t_i\}}$, with  
% $\rec^t X^A.\mathcal A$ in an annotated interaction, let
 $\{t_i\}$ be the set of threads occurring in,
but not initiated in, $\CAL{A}$ (a thread is initiated in $\CAL{A}$
whenever it occurs passive in a session initiation). 
% Then, we further annotate this recursion as $\rec^{t:\{t_i\}} X$
% and each free $X^A$ in $\mathcal A$ as $X^A _{t:\{t_i\}}$. 

%%% Local Variables: 
%%% mode: latex
%%% TeX-master: "../Thesis"
%%% End: 
